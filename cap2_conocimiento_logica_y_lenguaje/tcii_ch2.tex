% Created 2023-05-25 dj. 16:19
% Intended LaTeX compiler: pdflatex
\documentclass[a4paper, 11pt, twocolumn, spanish]{article}
\usepackage[utf8]{inputenc}
\usepackage[T1]{fontenc}
\usepackage{graphicx}
\usepackage{longtable}
\usepackage{wrapfig}
\usepackage{rotating}
\usepackage[normalem]{ulem}
\usepackage{amsmath}
\usepackage{amssymb}
\usepackage{capt-of}
\usepackage{hyperref}
\usepackage[T1]{fontenc}
\usepackage[margin=.75in]{geometry}
\setlength\parindent{0pt}
\author{Jordi Serra}
\date{\today}
\title{El Helenismo\\\medskip
\large Apuntes de Teoría del Conocimiento ii.\\
Conocimiento, Lógica y Lenguaje}
\hypersetup{
 pdfauthor={Jordi Serra},
 pdftitle={El Helenismo},
 pdfkeywords={},
 pdfsubject={},
 pdfcreator={Emacs 28.2 (Org mode 9.5.5)}, 
 pdflang={English}}
\begin{document}

\maketitle
\tableofcontents


\section{Introducción}
\label{sec:org026d571}
\subsection{Lenguaje y realidad}
\label{sec:orge699822}
Se entiende por lenguaje, en un sentido amplio, cualquier medio de
comunicación entre seres vivientes. Bajo este concepto se incluyen
lenguajes de palabra, medios de comunicación humano de tipo no
lingüístico —el simbolismo del arte, la música, los símbolos
matemáticos, el lenguaje de los sordomudo, etc.—, y comunicación
animal.\\

En un sentido más restringido, el lenguaje es un conjunto de sonidos
portadores de un sentido o significado.\\
El \textbf{significado} es la condición básica del hecho lingüístico.\\
Los \textbf{tres problemas del lenguaje}:
\begin{enumerate}
\item Cómo algo físico —sonidos articulados— puede convertirse en
portador de un significado espiritual o universal.
\item Cómo es posible la comunicación de los significados a través del
lenguaje y qué relación tiene esta forma idiomática de
comunicación con otras formas de comunicación no idiomáticas.
\item Qué relación existe entre las palabras y las cosas que éstas
representan y en qué medida el lenguaje es vehículo del
pensamiento para conocer la realidad —si las palabras son cosas
distintas a las que representan.
\end{enumerate}

El estudio del lenguaje dió un giro radical a partir de los estudios
del lingüista Ferdinand De Saussure, cuando distinguió entre lenguaje
como lengua y como habla.\\
El \textbf{lenguaje como lengua} es un sistema cerrado y autónomo, objeto de
la Lingüística estructural.\\
El \textbf{lenguaje como habla}, es su uso mismo en la comunicación, es decir
algo a alguien sobre algo. Desde esta perspectiva es estudiado por la
Lingüística del discurso que parte de Émile Benveniste (1902-1976).\\

De modo que, por un lado es una estructura, la red secreta que hace
que las cosas se miren en cierta forma unas a otras. Platón había
sostenido que articulamos el mundo en y por el lenguaje, no habiendo
realidad alguna en las cosas singulares sino el logos, red de
significados interrelacionados.\\
Por otro lado, ésta estructura se concreta con ocasión del discurso
hablado y las referencias que en él intervienen. Para Aristóteles, el
logos implica esencialmente referencia al objeto.\\
Hoy decimos que toda lengua es una ordenación abstracta del mundo
producida por la comunidad, que se actualiza en el discurso
individual. A través del discurso se manifiesta el mundo que aparece
como terreno común que todos reconocen y que liga entre sí a todos los
que hablan.

\subsection{El lenguaje como forma de vida}
\label{sec:org8ef068e}
El lenguaje cumple una función simbolizadora constituyente de un
cosmos de objetos que es un mundo de significados actualizados en un
discurso. Nos relacionamos con la realidad a través del
lenguaje. Tenemos realidad porque pertenecemos y vivimos inmersos en
el lenguaje como ámbito envolvente. Wittgenstein: el lenguaje es una
forma de vida. El mundo no se presenta en el lenguaje como un objeto,
sino que el lenguaje revela su sentido en un \textbf{proceso que es, a la
vez, hermenéutico e histórico}.\\

En este sentido, el pensamiento se sustrae de la estructura
lingüística con el proceso de interpretar y corregir sus
interpretaciones. Esto invalida la teoría instrumental del lenguaje,
basada en el principio de que el lenguaje es un instrumento útil para
expresar lo pensado. Pero las palabras no son recipientes
prefabricados para archivar en ellos las ideas. El mismo pensar es
lingüístico, funciona como lenguaje. Tenemos un ejemplo en el deseo de
un lenguaje universal de signos y de signos artificiales definidos
unívocamente.

\subsection{Conocimiento y lenguaje en la obra de Wittgenstein}
\label{sec:org156dcbd}
La obra de Wittgenstein (W.) es un punto de referencia básico para la
comprensión de la filosofía contemporánea en su conjunto. Su
pensamiento tiene \textbf{dos etapas} diferenciadas por sus dos obras más
importantes: (\emph{i}) el \emph{tractatus logico-philosophicus} (1921) y (\emph{ii}) las
\emph{Investigaciones filosóficas} (1935-1945).

\subsubsection{Filosofía y ciencia}
\label{sec:org1e48bdc}
En el \emph{tractatus logico-philosophicus} (1921) presenta las nociones
básicas que tendrán una gran influencia en el círculo de Viena y de su
filosofía neopositivista. Esto es, solo los enunciados formales de la
matemática y la lógica, los enunciados de las ciencias empíricas,
pueden tener sentido. Todos los demás, inclusive los de la filosofía,
deben ser tomados de antemano como absurdos. Sólo los enunciados de la
ciencia resultan verificables empíricamente. mientras que, respecto de
la filosofía, no hay modo de comprobar o contrastar con la experiencia
su contenido concreto. La capacidad de verificación empírica como
único criterio de todo sentido es, pues, el supuesto básico de la
labor analítica.\\

Wittgenstein no niega la existencia de todo lo que no puede ser
expresado con sentido matemática o científicamente. Cuando se han
formulado y contestado todas las preguntas científicas posibles, los
problemas de nuestra vida no han sido ni siquiera tocados.\\

Para W. lo que está fuera de los límites del lenguaje, lo que es
\textbf{impensable e indecible}, puede existir, y es lo místico. No se
rechaza lo metafísico, lo que se niega es la posibilidad de
contrastarlo, relegándolo a la mística.\\

Las conclusiones para la filosofía son:
\begin{itemize}
\item Debe renunciar a constituirse como una teoría o compendio de
verdades sobre la realidad, sobre el mundo y el hombre.
\item Cualquier establecimiento de contenidos teóricos sólo corresponde
a la ciencia.
\item Solo debe quedar como una actividad de clarificación mediante una
labor de análisis de la estructura lógica del lenguaje.
\item Le corresponde, cómo análisis formal, la tarea de autodisolver lo
que tradicionalmente ha venido siendo, demostrando el sinsentido
de los enunciados en que se ha estado expresando.
\end{itemize}

\subsubsection{Significado y reglas de uso del lenguaje}
\label{sec:org2dd2738}
Este planteamiento no es fijo dentro del pensamiento de W. ya que en
\emph{Investigaciones filosóficas} se retracta de esta reducción de la
diversidad del discurso a proposiciones categóricas, ligadas entre sí
por funciones de verdad, así como la correlación entre estas frases y
los datos de la observación.\\

Wittgenstein cree que es necesario tener en cuenta la complejidad
lingüística y desconfiar de los procedimientos que supongan una
conjunción no problemática entre los elementos del lenguaje y los
elementos de la realidad. El lenguaje no refleja el mundo ni tiene
como único objetivo describirlo. Es una forma de conducta entre otras,
con la pluralidad de funciones: Ordenar, describir, informar, hacer
conjeturas, etc. cada una de las cuales puede describirse como un
juego de lenguaje. Las proposiciones son significativas porque son
expresiones de estos juegos de lenguaje. Los diversos usos del
lenguaje manifiestan como característica común un cierto aire familiar
que los asemeja, se someten a reglas, pero cada cual a la suyas
propias.\\

El significado pues, no en la verificabilidad de lo que se dice, sino
que hay que buscarlo en el uso que se hace de las palabras. Es decir,
es el contexto lo que da sentido a las palabras. La mayoría de errores
filosóficos vienen de confundir los contextos o de juzgar un contexto
por las reglas de otro. \textbf{La tesis principal de las investigaciones
filosóficas} es que todo lenguaje consiste en multitud de juegos de
lenguaje. El lenguaje correcto es aquel que observa el recto uso de
las reglas. Toda palabra tiene sentido si es empleada en su
contexto. El sentido lo dan las reglas de uso, como las piezas en el
ajedrez y las reglas de movimiento.

\section{El proyecto epistemológico del \emph{tractatus}}
\label{sec:org98b5db7}
\subsection{El sentido del lenguaje no puede ser expresado por el lenguaje}
\label{sec:org8e7900f}
Russell (R.) sostenía una doctrina ontológica que decía que, al final
del análisis, existe en el universo, están hechos atómicos. Esta
doctrina es deductiva, no empírica, desde un análisis no-empírico del
lenguaje, hasta la naturaleza de la realidad que el lenguaje
describe.\\
Wittgenstein llevará hasta el límite los principios del atomismo
lógico de Russell y mostrará sus inconsecuencias e intentará acabar
con este resto de metafísica subyacente a la obra de Russell.\\

Según el atomismo lógico, una proposición puede ser significativa si o
bien hay o puede haber un hecho atómico que al que corresponde, o bien
si es una función de verdad —en caso de ser compleja— de las
proposiciones de este tipo. Pero la mayoría de proposiciones que el
atomismo lógico, incluído el del propio Wittgenstein, intentaron
establecer no son de ninguna de estas dos clases. La mayoría de estas
proposiciones no afirmaban hechos, sino que intentaban más bien hablar
sobre hechos y, concretamente, sobre las relaciones entre
proposiciones y hechos.\\

Así, tales proposiciones no podían ser significativas ya que
intentaban decir lo que no puede ser dicho. Wittgenstein concluye que
afirmar que lo que él mismo había dicho era u n sin-sentido. Entender
su propia obra era caer en la cuenta de que no había dicho nada en
absoluto.\\
Wittgenstein termina el tractatus con la frase \emph{de lo que no se puede
hablar, mejor es guardar silencio}. La propia doctrina atomista lógica
muestra su carácter autonegador y la falta de sentido de sus propias
afirmaciones.

\subsection{La proposición como unidad de significado}
\label{sec:orgcb363a7}
\subsubsection{La teoría \emph{pictórica} del conocimiento}
\label{sec:org4d57ed1}
Para Wittgenstein, la \textbf{unidad de significado} es la proposición misma,
no el término singular. La palabra tiene significado como parte de una
proposición, no por sí misma. Una proposición es una representación
(picture, foto) de la realidad. Representa un estado de hechos
(affairs) o situaciones. Una proposición no puede pintar una imagen de
la realidad, solo mostrarla porque hay cosas que se escapan de su
forma. Las proposiciones son de dos tipos, atómicas o moleculares. No
son posibles las proposiciones reflexivas de la filosofía.\\

En la teoría del significado de W. las proposiciones tienen más peso
que los términos por sí solos. Pero, al igual que Russell, los signos
simples (términos?) usados en las proposiciones se llaman nombres. Un
nombre significa un objeto. El objeto es su significado.\\

Para W., igual que R., la conexión entre el lenguaje y la realidad se
da en la relación que hay entre las proposiciones atómicas con los
hechos atómicos. En la representación de un objeto existe una
correspondencia entre las partes o elementos de la representación y
las del objeto. Estos elementos no sólo deben estar presentes, sino
que también su estructura, forma y disposición deben ser los mismos:
\emph{La configuración de los objetos forma el hecho atómico}.\\

La proposición es capaz de representar (picturing) a esos hechos
no-verbales (?). Es por esto por lo que el lenguaje puede referirse al
mundo, puede significar algo distinto de sí mismo. Los elementos de
estos hechos son, por parte del lenguaje, nombres y signos denotativos
simples (signos simples de significado), mientras que por parte de la
realidad, objetos particulares.\\

Un objeto particular no puede ser una composición de hechos, como por
ejemplo \emph{punto blanco o una pieza roja}. El objeto, aunque pueda tener
atributos que caractericen una configuración, el objeto en sí no puede
ser una configuración, debe ser simple, ya que sino podríamos caer en
contradicción: El mismo punto puede ser blanco o rojo y concluir
erróneamente que es incoloro.\\

A partir de las proposiciones atómicas se pueden componer
proposiciones más complejas, las \textbf{proposiciones
veritativo-funcionales}.\\

Entre los posibles grupos de condiciones de verdad hay dos casos
extremos. En uno cuando las condiciones de verdad son \textbf{tautológicas}. Es
decir, la proposición es verdadera para todas las posibilidades de
verdad de las proposiciones elementales.\\
En el otro caso, la proposición es falsa para todas las posibilidades
de verdad, las condiciones de verdad son \textbf{contradictorias}.
\begin{quote}
En general, la proposición muestra aquello que dice pero la tautología
y la contradicción muestran que no dicen nada. Tautología y
contradicción no son figuras de la realidad. No representan ningún
posible estado de cosas. En efecto, una permite todos los posibles
estados de cosas, la otra ninguno.
\end{quote}

\subsubsection{La reafirmación del empirismo}
\label{sec:org1969ed6}
Para decir si una proposición es verdadera o falsa, se debe siempre
compararla con la realidad. Es imposible decir sólo si desde la
representación si ella misma es verdadera o falsa. No hay
representaciones que sean verdaderas a priori.\\

Wittgenstein sostiene los \textbf{principios empíricos}: no hay modo de pasar
de un conocimiento puramente empírico a uno supraempírico. No hay
relaciones causales que nos permitan pasar de una realidad conocida,
de la experiencia, a una realidad superior que fuera su causa.\\

Hume, empírico radical, había llegado a la conclusión, mediante el
análisis psicológico del entendimiento humano, de negar la metafísica
y a considerar la lógica, la matemática y las ciencias experimentales
como las únicas ciencias respetables. Russel y Bradley se oponen, en
nombre de la lógica, al psicologismo de Hume y propugnan un cierto
tipo de metafísica, sin éxito como hemos visto. Wittgenstein concluye
que sólo cabe hablar de hechos. Sólo tienen sentido las frases
atómicas, cuya verdad consiste en la constatación de los hechos, y las
frases moleculares, cuya verdad depende de la de las
atómicas. Cualquier reflexión que sobre los hechos quiera hacerse
resulta una imposibilidad lógica.

\subsection{La distinción entre lo dicho y lo mostrado}
\label{sec:orgfe0c6f5}
Wittgenstein distingue entre lo dicho y lo mostrado, la relación entre
el lenguaje y los hechos puede demostrarse pero no decirse. El
positivismo lógico no concibe esta distinción, no hay otra realidad
que la de los hechos verificables. Wittgenstein y Russell, con
influencias de la tradición racionalista germánica, difieren del
empirismo radical de Hume. Pero la metafísica de W. y R. venía
establecida como la exigencia de una lógica. No se pueden establecer
proposiciones verdaderas a priori. Los únicos juicios verdaderos de
modo inmediato son los juicios tautológicos, pero que no dicen nada
nuevo. En este sentido, la filosofía no tiene objeto propio, sus
proposiciones no son falsas sino sin-sentido. Los problemas
tradicionales de la filosofía no son sino pseudo-problemas que surgen
de no entender la lógica de nuestro lenguaje. La filosofía no es un
cuerpo doctrinal, sino una actividad. Su tarea consiste en
aclaraciones, mostrar pero no decir. No da como resultado
proposiciones verdaderas. La totalidad de las proposiciones verdaderas
es el cuerpo de la ciencia natural.\\

Aún esta tendencia empírica, W. no es consecuente del todo con lo que
dice. A pesar de afirmar que \emph{los límites de mi lenguaje son los
límite de mi mundo}, W. deja un hueco ambiguo para una metafísica y
una ética mediante las nociones de lo inexpresable o indecibilidad,
situándolas más allá del mundo de lo expresable. W. lo llama lo
místico, lo inexpresable es lo místico, de lo que no se puede ni
afirmar ni negar nada con fundamento.\\

El neopositivismo lógico, se propondrá como objetivo demostrar que
toda metafísica carece de sentido.

\section{Conocimiento y juegos de lenguaje}
\label{sec:org8a61c20}
\subsection{Wittgenstein se retracta}
\label{sec:orgd101e36}
Entre la publicación del tractatus en 1921 y la publicación de su gran
obra \emph{investigaciones filosóficas} en 1953, W. no publica nada excepto
un pequeño artículo. Desde entocnes, W. quiso que el \emph{tractatus} y las
\emph{investigaciones} se publicaran juntas a modo de contraste. Esta obra
tendrá una enorme influencia en la filosofía analítica. W. abandona la
noción de significado y se propone como objetivo de la filosofía
analítica describir los diversos usos o juegos del lenguaje, las
maneras cómo utilizamos en la práctica el lenguaje, unidas a
actividades que realizan en un contexto, un medio natural, técnico y
cultural.\\

Es interesante comparar la multiplicidad de herramientas del lenguaje
y de sus modos de empleo, la multiplicidad de géneros de palabras y
oraciones, con lo que los lógicos han dicho sobre la estructura del
lenguaje.\\
Hablar el lenguaje forma parte de una actividad o forma de
vida. Ejemplos: Relatar un suceso, inventar una historia, resolver un
problema, actuar en teatro, etc.\\

No se trata de analizar las relaciones entre palabras y
objetos. Wittgenstein se retracta de su perspectiva anterior del
lenguaje como imagen de la realidad y critica los mitos que esa
concepción conlleva, en especial el del pensamiento como una especie
de lenguaje interior, inmaterial y racional que realizaría el ideal
lingüístico que las lenguas naturales y concretas no consiguen llevar
a cabo como espíritu o alma. El pensamiento no es más que un uso
monológico y silencioso del lenguaje, que es fundamental y
originariamente público, dialógico (relativo al diálogo) y
social. Así, el pensamiento no es anterior ni esencialmente diferente
del lenguaje, sino que de él deriva y lo presupone.

\subsection{La irreductible diversidad de los usos del lenguaje}
\label{sec:orgb20e8ad}
\subsubsection{El fin del privilegio de la teoría}
\label{sec:org97bf78f}
El principio básico en las \emph{investigaciones} es que no existe un
lenguaje ideal que refleje los únicos hechos existentes, aquellos que
puedan ser verificados, sino que hay multitud de lenguajes que no
tienen entre sí nada en común. Ni se unen en un lenguaje superior, ni
apuntan a una realidad que tras él se oculta. Hay una pluralidad de
realidades y son estas las que hay que mirar en sí mismas, tratando de
captar la función que desempeñan en los distintos contextos. Pensar
que existe un lenguaje ideal común a todos los lenguajes es una
confusión que el propio lenguaje ha creado. Los mismo análisis
realizados hasta ahora, no sólo son infructuosos, sino que además son
los causantes de esta confusión, por haber mantenido la tesis de que
hay un lenguaje ideal que refleja una realidad subyacente. De esta
manera W., además de negar la teoría anterior, niega los hechos que
esta teoría intentó explicar. A partir de aquí W. sostiene que no se
trata de explicar nada, sino tan sólo de describir.\\

El lenguaje no es algo único e ideal, no es algo divino o
transcendental que hace participar al hombre en un modelo espiritual e
inmutable. Es empírico. Es complejo y cambiante. Forma parte de la
historia natural y cultural de los seres humanos. No tiene sentido
privilegiar el juego de lenguaje de la explicación de los hechos. Los
usos descriptivos son también múltiples. Pretender reducir la
complejidad polimórfica de los lenguajes al lenguaje de la descripción
teórica es una ilusión y un abuso. La descripción teórica unifica y
homogeneiza a costa de negar la diversidad y el cambio.\\

\subsubsection{No hay ningún universo de sentido inmutable}
\label{sec:orgfbad37c}
Lo que caracteriza a los juegos de lenguaje es su carácter social,
público, el hecho de ser compartidos por un determinado número de
hablantes que juegan el mismo juego y observan las mismas reglas de
uso. Su estabilidad depende de esta práctica común, unida a la
educación y a la costumbre, a la forma de vida, compartidos. lo que
determina la gramática y la semántica es el uso intersubjetivo y no
una relación especial el lenguaje con un mundo de referencias
trascendentes, ni conceptos universales que se captan por la intuición
o se deducen racionalmente, ni el reflejo de formas esenciales de las
cosas.\\

Los juegos del lenguaje cambian e incluso desaparecen. No hay un
universo subyacente a ellos de sentido inmutable. Sólo las reglas de
uso dan al lenguaje su relativa estabilidad e identidad como
institución, reglas que gobiernan una actividad común pero que solo
existen mientras la acción común las respete y las confirme en su
vigencia. Por seguir una regla, no es más que una práctica habitual
más allá de la cual no tiene sentido buscar un fundamento único. Esos
juegos de lenguaje se practican, cambian y, hasta, desaparecen.\\

\subsection{El significado es el uso}
\label{sec:org127a730}
\subsubsection{El abandono del modelo referencial}
\label{sec:orgc2282ca}
Así pues, W. rompe con el núcleo filosófico de su tractatus y el eje
de la tradición filosófica desde Platón a Husserl. En general, la
filosofía ha basado el significado en una relación que refiere a las
proposiciones lingüísticas a realidades no verbales y que el sujeto es
capaz de captar. Las \emph{investigaciones} se oponen a esta concepción y
defienden que el significado no depende de la referencia ni es la
referencia. EL significado de toda proposición depende de su uso, el
cual puede ser también un uso referencial, que pretenda designar algo
extralongüístico.\\

El uso nunca es único. Cualquier palabra remite a una familia de usos
cuya coherencia es análoga. Es un autoengaño querer reducir el
significado de una palabra a un concepto unívoco que quedara
comprendido en su definición. No se puede sustituir la diversidad
experimentada y practicada de los usos por la unidad pensada del
significado ideal. Para W. es la fuente del dualismo que opone el
mundo material, aparente y cambiante del lenguaje, al mundo
espiritual, racional e inmutable de la realidad.

\subsubsection{Describir en lugar de explicar}
\label{sec:orge40cf84}
No se trata de estudiar el lenguaje para hacerlo científico, sino de
verlo tal cual es y descubrir el uso y función de los lenguajes que
empleamos en cada situación, si realmente queremos comprender el
lenguaje. Hay que olvidar todo intento de justificar esencias y
realidades últimas mediante el establecimiento de un lenguaje
científico, como pretendía el atomismo lógico. La inicial tarea de la
filosofía es proporcionar una terapia a esa enfermedad, deshacer los
embrollos descubriendo sus causas y conseguir una claridad completa.\\

Para esta tarea es preciso advertir que el pensamiento está embrujado
por el lenguaje. Es preciso aprender a ver los lenguajes en su
dimensión plural, contextual, vital, para olvidar la necesidad del
lenguaje ideal. Cada lenguaje se justifica por sí mismo como una forma
de vida. La vida cambia y con ella los hechos físicos. De igual manera
los usos y funciones del lenguaje ordinario. Así pues, su nuevo
principio es \emph{no preguntes por el significado, pregunto por el
uso}. El lenguaje es una actividad que tiene muchos usos y funciones,
hay que advertir su complejidad. Ninguna de las palabras que usamos
tiene un significado fijo, cambia según las situaciones en que se
usa. Ni tiene una vigencia permanente, desaparece en un momento
determinado y da lugar al nacimiento de otro significado.\\

Propone un análisis liberado de todo prejuicio teórico, perspectiva
fundante y unitaria, que hará desaparecer esta noción misteriosa de
significado y permitirá ver la lógica propia irreductible de cada
enunciado. Esta lógica vendrá determinada por el contexto del juego de
lenguaje en uso.\\
Hasta ahora, los filósofos, especialmente los atomistas lógicos,
incluido el autor del \emph{tractatus} han intentado aplicar un conjunto
único de reglas en orden a construir un lenguaje ideal oculto tras las
imperfecciones del lenguaje común, a ejemplo del matemático. Y en
olvidar la diversidad de funciones del lenguaje, han aparecido una
multitud de perplejidades filosóficas.\\

Hay un paralelismo entre juego de lenguajes, formas de vida y
aprendizaje de los diversos términos de su uso. El nominalismo
sostiene la relación nombre-cosa como algo fijo y permanente que se
opone a la concepción del lenguaje como pura actividad inmanente a la
propia forma de vida, cambiante en la medida en que ella
cambia. Aprender algo es ser capaz de hacerlo. Lo que importa es
descubrir la función desempeñada por cada palabra en el juego del
lenguaje correspondiente. Solamente en su uso podremos aprender el
significado de las mismas.\\

Con esta autocrítica expresa que las cosas están bien como están, tal
como estaban antes de haber sido introducidas por la filosofía las
conclusiones que engendraron el afán de convertirla en un lenguaje
ideal. Dejar las cosas como están y tratar de ver como son.

\subsection{Lenguaje, conocimiento y realidad}
\label{sec:orga46ae32}
Algunas dudas emergen en relación a la relación entre lenguaje y
realidad respecto de la última posición de Wittgenstein. Por un lado
busca una fundación de los juegos llamados secundarios del lenguaje
sobre un juego primario formado por expresiones lingüísticas de
sensaciones en las que manifestaría nuestro contacto con la
realidad. Por otro lado, defiende la comprensión del lenguaje como
pura convencionalidad inexorable.\\

No existe una esencia de la palabra, sino un uso y unas reglas que
determinan las conexiones correctas en el uso de esas palabras.\\

Hay quien sostiene que en las \emph{investigaciones} es notoria la relación
de verticalidad entre los juegos de lenguaje y la realidad, no sólo
entre los juegos de lenguaje, pero también entre estos y la realidad.

\subsubsection{La verdad como coherencia interna}
\label{sec:org8619b61}
Superada la teoría pictórica (\emph{tractatus}) y que se basaba en el
supuesto de una homología entre proposiciones y realidad, la
introducción del concepto de juego y el reconocimiento de una
pluralidad de lenguajes implica que los lenguajes ya no son
reductibles a ninguna clase de unidad ni por la vía lógica (lenguaje
como expresión trascendental de la estructur objetiva del
pensamiento), ni tampoco por la vía ontológica (el lenguaje como
imagen o expresión de la realidad). El sentido o verdad de un lenguaje
la determina sólo la conexión sistemática de sus elementos sobre la
base de uso de reglas que resulta eficaz en la práctica. El sentido de
un término ni le viene del hecho de ser expresión primaria o
secundaria de una expresión. sino de una posición funcional en un
juego de lenguaje. Está en función de un orden introducido por unas
reglas que son convencionales. Así, el significado no se basa ni en
los hechos empíricos que representa, ni en las formas a priori de su
estructura lógica.\\

Para W.  el juego de lenguaje es convencionalidad
inexorable. Convencionalidad quiere decir que el lenguaje no debe su
verdad o su significado nada más que al hecho de ser un sistema de
reglas que funciona objetiva y coherentemente porque es aplicado por
todos los hablantes al resolver con éxito sus problemas y necesidades
de comunicación. Así, la verdad de un lenguaje no viene de una
justificación externa, sino que es su propia coherencia interna que la
funda. Preguntarse sobre la verdad de un lenguaje es preguntarse por
las condiciones de funcionamiento de las reglas.\\

El cambio de W. entre el \emph{tractatus} y las \emph{investigaciones} es que el
lenguaje no es una pura reducción a un sistema formal lógico, sino a
esta convención. Y esto no es posible hasta que no se eliminan todos
los signos lingüísticos, todo significado intuitivo.\\
Lo que hay como sustrato del lenguaje no son ciertas esencias. Pero
tampoco son impresiones ni sensaciones procedentes de una naturaleza
humana común como realidad última. Lo que hay es vacío, la simple
coherencia de unas conexiones y de unas relaciones en un sistema que
nada tienen que ver con la descripción de unos contenidos.

\subsubsection{La objetividad de un operar común}
\label{sec:org0a5aa48}
Este proceso de reducción del lenguaje a su completa convencionalidad
y formalización es un proceso inexorable. Porque las reglas del
lenguaje y su uso son convencionales, su operar resulta inexorable y
objetivo. La objetividad y eficacia del lenguaje se deben a que no
podemos no usar sus reglas y que con estas se opera con una exactitud
inexorable que no hay lugar para reducciones psicologistas de su
validez, ni para interpretaciones pragmáticas de este funcionamiento
exitoso. Pues el lenguaje no se realiza privadamente y según a uno le
convenga, sino que es un operar común. Lo característico de la
convención es que todos tenemos que jugar el mismo juego con las
mismas reglas.\\

Es un sinsentido preguntarse por un fundamento de la verdad del
lenguaje como origen de su validez. Ni el significado ni la validez de
un lenguaje se demuestra refiriéndose al contenido de sus signos, ni
al hecho de derivar de un lenguaje ideal, sino que se debe a la simple
conexión que se da entre esos signos y a las reglas convencionales que
regulan esa conexión. Son sólo esas reglas las que hacen comprensible
y comunicable una expresión por el hecho de someterla a un orden e
integrarla a un juego común.

\subsubsection{La filosofía como terapia del lenguaje}
\label{sec:orgb3b5581}
Wittgenstein resalta con insistencia el carácter de pura
convencionalidad que tienen los lenguajes, librándonos de la aureola
de lo ideal como trasmundo de lo real. Esto significó un giro
importante del concepto mismo de filosofía. La filosofía no puede ya
seguir estando animada por la ilusión de encontrar lo ideal más allá
de lo real ni dentro de lo real. Tampoco puede consistir en la
búsqueda de una unidad formal deducible como sintaxis universal o como
expresión de la estructura lógica y trascendental del mundo. La
pluralidad de los juegos de lenguaje reduce la coherencia y la
objetividad de cada uno de ellos al funcionamiento de sus reglas, a
sus operaciones y a sus usos comunes. La filosofía es pues, análisis
de nuestras múltiples formas de expresión. Sólo puede ser ilustración
progresiva de las formas del lenguaje, interesada en ampliar
continuamente la demostración de que el lenguaje no es más que una
familia de construcciones gramaticales más o menos emparentadas entre
sí.

La filosofía analítica considera aquellos problemas que no están
suficientemente aclarados y, por tanto, no resueltos y los elimina
como problema. Es una manera de curar el lenguaje, da un sentido a la
filosofía como terapia del lenguaje.\\
Pero a la filosofía no le corresponde reformar el lenguaje (curarlo)
sino mostrar simplemente cual es el modo correcto de usarlo. Al hacer
el análisis, le corresponde delimitar el espacio operativo de cada
juego lingüístico en su inexorable efectualidad (un llevarse a cabo),
mostrar como tal juego de lenguaje en concreto tiene una eficacia
racionalizadora en virtud de la introducción de un orden y de una
regulación propios.\\
En vez de curar el empleo ordinario de un lenguaje, lo que debe
tratarse más bien es de que ese empleo nos cure a nosotros de los
problemas indecibles que nos ha creado la filosofía.\\
Ahora ya no se rechaza la apariencia sino que lo que se intenta es
volver a llevar las palabras de su empleo metafísico a su empleo
cotidiano, abandonando en esa desacralización del lenguaje que el
propio concepto de juego como convencionalidad inexorable ya
representa.

\subsubsection{El problemático lugar de la crítica}
\label{sec:org5325641}
Como análisis filosófico de un juego de lenguaje, sólo le corresponde
delimitar su espacio operativo en su inexorable efectividad,
describirlo en su funcionalidad práctica y señalar el tipo de terapia
que puede resultar útil en ese juego. Pero esto plantea dos cuestiones.
\begin{itemize}
\item No es evidente si el análisis se limita a aclarar el
funcionamiento del juego lingüístico dado, o si su propósito mismo
de acción clarificadora implica cierta dinámica de
transformación. Es decir, delimitar este espacio operativo, poner
un orden, retrotraer el lenguaje a un estado de hecho, puede ser
una manera de transformar y no solo retornar a un estado de
hecho. La tarea de clarificar y de disolver problemas puede
interferir y transformar el propio lenguaje.\\

En la estructura del uso de un juego de lenguaje deben estar
incluídos también los usos erróneos y equívocos de ese juego. Hay
que ser consciente del uso de estos usos de tal modo que
clarificar y enseñar a jugar bien en este caso se entienda también
como transformar el juego en su efectividad, como parte del mismo
juego. Si poner orden es ya un modo de transformar, y se escoge un
determinado orden, entre otros muchos posibles, entonces algunos
elementos del planteamiento deberían cambiar al tener en cuenta
que el juego, después de ordenarlo, ya no será idéntico al que era
antes.\\

Wittgenstein afirma claramente que el orden que puede poner el
análisis filosófico en un juego de lenguaje es un simple
restablecer. No es la apertura de un nuevo paradigma ni puede
representar un cambio del juego normal jugado hasta ahora.

\item Pero esta imagen de estática de los juegos lingüísticos como
situación normal puede parecer contradictoria en la línea de la
primera cuestión. Permanece como criterio de verdad la adecuación
del juego lingüístico a un determinado estado de hecho que ahora
se designa como situación normal.
\end{itemize}

Con la comprensión del lenguaje como pluralidad de juegos
lingüísticos, hay un desencantamiento de la lógica y de la filosofía
metafísica como búsqueda de lo esencial y la unidad. Pero el concepto
mismo de juego se cristaliza, se convierte en una estructura
estable. \\
Esto hace difícil concebir usos diversos del juego lingüístico llamado
normal.

\subsection{La experiencia del mundo como un todo limitado}
\label{sec:orgdf86a2f}
En la obra de W. se insiste de la impotencia de la filosofía para
transformar nada. La filosofía no puede afectar al lenguaje que
analiza, ni puede producir nuevas experiencias. La filosofía sólo
sirve para poner un orden que no puede ser más que el orden del juego
normal y según las reglas con las que es jugado hasta ahora, según su
empleo cotidiano. Se limita a mostrar. El problema de los cambios de
las reglas en un juego no es formalizable ni practicable en la
filosofía. Cuando pretenda sobrepasar esos límites se traiciona,
porque se convierte enseguida en metafísica, en búsqueda de la esencia
ideal y la unidad. Dentro de sus límites, la filosofía sirve sólo para
aclarar ese orden. Al desencanto respecto a la lógica corresponde este
desencanto respecto a la filosofía.\\

Cuando W. se refiere a la mística, se refiere a esa experiencia del
mundo como todo limitado. En el \emph{tractatus}, expone que los límites
del lenguaje son los límites del mundo, esa experiencia del límite es
la que da origen a lo místico. Así, lo místico no es realidad
trascendente, sino la experiencia radical del mundo dentro de sus
límites. Esto significa que en el mundo no hay valor (Nietzsche) y que
el mundo es como es y sucede como sucede. La representación
mundo-representación ya no puede ser trascendente. Sentir sus límites
es precisamente lo místico. Pero lo místico no es sólo el conocimiento
de los límites de la formalización del mundo, sino que es también el
conocimiento del formalismo mismo como límite.\\
Excluye de la expresión lingüística toda indicación a un inefable, con
esto funda la posibilidad de proposiciones dotadas de sentido.\\
Con esto muestra también negativamente lo inefable como límite. Es
decir, toma conciencia de que no puede decirse.

Sin lo místico el formalismo tendería a abarcarlo todo. Expulsar lo
místico significaría creer que no existe nada que callar, con una
percepción errónea de la esfera del formalismo. Si lo místico no
obligara a conocer ningún límite, entonces el formalismo se
presentaría como verdad y eliminaría todo límite. Lo místico es el
primer paso entre el \emph{tractatus} hacia el punto de vista del juego de
lenguaje que desarrolla en las \emph{investigaciones}. Reconocer todo lo
que es necesario callar es esencial para definir los límites entre los
que es posible describir algo.
\end{document}