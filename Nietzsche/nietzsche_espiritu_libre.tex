% Created 2023-08-24 dj. 11:41
% Intended LaTeX compiler: pdflatex
\documentclass[a4paper, 10pt, twocolumn, spanish]{article}
\usepackage[utf8]{inputenc}
\usepackage[T1]{fontenc}
\usepackage{graphicx}
\usepackage{longtable}
\usepackage{wrapfig}
\usepackage{rotating}
\usepackage[normalem]{ulem}
\usepackage{amsmath}
\usepackage{amssymb}
\usepackage{capt-of}
\usepackage{hyperref}
\usepackage[T1]{fontenc}
\usepackage[margin=.75in]{geometry}
\setlength\parindent{0pt}
\author{Jordi Serra}
\date{\today}
\title{Apuntes de Teoría del Conocimiento ii\\\medskip
\large Sobre Nietzsche y la filosofía del espíritu libre}
\hypersetup{
 pdfauthor={Jordi Serra},
 pdftitle={Apuntes de Teoría del Conocimiento ii},
 pdfkeywords={},
 pdfsubject={},
 pdfcreator={Emacs 27.1 (Org mode 9.6.2)}, 
 pdflang={English}}
\begin{document}

\maketitle
\tableofcontents


\section{Etapa intermedia de Nietzsche}
\label{sec:org03df3ca}
El año 1876 es un año importante en la vida de N. y de su obra. Es un
año de crisis que se distancia de su vida de juventud, resaltando
diferencias ideológicas con su pensamiento previo debido a la
maduración propia de su pensamiento. Esto se manifiesta con un
distanciamiento de la senda en la que el wagneriarismo habí
evolucionado y empezó el distanciamiento con el propio Wagner, con
quién rompió definitivamente dos años después.

El desarrollo de esta crisis se influido por la grave enfermedad que
contrae este mismo año. Hay quien sostiene que la contracción de la
enfermedad supuso un aceleración del cambio de su pensamiento. En 1879
abandonó el mundo de la filología clásica y el mundo académico y
empezó una nueva vida como filósofo errante.\\[0pt]

Empezó la redacción de \emph{Humano, demasiado humano} en la que ya se
expresa el espíritu nuevo de N. y el distanciamiento con Richard
Wagner.

La crisis tenía varios frentes. El más importante era el del
replanteamiento de su condición académico-social. Nietzsche se alejaba
críticamente cada vez más de su profesión académica y pedagógica de
filología clásica. De hecho N. ya buscaba un método con el que pudiera
formular con más eficacia el tipo de crítica que ya estaba ejerciendo
un tanto difusamente y que se puede detectar en sus escritos de
juventud.\\[0pt]

El núcleo de la preocupación espiritual de N., en el que se
entremezclan su disgusto con la profesión de filólogo académico, el
malestar de sus malas relaciones con W. y el sufrimiento de la
enfermedad, con constituye su lucha por dominar el método genealógico
del que una primera y madura formulación se expresa en la publicación
de \emph{Humano, demasiado humano}. Durante los años previos a la
publicación de ésta obra, N. se obsesiona, debate e intenta analizar
genealógicamente los prejuicios en los que la filología clásica
académica se asienta y la imagen de Grecia que de ella emana.\\[0pt]

Con ello revela de qué modo madura la orientación de su reflexión
filosófica. Con el desenmascaramiento de la falsedad de los prejuicios
de la filología y de las razones de esa falsedad, ahonda en la que
será la misión más importante e insistente de toda su obra: La crítica
a la modernidad como época de la decadencia y del nihilismo.\\[0pt]
Si bien, la noción de nihilismo no la expresa de manera explícita aún,
es con esta con la que revela el sentido de nuestra diferencia con los
griegos, y de ahondar en la reivindicación del clasicismo de la
Antigüedad frente al mundo moderno.\\[0pt]


En \emph{Humano, demasiado humano}, \emph{Aurora} y \emph{la gaya ciencia} desarrolla
y madura su pensamiento ensayado en las obras de su juventud. El uso
crítico de un método de análisis y explicación conduce a una filosofía
que se comprende como crítica de la cultura. Nietzsche expone un
concepto de filosofía como hermenéutica de la interpretación en la que
las ideas no son verdaderas o falsas, sino signos que delatan la
naturaleza.\\[0pt]
Se deduce pues, que allí donde se coloque a la naturaleza en lugar de
la cultura, se tratará siempre de una construcción de la cultura
misma. Esto equivale a afirmar que no existe ningún origen
esencial. Se busca la procedencia y no el origen. No existe ninguna
esencia en el punto de partida, sino el proceso ininterrumpidamente
recurrente de una sucesión de apariencias o máscaras.\\[0pt]

En el pensamiento del N. joven expresa el concepto griego clásico de
cultura es modélico por ser de una naturaleza, \emph{physis}, artística. En
\emph{el nacimiento de la tragedia} nos habla del sueño y la embriaguez
como impulsos artísticos de la \emph{physis}. Hay en esta manera de
entender la naturaleza, los principios de su genealogía, pues esta la
entiende, expresa, como una fisiología de la cultura que remita a una
articulación interna de la \emph{physis} entendida como vida que traduce en
los estados corporales creativos: el sueño y la embriaguez.\\[0pt]

Las fuentes determinantes de todo pensamiento y de toda creación
cultural son de naturaleza no racional y no consciente. Esta tesis se
formula de tres maneras. La primera en su etapa juvenil, romántica con
la que expresa la \emph{metafísica de artista} y las nociones de lo
apolíneo y dionisíaco. Una segunda que se inicia con \emph{Humano,}
\emph{demasiado humano}, que gira en torno al esfuerzo por comprender lo
pulsional fuera de la metafísica. En esta fase madura la idea de que
no hay instintos o impulsos como fuerzas en sí. Y una tercera de las
obras escritas a partir de \emph{Así habló Zaratustra}, cuando se formula
ya la tesis en el marco de la hipótesis de la voluntad de poder.\\[0pt]

En los escritos de juventud, Nietzsche ensaya una ordenación de las
artes en la tragedia antigua no explicable en virtud de una lógica
racional, sino que remite a otra lógica más profunda que tiene que ver
con una fisiología de los impulsos artísticos dentro de una
determinada concepción del ser humano y de su dinámica
cuerpo-espíritu.\\[0pt]
La explicación se refiere a cómo surge la creación artística a partir
de la oposición dual de lo apolíneo y dionisíaco —impulsos de la
naturaleza— y su tensión polarizada.\\[0pt]
Nietzsche aplica estas categorías de lo apolíneo y dionisíaco a la
interpretación del conocimiento y de las demás esferas o formas de
cultura —i.e., religión, política etc.\\[0pt]
El contraste con este modo de explicación lo ofrece el análisis de
Sócrates y Platón, con la ilusión de una autonomía de lo racional como
realidad en sí. Si Sócrates, con su ironía, dejaba perplejos a sus
interlocutores, era porque les pedía que dieran razón, que explicasen
racionalmente un comportamiento o una situación que revelaba, en
realidad, un instinto.\\[0pt]

Nietzsche piensa ya en su juventud que las explicaciones racionales
son la parte emergida de una pulsión, de un proceso pulsional que está
actuando debajo y que es donde radica el verdadero sentido de lo que
se trata. Esta relación entre la pulsión y las formas de la cultura es
pensada bajo la influencia de Shopenhauer, en el sentido de que la
forma, el fenómeno, la representación, lo apolíneo es la apariencia o
la aparición de lo dionisíaco, el fondo pulsional o energía básica del
mundo y del cuerpo.\\[0pt]
Con la evolución de su pensamiento, Nietzsche empieza a pensar que esa
forma apolínea, esa representación en la que consisten las formas de
la cultura no es el reflejo, la imagen de la fuerza en sí de la
naturaleza presente en cada individuo, sino traducciones, como en el
sueño, producidas por esa fuerza incognoscible e inalcanzable en
lenguajes ontológicamente diferentes. Dice que con las palabras no se
llega jamás a la verdad ni a una expresión de adecuada; prueba de ello
es la multitud de lenguajes existentes, ya que si no fuese así, no
habría tantos. Sostiene que la cosa-en-sí es totalmente
inaprehensible. El lenguaje designa solo las relaciones de las cosas
con los seres humanos y para su expresión recurre a la metáforas más
atrevidas.\\[0pt]

En consecuencia, no se puede hacer del instinto un principio en
sí. Las pulsiones o instintos son dispositivos variables que se forman
a partir de determinadas condiciones históricas y de su evolución. Son
el resultado de un aprendizaje y de un conocimiento llevado a cabo por
la cultura y el proceso de socialización.\\[0pt]
En suma, los instintos so algo que se ha construido, y por tanto son
variables no son principios ontológicos innatos, los mismos en todos
los seres humanos.\\[0pt]
Son el resultado de incorporar, de introducir en el propio cuerpo
mediante un determinado tipo de educación y entrenamiento, formas muy
básicas de valorar o estimar, de sentir atracción o repugnancia ante
determinadas acciones o ideas.\\[0pt]
Los valores de nuestra moral son el resultado de esa educación
impuesta al cuerpo, durante un tiempo muy largo, de estimaciones que
desde muy antiguo se han sentido como buenas o malas, u que se han
venido incorporando y transmitiendo luego de generación en generación
a lo largo de los siglos. La fuerza de los conocimientos no reside en
su grado de verdad, sino en su antigüedad, en el hecho de estar
incorporados, en su carácter de condiciones de la vida.\\[0pt]

La pulsión, el instinto son productos de esas primitivas valoraciones
o interpretaciones y, una vez incorporadas, funcionan en nosotros como
las disposiciones con las que valoramos e interpretamos la realidad.\\[0pt]
Hay aquí un proceso circular consistente en que con nuestra formas
instintivas de evaluar nosotros valoramos e interpretamos la
realidad. A su vez , esas formas instintivas nuestras de valorar son
ellas mismas el producto de valoraciones e interpretaciones
originarias.\\[0pt]

A la luz de esta condición deberán interpretarse dos características
básicas de la pulsión. La primera, su carácter imperativo: La pulsión
es una presión más o menos fuerte, a veces irresistible y, en todo
caso, imposible de anular o neutralizar salvo por la puesta en juego
de otros impulsos que desvíen la fuera del primeo hacia otras metas
que no sean las de su satisfacción inmediata —proceso de
sublimación. Los valores son tendencias incrustadas en la vida misma
del cuerpo capaces, por tanto, de ejercer una fuerza destinada a
orientar nuestro comportamiento de manera imperativa, es decir,
obligándonos a realizar cierto tipo de actos e impedir la realización
de otros.\\[0pt]

La segunda característica de la pulsión es que de este modo regula el
comportamiento de los individuos con una seguridad incomparablemente
mayor a como podría hacerlo la razón. Por su carácter imperativo, la
pulsión o el instinto ejerce su acción de modo infalible, automático y
sin la menor vacilación, en contraste con lo que sucede con una acción
guiada por la reflexión racional y la conciencia. Hay así una estrecha
relación entre la eficacia perfecta, la seguridad, de la pulsión
irresistible y el carácter infraconsciente de su regulación.\\[0pt]

Resumiendo: La perspectiva genealógica Nietzschiana mira las
formaciones de la conciencia o de la cultura en su condición de
lenguajes figurados, de síntomas en los que se traducen los dos
procesos fisiológicos en virtud de los cuales la vida desarrolla su
impulso fundamenta de autoexpandirse u autoproyectarse creativamente
en las formaciones de cultura.\\[0pt]
El propósito del método genealógico es, pues, desvelar, a partir de
cualquier manifestación consciente o de cualquier proceso de la
cultura, la actividad inconsciente de los instintos que los originan, y
comprender todas sus manifestaciones conscientes como textos a
descifrar en cuanto semiótica profunda de los estados del cuerpo.\\[0pt]

Desde una perspectiva metafísica dual, que caracteriza nuestra
tradición occidental y que distingue una determinada valoración moral
entre espíritu y cuerpo, surge la siguiente cuestión. El método
crítico de Nietzsche significa algo más que una burda reducción del
conocimiento, de la cultura y del espíritu y de la conciencia a meros
subproductos derivados de procesos fisiológicos.\\[0pt]

Pero N. intenta salir de esta visión platónico-dualista, sin la
presión del desprecio moral del cuerpo que el ser humano representa.\\[0pt]
Se podría ver el cuerpo como el fenómeno entero del cuerpo, superior a
nuestra consciencia, a nuestro espíritu, a nuestro pensar, sentir,
querer conscientes como el álgebra a la tabla de multiplicar.\\[0pt]
Cuando se ve al ser humano como un todo de cuerpo-espíritu entonces no
hay problema en comprobar que la conciencia o el espíritu aparezcan,
en ese todo, como una parte más.\\[0pt]

Llega a la misma conclusión desde la perspectiva de lo
apolíneo-dionisíaco. Dionisos era el nombre que, entre los griegos,
habría recibido la energía elemental de autoafirmación de la vida que
se caracteriza principalmente por la sobreabundancia de fuerza y la
desmesura. Apolo, en cambio, designaría las formas luminosas de la
apariencia que seducen a existir formas que nacen y se sumergen de
nuevo en el flujo terrible y anonadante del devenir, en aquella
voluntad de devenir que simboliza Dionisos y que irradia de la
profundidad misma del cuerpo.\\[0pt]
El N. maduro hace crítica de las obra de juventud y aclara que la
relación entre lo apolíneo y dionisíaco no es un conciliación, sino
que es una relación de una lucha, el de una oposición que enfrenta a
ambos impulsos, los divide a la vez que los une y que, por no
encontrar nunca una síntesis, están constantemente en devenir.\\[0pt]
Es decir, el sustrato del cuerpo como vida no es otra cosa que una
lucha a la que se aplicaba , en su juventud, aquella comprensión
primera de la reciprocidad agonal (de lucha) de lo apolíneo y de lo
dionisíaco u de su copertenencia al eterno proceso creador y
destructor de la naturaleza.\\[0pt]

Empieza N. a expresar su idea de que lo que hay en el fondo de toda
conciencia y de toda forma cultural es la polaridad básica de los
impulsos con anclaje en el cuerpo: placer y dolor, o dualidad
vida-muerte.\\[0pt]
En esta polaridad de impulsos de vida e impulsos de muerte tiene
origen último lo que llamamos cultura, aquello en lo que consideramos
que el ser humano se diferencia del simple animal. Nietzsche remarca
el dinamismo básico que rige a esos impulsos, la tipología propia de
su comportamiento y de su modo de funcionar: \textbf{la fe en los afectos}.\\[0pt]

Los afectos son una construcción de la inteligencia, una invención de
causas que no existen. Todas las sensaciones comunes del cuerpo que no
comprendemos son interpretadas de forma inteligente, se busca en las
personas, en las vivencias, etc., una razón para sentirse de una
determinada manera.\\[0pt]
Se establece algo nocivo, peligroso como si fuera la causa de nuestro
malestar: en realidad se lo busca y se añade a nuestro malestar para
poder pensar nuestro estado.\\[0pt]
Aflujos frecuentes de sangre en el cerebro asociados a una sensación
de sofoco se interpretan como ira. Las personas y las cosas que
estimulan en nosotros la ira desencadenan este estado fisiológico.\\[0pt]

En un segundo momento ciertos procesos y sensaciones comunes se
asocian así regularmente de modo que ante ciertos procesos se produce
aquel estado, aquel sentimiento común, e implica en particular esos
aflujos de sangre, esa excitación, etc. O sea, por afinidad. Entonces
decimos que el afecto se ha excitado. En el placer y el displacer hay
ya juicios: los estímulos se distinguen según promuevan o no el
sentimiento de poder.

\section{Nietzsche ilustrado}
\label{sec:org502a45f}
En \emph{humano, demasiado humano} el pensamiento de Nietzsche inicia su
período ilustrado.\\[0pt]

Podemos pensar la Edad Moderna filosófica como edad de la crítica, en
oposición a la Edad Media que sería la edad de la fe. La crítica
significa la no aceptación confiada de las cosas tal como están o tal
como se nos dicen. Significa disconformidad, disposición al cambio y a
a transformación.\\[0pt]

El concepto y la práctica de la crítica se desarrolla con Descartes y
su duda metódica y siguen un proceso de crecimiento y de
radicalización, distinguiéndose tres fases. En un primer estadio lo
representan la revolución científica del s. xvi, el racionalismo y el
empirismo del s.xvii, con sus análisis críticos del proceso de
conocimiento, de la moral y de la política. Una segunda fase de
esplendor representada por el movimiento Ilustrado s.xviii, en el que
se encuadran expresiones tan importantes de esta actitud como las
críticas kantianas o la revolución francesa. Finalmente una tercera
fase de radicalización en los s. xix y xx en la que la crítica no
puede evitar aplicarse ella misma a sus propios presupuestos.\\[0pt]

Nietzsche puede ser considerado un ilustrado en estos dos sentidos
precisos: Primero en que trata de mostrar, mediante un determinante
método crítico, la iconsistencia de la metafísica, de la religión y de
la moral dogmáticas, incluyendo también bajo estas categorías el
pensamiento de algunos de los más significativos filósofos
ilustrados.\\[0pt]
Segundo, en que considera necesario completar la Ilustración de la
razón mediante la tarea de una reeducación y saneamiento de los
impulsos.\\[0pt]

A la perspectiva kantiana de un pensamiento autónomo capaz de
autodeterminación, Nietzsche añade es que la crítica ilustrada no
puede dejar fuera de su ámbito a la moral. No dejarse guiar por otro
implica liberarse de muchos prejuicios morales que determinan nuestro
entendimiento impidiendo su autodeterminación.\\[0pt]
Por ejemplo, N. sostiene que en la iglesia además de la no-verdad hay
que percibir la mentira, hacer que la Ilustración entre en el pueblo
tan a fondo que los sacerdotes lleguen a tener mala consciencia.\\[0pt]
Análogamente hay que hacer con el Estado. Es tarea de la ilustración
hacer ver a los príncipes y políticos que s entera manera de obrar es
una mentira intencionada, privarlos de buena consciencia, y sacar del
cuerpo del hombre europeo esa hipocresía inconsciente. La nueva
ilustración, contra la iglesia, los sacerdotes, los políticos, los
bonachones, los compasivos, los cultos de lujo, contra la hipocresía.\\[0pt]

Lo ilustrado en Nietzsche es esta implacable decisión de sacudir al
pequeño burgués europeo moderno para que vea la insensatez de sus
propias máscaras, y burlarse de quienes, tras el crepúsculo de sus
dioses y la disolución de sus valores absolutos, aspiran ahora a
volver al estado de naturaleza recuperando lo originario como nuevo
fundamento.\\[0pt]
Lo que algunos consideran lo antiilustrado de N. no sería sino ese
efuerzo por aplicar la misma crítica ilustrada a aquellas de sus
premisas tácitas que la envuelven en una falsa consciencia
ideológica. Por esta razón podría decirse también que lo que
N. propone es una nueva Ilustración:\\[0pt]
1.- El descubrimiento de los errores fundamentales —tras los cuales
están la cobardía, la negligencia y la vanidad del hombre. como por
ejemplo por lo que concierne a los sentimientos (y al cuerpo); el
extravío de los espíritus puros; el animal en el hombre, moralidad
como doma; Dios y más allá como fundamentos falsos del impulso
conformador; conocimiento puro; etc.\\[0pt]
2.- El segundo nivel el descubrimiento del instinto creador, también en
sus escondrijos y degeneraciones.\\[0pt]

En esta etapa de transición entre el N. joven y el maduro, lo que
adquiere forma y método es este concepto de filosofía cuyo objetivo es
liberar al pensamiento de los errores de la moral. El
desenmascaramiento de la moral, practicada desde la exigencia de
nuevos valores, tiene el sentido de intentar establecer los supuestos
para el nacimiento y afirmación de una nueva moral. Lo primero que lo
caracterizaría es que no se concebiría ya a sí misma ni se prendería
como absoluta, sino que establecería la diferencia entre lo buen y lo
malo en relación con su circunstancia histórica. La jerarquía de los
bienes no ha sido siempre la misma en todas las épocas. Por ejemplo,
la venganza podría ser considerada como moralmente buena en una
cultura más antigua e inmoral según la cultura actual.\\[0pt]
Inmoral significa pues, que uno no es lo suficientemente sensible a
los motivos superiores, más sutiles y espirituales que toda cultura
lleva consigo.\\[0pt]
El verdadero ilustrado sería el que es capaz de comprender la moral
sobre la base de su historicidad, porque se ha librado ya de la
necesidad de postular su atemporalidad y su carácter absoluto. En
términos de N. sería el \textbf{espíritu libre}.\\[0pt]

Se llama espíritu libre a quien piensa de manera distinta a lo que se
esperaría ateniendo a sus orígenes, su entorno, su posición social y
su oficio, o a las opiniones dominantes de la época. Él es la
excepción, los espíritus sometidos a la regla: ellos le recriminan que
sus libres principios derivan de la manía de llamar la atención, o
incluso que parecen revelar acciones libres, acciones incompatibles
con la moral sometida. Además, no es propio de la esencia de un
espíritu libre el tener opiniones más justas moralmente sino más bien
el haberse libreado de la tradición, con éxito o no. Pero normalmente
tendrá de su parte la verdad: él exige razones, los demás fe.\\[0pt]

La importancia de punto de vista histórico radica, para este espíritu
libre, en que le permite comprender la función que los errores de la
moral han cumplido en la evolución de la cultura. Sin ellos, el ser
humano no habría podido superar el nivel de animalidad.\\[0pt]
La \textbf{moral aparece como un error necesario} para la vida, al mismo
tiempo que la hace estar subordinada a esa necesidad, lo cual destruye
sus pretensiones de absoluto incondicionado.

La \textbf{verdad para el espíritu libre} no es más que la búsqueda de la
verdad , y ésta es la clave que introduce N. en su pensamiento y
obra. Por un lado, el nuevo filósofo, el espíritu libre se transforma
en caminante que explora y que busca. Por otra, requiere la guía y el
apoyo de la ciencia valorada como antimetafísica, capaz de servir la
tarea de destrucciń de las ilusiones y engaños que han servido de
fundamento a la moral dogmática.\\[0pt]

En sus obras \emph{humano, demasiado humano} y \emph{la gaya ciencia}, N. emplea
la ciencia como instrumento de la crítica, lo que significa una
interpretación crítica de la misma ciencia. El saber es una tarea de
búsqueda cuyo objetivo es la liberación del peso de la tradición
especialmente de la moral y sus valores superiores.\\[0pt]

En \emph{Aurora}, N. habla de una autosuperación de la moral desde el
impulso de la veracidad. Debido a esta veracidad, derivada de la
moral, nos volvemos incapaces de creer en los valores supremos de la
antigua moral. El que se entrega a la pasión del conocimiento
experimenta dentro de sí un nuevo deber moral, el deber de la
veracidad que le conduce justamente a perder la confianza en la moral
y producir en sí mismo su autosuperación.\\[0pt]

Desde la pasión por el conocimiento en \emph{Humano, demasiado humano} y en
\emph{Aurora}, se llega en \emph{la gaya ciencia} a la voluptuosidad
(:complacencia en los deleites sensuales) del saber, que tiene como
condición la facultad de soportar el sufrimiento.\\[0pt]
La ciencia produce sufrimiento cuando pone al descubierto realidades
que exigen la autotransformación y el abandono de actitudes
confortables que deben quedar superadas. a tarea del conocimiento se
convierte entones en un proceso de experimentación.\\[0pt]
Pero la realización satisfactoria de este proceso tiene una condición:
Desprenderse de la servidumbre milenaria a los ideales y valores
trascendentales superiores y metafísicos para conquistar la libertad.\\[0pt]

En suma, la muerte de Dios es la premisa necesaria para la creación de
la nueva moral y de sus nuevos valores.

\section{Configuración del método genealógico}
\label{sec:org7ecccb2}
La genealogía es el intento de comprender los fenómenos de la cultura,
e.g. la moral, mostrando el ángulo pulsional o instintivo desde el que
son explicables.\\[0pt]

Nietzsche —a diferencia de los anteriores filósofos que intentarion
destruir las pasiones y los instintos para poder escapar a su poder
condicionante— propone prestarles atención, averiguar la lógica que
los rige y reconocer su función determinante en toda interpretación y
en toda valoración.\\[0pt]
Así, por un lado, el concepto de genealogía significa que toda
creación cultural —moral, arte, religión, política, ciencia, técnica,
etc.— es proyección de fuerzas elementales, orgánicas, relativas a un
determinado nivel de energía o de fuerza vital. Por otro lado,
genealogía significa que, en último término, es siempre el cuerpo
quien interpreta.\\[0pt]
Por tanto el cuerpo es lo anterior a toda objetividad, el \emph{a priori}
de toda creación y transformación de la cultura, la fuerza que se
ejerce y se siente de forma no reflexiva en cuanto autofirmación de sí
misma.\\[0pt]

Luego, la moral, política, ciencia, etc. construyen un mundo de
valores, conceptos o normas que se distancian de esta fuerza primitiva
de la autoposición de sí que parte del cuerpo. Es decir, estas
construcciones teóricas se desligan del sujeto que les ha dado origen
y se sitúan en un plano objetivo. De este modo, las ideas, leyes,
instituciones, etc. se perciben ilusoriamente como hechos o
subjetivos, como entidades independientes del sujeto cuando el
individuo ha olvidado la condición que tienen de ser meras
construcciones suyas.\\[0pt]

Desde esta perspectiva, la verdad es una colección de metáforas,
antropomorfismos en movimiento. Es decir, una colección de relaciones
humanas que han sido realzadas extrapoladas, adornadas poética y
retóricamente y que, tras un prolongado uso, a un pueblo le parecen
fijas, canónicas, obligatorias: Las verdades son ilusiones que se ha
olvidado que lo son, metáforas que se han quedado gastadas y sin
fuerza sensible, monedas que han perdido su imagen y que ahora ya no
se perciben como monedas sino como simple metal.\\[0pt]
Es decir, el marco del método genealógico es una fisiología
trascendental, basada en la reciprocidad al eterno proceso creador y
destructor de la \emph{physis} misma.\\[0pt]

Así, lo que hay en último término en el trasfondo de toda
interpretación no es otra cosa que la polaridad básica de las fuerzas
fundamentales de la vida que animan al cuerpo:
\begin{itemize}
\item alegría-placer: la fuerza se expande, crece, vence, desahoga
creativamente.
\item dolor o pena: Cuando la fuerza es vencida, oprimida, impedida por
una resistencia contra la que se debilita luchando en vano.
\end{itemize}

La \textbf{unidad viviente} de la \emph{physis} de nuestro cuerpo es esta unidad
vida-muerte. Las células de nuestro cuerpo nacen y mueren, luchan
entre sí, crecen y se debilitan, de modo que nuestra vida es al mismo
tiempo también una muerte continua.\\[0pt]
Así, placer-dolor, equilibrio-desequilibrio, victoria-derrota,
etc. son las sensaciones elementales que subyacen a nuestra escala de
valores morales, a nuestras teorías científicas vigentes, a nuestras
creencias religiosas y a nuestras preferencias o antipatías
políticas. Preceden a todos los juicios lógicos, racionales y
conscientes que determinan funciones tenidas como más elevadas o
superiores a la sensación.\\[0pt]

Este nuevo punto de vista de Nietzsche de análisis y refleión sobre la
moral, podría valorarse como un revolución copernicana axiológica
(:teoría de valores): No es la moral la que hace al ser humano, sino
el ser humano el que hace la moral.\\[0pt]
No hay valores en sí, un bien y un mal incondicionados y como formando
parte de la estructura del ser mediante cuyo cumplimiento y
conformación con ellos el ser humano se humaniza y encuentra su
dignidad como persona, sino que el hombre es el que produce los
valores, los instaura lo mismo que produce el conocimiento el arte o
la técnica.\\[0pt]

AL no haber advertido esto, Nietzsche acusa a todos los filósofos
anteriores de superficialidad e hipocresía. Por muy radicales y
críticos que se manifestasen a la hora de elucidar racionalmente una
doctrina o una tesis dada, siempre silenciaban y dejaban sin analizar
el trasfondo de los motivos últimos de las posibilidades de
articulación de los valores y de las decisiones teóricas que toda
doctrina y toda tesis conlleva.\\[0pt]
El propósito de justificarlo todo y de dar razón de todo se dejaba sin
justificar ni analizar las condiciones mismas que hacen posible el
funcionamiento del pensamiento y comportamiento racional: por qué es
preferible lo verdadero a lo falso; por qué es preferible lo estable,
inmutable, el ser al devenir, el cambio, la transitoriedad; por qué es
o debe ser preferible siempre y de forma absoluta lo bueno a lo malo;
por qué hay que creer y aceptar un bien en sí opuesto a un mal en sí
como formando parte de la estructura misa de la realidad.\\[0pt]
Nada de esto se analiza en ningún momento, sino que son cosas que se
afirman, sin más, dogmáticamente.\\[0pt]
\end{document}