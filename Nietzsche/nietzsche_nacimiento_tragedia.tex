% Created 2023-08-22 dt. 13:25
% Intended LaTeX compiler: pdflatex
\documentclass[a4paper, 10pt, twocolumn, spanish]{article}
\usepackage[utf8]{inputenc}
\usepackage[T1]{fontenc}
\usepackage{graphicx}
\usepackage{longtable}
\usepackage{wrapfig}
\usepackage{rotating}
\usepackage[normalem]{ulem}
\usepackage{amsmath}
\usepackage{amssymb}
\usepackage{capt-of}
\usepackage{hyperref}
\usepackage[T1]{fontenc}
\usepackage[margin=.75in]{geometry}
\setlength\parindent{0pt}
\author{Jordi Serra}
\date{\today}
\title{Apuntes de Teoría del Conocimiento ii\\\medskip
\large Sobre Nietzsche}
\hypersetup{
 pdfauthor={Jordi Serra},
 pdftitle={Apuntes de Teoría del Conocimiento ii},
 pdfkeywords={},
 pdfsubject={},
 pdfcreator={Emacs 27.1 (Org mode 9.6.2)}, 
 pdflang={English}}
\begin{document}

\maketitle
\tableofcontents


\section{Sobre}
\label{sec:orga99b898}
Desde niño Nietzsche había sentido que el objeto principal de la
filología no era tanto la simple adquisición y acumulación de un saber
erudito, sino la formación, la \emph{Bildung}. Esto es, la autoeducación y
la autorrealización.\\[0pt]
Como consecuencia del enorme auge que conocieron los estudios
históricos en Alemania y la nueva concepción de la historia que aportó
el desarrollo del historicismo del s.xix, la filología clásica acabó
siendo integrada en el conjunto de las disciplinas universitarias como
ciencia histórica.\\[0pt]
Nietzsche se resiste a que la filología clásica quede relegada y
neutralizada como simple disciplina especializada y dominada por el
positivismo metodológico en las aulas académicas.\\[0pt]
Para Nietzsche, la filología clásica, además de la preparación de los
textos desde el punto de vista de la crítica contextual y del
comentario, tiene por finalidad recuperar del modelo griego la idea de
una educación como construcción de la individualidad que haga posible
reconducir la fragmentación del hombre y la sociedad modernas a su
unidad originaria.\\[0pt]
La filología por sí sola no puede llevar a cabo la integración
necesaria entre investigación especializada y comprensión del
conjunto, por lo que debía completarse con el recurso a la
filosofía. En esta época, la filosofía se entendía como un recurso a
para un ideal del saber como un todo para que el conjunto orgánico de
sus ramas especializadas reflejase la idea de una identidad cultural
que recubriera y suturar la fragmentación política de hecho.\\[0pt]


\section{El nacimiento de la tragedia}
\label{sec:orgd93986d}
El Nietzsche filólogo y académico fue formado bajo la tutela rigurosa
de Ritschl en técnicas positivas del estudio de la Antigüedad. Sin
embargo se adhirió al proyecto artístico de Richard Wagner que iba en
una dirección distinta, si no contraria.\\[0pt]

En los tiempos de sus nombramiento como catedrático en Basilea
coincidió con el encuentro con Wagner, con quién se comprometió con la
justificación ideológica y la difusión del proyecto wagneriano de una
renovación de la cultura alemana.\\[0pt]

Durante esta etapa el trabajo de Nietzsche queda determinado entre una
visión estrictamente filológica y por las exigencias de su actividad
pública en favor de la causa Wagneriana. Sus inclinaciones filosóficas
chocaron con las funciones y obligaciones de un profesor de filología
y con los compromisos y tareas de un propagandista wagneriano, hasta
que en 1875-1876 se acabe alejando definitivamente de Wagner y
abandonde la universidad para realizar sus aspiraciones iniciando su
vida de filósofo errante.\\[0pt]

\emph{El nacimiento de la tragedia} tuvo su origen en una propuesta de
Ritschl a Nietzsche para que escribiese una obra filosófica seria,
para justificar el nombramiento de su cátedra sin aún haber escrito su
tesis doctoral.\\[0pt]
Nietzsche redactó una serie de conferencias en que se basaría la obra,
pronunciadas durante el invierno del 1870-1871. Pero ya en estas
conferencias, N. empieza a mostrar el wagnerismo como elemento
unificador de la diversidad de temas filológicos entre los que N. se
movía en este momento y que aún no se había atrevido a hacer aflorar
en sus trabajos académicos.\\[0pt]
No podía ceñirse ya a las estrictas exigencias metodológicas de la
filología y se veía empujado a recurrir al arte, lo que significaba el
distanciamiento de su entorno académico y su opción por Wagner para
sumarse activamente a su empresa de renovación artística de la cultura
alemana.\\[0pt]

La causa de Wagner era la búsqueda de la condiciones del surgimiento
de una nueva cultura artística en el marco de las nuevas
circunstancias creadas por la llegada del capitalismo y las ideologías
igualitarias traídas por la modernización, la industrialización.\\[0pt]
Su planteamiento prolongaba y exaltaba algunos de los puntos más
significativos del movimiento romántico alemán, con los que había
construido un contexto ideológica en el que encajaba la creación y
representación de su música y desde el que justificaba su alcance
culturalmente renovador y revolucionario. Partía de la conciencia de
estar ante un cambio de época, producido por un agotamiento de la
Ilustración, y de la necesidad, para la civilización presente, de
activar las fuerzas con las que debía intentarse superar esa crisis.\\[0pt]

Nietzche, en su función de justificador ideológico de la obra de
Wagner, presentó las óperas de Wagner como reactualizaciones de la
tragedia griega. Su obra pivotaba entre la ciencia —la filología
clásica y estudio metodológico del mundo griego—, y arte —la ópera de
Wagner— y filosofía —su influencia de Schopenhauer:

\begin{quote}
Dentro de mí, ciencia, arte y filosofía crecen juntos de tal formo que
corro el riesgo de parir un centauro.
\end{quote}

\subsection{Dioniso y Wagner. El extraño centauro}
\label{sec:orgfc6fa9d}
Para N., entender el presente a partir de la antigüedad griega suponía
confrontar la época moderna con la época trágica. En su opinión ni el
clasicismo ni el romanticismo habían logrado comprender la cultura
griega. La comprensión íntegra de los griegos clásicos sólo se logra
cuando se descifra el sentido, la función y el alcance de lo que fue
la tragedia clásica.\\[0pt]

El olvido de lo trágico de a modernidad no es algo accidental. No es
debido a la ignorancia ni al descuido, sino que es algo constitutivo
de y consustancial a lo más propio y esencial de la modernidad.Así,
por \emph{vuelta a los griegos} habría que entender un trabajo de revisión
y de reactualización para sobreponerse a este olvido.\\[0pt]
Conocer la cultura griega sería percibir y conectar con esa \emph{fuerza}
\emph{plástica} de un ser humano, de un pueblo, de una cultura, la fuerza
de desarrollarse de manera original e independiente a partir de si
mismo, de transformar y asimilar lo pasado y lo extraño, de cicatrizar
heridas, reponer lo perdido regenerar formas destruidas. E los
griegos, esa fuerza plática fue la visión trágica del mundo, y en los
modernos su olvido. En ambos es posicionamiento vital y autoformación
a partir de una determinada actitud ante la temporalidad como devenir
constitutivo del mundo.\\[0pt]

El \emph{nacimiento de la tragedia} comienza con el intento de desmontar
piedra a piedra el edificio construido sobre la cultura apolínea,
hasta llegar a los mismo fundamentos en que se basa. Se propone
denunciar la ingenuidad de los \textbf{clásicistas}, admiradores de la bella
apariencia y mesura griegas, y obligar a los románticos a volver sus
ojos desde su infinito especulativo hacia un horizonte para ambos
inédito. Para los clasicistas, los antiguos griegos eran un modelo de
arte y de humanidad porque armonizaban naturaleza y libertad. Se daría
algo así como una relación inmediata, no conflictiva, entre hombre y
naturaleza que se expresa de manera natural en la perfección de su
poesía, arte e instituciones políticas.\\[0pt]
Este equilibrio griego entre lo natural y lo humano se habría roto con
el auge de la filosofía, habiendo sido la causa misma de esta ruptura
un desarrollo excesivo de la razón que empieza con Sócrates y Platón,
y que alcanza su apogeo en la época moderna. Este hiperdesarrollo de
la razón propicia que lo subjetivo comience a dominar sobre lo
objetivo, y que la reflexión y la racionalización traten de abarcarlo
todo. El individuo se escinde y toda su fuerza se concentra en el
entendimiento de la verdad, mientras que sus instintos se debilitan
cada vez más llegando a perder casi por completo su poder.\\[0pt]

El \textbf{romanticismo alemán} en respuesta al planteamiento clasicista,
sostenía que la idealización de lo griego no debía conducir a una
comprensión negativa de la modernidad, como expresión de una
decadencia, sino que habría que considerar la modernidad como un
verdadero progreso en cuanto a conquista histórica de libre
subjetividad.\\[0pt]
Habría que aspirar, pues, a una situación de síntesis en la que la
humanidad reencontrara la perfección y el equilibrio tal como existía
en la época griega. Pero entendiendo esto como un producto del
desarrollo de la libertad y de la autoconsciencia propio de la
modernidad. La \emph{edad de oro} no está detrás, en lo griegos y al
principio de nuestra historia, sino delante, no siendo la historia
otra cosa que el ámbito de un desarrollo dialéctico hacia lo absoluto
impulsado por el antagonismo entre naturaleza y libertad.\\[0pt]


Nietzsche contesta con un mismo argumento a ambas posiciones,
clasicista y romántica: La bella apariencia de lo apolíneo no es sino
el anverso de la profundidad insondable de lo dionisíaco. La
ejemplaridad griega no estriba en su sentido ingenuo de lo bello, sino
en el modo como los griegos lograron sobreponerse a los aspectos
desmesurados, terribles y trágicos de la existencia. Dioniso es la
expresión de un estado primitivo y salvaje, dominado por la desmesura,
que precede a la formación de la civilización griega, la cual lo
sometió a la belleza de la mesura con sus dioses olímpicos y su arte
apolíneo. Lo apolíneo encuentra justificación en lo dionisíaco.\\[0pt]
El emerger de la civilización griega es un acto de soberanía y
dominación sobre la violencia de la barbarie arcaica. los griegos
clásicos pudieron desarrollar una actitud afirmativa de la vida,
lograron hacer deseable la existencia sin apartar la mirada del
sufrimiento y de la muerte que conlleva.\\[0pt]
Esto lo consiguieron transfigurando lo terrible con su arte y con su
religión estética. Lo que salva a Grecia de la potencia destructiva
dionisíaca es su capacidad de idealización, expresada sobretodo en el
arte. Lo apolíneo sirve a lo dionisíaco para que la desmesura de lo
dionisíaco se convierta en una experiencia superior. Y esto es lo que
llevó a cabo la tragedia.\\[0pt]
Por esto, tampoco podía N. aceptar la comprensión Dionisos como un
nuevo dios de la redención asimilable a Cristo, desde el que sería
posible construir una nueva mitología y un nuevo mesianismo (relativo
al mesías) de la época moderna.

\subsection{El drama musical}
\label{sec:orgc634331}
Sin embargo, el joven N. coincide con clacistas y románticos en que
para aspirar a una cultura unificada capaz de reproducir totalidades
en todas sus expresiones, o sea, para salvar el hombre moderno de su
fragmentación, es preciso volver a la fuente de toda creatividad, al
principio del que brota la fuerza que produce las obras de arte.\\[0pt]
La tragedia griega esla forma de arte más elevada y cumbre de la
evolución artística y de la civilización de los griegos porque,
gracias al equilibrio en ella de los elementos apolíneo y dionisíaco,
lo dionisíaco puede salvarse y conservarse como experiencia
originaria.\\[0pt]

Lo que unifica a los individuos singulares y hace que se sienta una
unidad disolviendo momentáneamente el principio de individuación es
esta experiencia que produce la celebración trágica y que consiste en
la simultaneidad de la exaltación y el significado. Es aquí donde
N. ve la culminación de la civilización griega. El hombre griego podía
reconocerse en el espejo de sus dioses olímpicos, en un mundo dominado
por la mesura de la belleza per porque el arte trágico transfiguraba
el fundamento pulsional de lo dionisíaco-natural, con lo que la
tragedia dionisíaca lograba el fin supremo del ate: Proyectar sobre la
cultura, las instituciones y la diversidad de los individuos un
sentido superior de unidad que lleva al corazón de la naturaleza.\\[0pt]

Esta experiencia de unidad es lo que se pierde en la modernidad. En la
modernidad se pretendía, mediante la ciencia y la técnica, convertir
al hombre en una fuerza capaz de dominar el mundo, bajo dos
condiciones. La Primera, el desencantamiento de la naturaleza, a
disolución de su misterio y su reducción a pura máquina. Y segunda, el
extrañamiento del hombre respecto de su propio cuerpo, convirtiéndolo
en mero artefacto o máquina.\\[0pt]

El romanticismo alemán (Goethe, Schiller, Herder, etc.) sostenían que
la razón no es ni puede ser el elemento fundamental sobre el que deba
construirse la cultura, simplemente porque deja fuera lo otro de la
razón, i.e., todos los factores impulsivos y no conscientes que
representan la energía básica de la vida, y que una cultura sana y
plena no debe reprimir no contradecir. Para ellos, una cultura mejor
se basa en un desarrollo en el que la naturaleza y el espíritu se
sintetizan como empresa de formación (\emph{Bildung}) de la humanidad.\\[0pt]
La poesía no es solo un modo de pensamiento precientífico e
imperfecto, sino el lenguaje de la imaginación en el que toma forma la
vitalidad de la naturaleza a través del genio. Análogamente pasa con
la mitología, que no es ese otro de la razón según la Ilustración,
sino que es el lenguaje de la fantasía como lenguaje no racional de la
vida.\\[0pt]
Así el arte es algo anterior a la razón. Sólo una vez producido un
lenguaje mediante una síntesis creadora de la imaginación, se le puede
aplicar la razón analítica y descomponerlo en sus elementos más
simples. Así, el análisis presupone siempre la síntesis. El logos
presupone el mito y se relaciona con él en un movimiento de vaivén.\\[0pt]
Así, el arte puede permitir la recuperación de una objetividad por
parte de la subjetividad moderna, por el cual puede volver a
apropiarse de la naturaleza por parte del espíritu.\\[0pt]
Lessing establecía las fronteras de la pintura y la posesía señalando
la particularidad de cada una de ellas, frente a quienes consideraban
la pintura como expresión plástica de la poesía y la poesía como
expresión verbal de la pintura.\\[0pt]
Pero Nietzshe se esforzaba para superar esta oposición por otra más
fundamental desde su punto de vista. Lo que separa las artes (música,
artes plásticas, etc.) no es tanto su forma fenoménica de lo que
representan sino el tipo de impulso que actúa en cada arte.\\[0pt]

En el \emph{nacimiento de la tragedia}, la contraposición entre lo apolíneo
y lo dionisíaco, que se enamrca en la metafísica schopenhaueriana de
la voluntad como ser del que surgen los fenómenos del mundo de la
representación, el arte apolíneo (las artes plásticas pero también la
poesía) tiene como su carácter más propio la mediación de la imagen u
de la palabra con las que se simboliza un ser o figura determinada.\\[0pt]
Apolo es el dios griego del principio de individuación, de la mesura y
de la claridad. El arte dionisíaco, la tragedia y la música, tiene un
carácter y un origen diferentes con respecto a todas las demás artes,
porque ella no es, como todas éstas, reproducción del fenómeno, sino
de manera inmediata reproducción de la voluntad misma. Representa, con
respecto a todo lo físico del mundo, lo metafísico. Y con respecto a
todo fenómeno, la cosa en sí.\\[0pt]
Lo apolíneo y lo dionisíaco son, para Nietzsche, impulsos o fuerzas
artísticas que brotan de la naturaleza misma y despliegan, con su
oponerse, la dinámica misma del ser.\\[0pt]

Así que cuando N. sostiene que la tragedia griega constituye la forma
más sublime del arte, lo razona diciendo que es así porque reúne en
una unidad la poesía, la música y la danza, expresando el sentimiento
de manera conjunta con la palabra, el ritmo y la gesticulación. Ahora
bien, no basta con que el drama musical incluya la palabra —signo de
racionalidad—, el gesto —expresión corporal—, y el sonido —potencia
comunicativa y sentimental básica—, sino que deben respetar un orden
jerárquico. El gesto y la palabra son expresiones individuales que
hunden sus raíces en la música, lenguaje directo de la pasión. Sólo a
partir de la música como el gesto y la palabra adquieren la
consistencia de poesía y danza, y nunca del revés.\\[0pt]

La tragedia griega, como obra de arte, era una obra de síntesis, por
lo que tenía que poder permitir a los espectadores participar en ella
como seres humanos completos. Era, pues, el modelo para las obras de
Wagner, tanto en lo relativo a sus características estructurales como
en lo referente a las condiciones de su producción y de su
representación. Pues la tragedia griega ja sido la obre de arte que
con mayor eficacia ha logrado convertir su representación en una
experiencia común y colectiva de justificación y de afirmación de la
vida mediante su poder de transfiguración estética. Es decir,
constituye la modalidad más perfecta de fusión del artista, la obra y
el espectador concentrando y proyectando así la fuerza plástica por la
que un pueblo o un ser humano se producen como obra gracias al proceso
de metamorfosis puesto en marcha por una tal obra de arte.\\[0pt]
La tragedia griega clásica es el modelo de la obra de arte capaz de
obligar al conjunto de la sociedad a asumir su propia forma. No se
asiste a ella desde la actitud individualista y puramente
contemplativa de un disfrute pasivo o de una perspectiva crítica
exterior a la obra, sino que, con su representación, transforma a una
colectividad de espectadores en una individualidad superior, una
comunidad de afirmación en y por la obra compartida, y obliga a entrar
en esa comunidad de la obra para participar en su vitalidad.\\[0pt]

La tesis filológica de \emph{el nacimiento de la tragedia} que sitúa el
origen del género trágico en la lírica dionisíaca como contrapuesta a
una lírica apolínea, entendiendo por lírica dionisíaca lo que es
musical en estado puro. Nietzsche no considera en las representaciones
trágicas de la antigua Grecia, el coro fuera el compendio de la masa
de espectadores, un \emph{espectador ideal}. Sino que consideraba al coro
\emph{un muro viviente que la tragedia trazaba en torno a sí para aislarse}
\emph{del mundo real y custodiar su terreno ideal y su libertad
poética}. Es decir, el coro aporta el presupuesto para la eliminación
de la separación entre representación y público.\\[0pt]
La idea de un público de simples espectadores es una idea moderna,
determinada por la concepción de la experiencia como relación
sujeto-objeto, y en las representaciones trágicas antiguas lo que
llegaba a tener lugar entre obra y público era un estado de
unificación de lo interno y lo externo una reciprocidad entre
representación y experiencia, y la posibilidad de una continuidad
entre lo inconsciente y lo consciente cuya profundidad abría el
horizonte de la temporalidad trágica.\\[0pt]
Pensar el público como simple espectador es asumir el dogma moderno de
la irrebasable distancia como separación actor-público, y como
diferencia artista-crítica entre el mundo y su representación.\\[0pt]
Frente a esto N. ensaya una estética sublime que intenta pensar la
relación de la imaginación productiva con el querer originario. El
poder creador de esta imaginación extraería de la música la fuerza
intensiva de su comunicabilidad y la fuerza extensiva de su finalidad
formadora. Todo arte requiere un estar-fuera-de-sí, un éxtasis; no
retornamos a nosotros mismos, sino que entramos en un ser ajeno,
actuando como si estuviéramos hechizados; el suelo vacila, así como la
fe en la indisolubilidad del individuo.\\[0pt]
La voluntad de cada individuo hace en el éxtasis, la experiencia de un
descentramiento de la consciencia correlativo a la disolución durante
esa experiencia de la arrogancia exclusiva y excluyente de los
privilegios del entendimiento.\\[0pt]

El artista trágico no comunicaba a los espectadores ningún argumento,
ninguna historia ni ningún concepto moral. No le era preciso recurrir
al suspense para mantener la atención de su público. La acción era
extraída de la mitología, por lo que su desenlace ara ya, de antemano,
conocido por todos. El poder de atracción de las tragedias radicaba en
hacer que el espectador se incorporase a la obra como copartícipe y
recreador de la obra misma, de su inspiración y de su vínculo
inmediato con las grandes potencias creadoras de la naturaleza.\\[0pt]
Su primera obra filosófica ensaya, pues, un tipo de reflexión como
\emph{anámnesis} (reminiscencia) desde la que el presente podría cambiar de
rumbo y la naturaleza recuperar su lugar.\\[0pt]
Con ello abre la perspectiva de una filosofía práctica con un
imperativo categórico nuevo, el \textbf{imperativo de genialidad}, por el que
los individuos se elevan sobre sí mismos formándose gracias a grandes
artistas que encarnan el ideal de humanidad integral al mismo tiempo
que lo transmiten. Sugiere un aristócrata intelectual en el que la
naturaleza se supera a sí misma como trascendencia de la
inmanencia. Toda vida de un pueblo refleja, de un modo confuso, la
imagen ofrecida por sus genios más grandes. Éstos no son el producto
de la masa, sino que la masa muestra su efecto.

\subsection{La hipótesis metafísica de la \emph{unidad primodial}}
\label{sec:orga2d43bc}

Según N. la tragedia tiene una mayor vinculación con lo dionisíaco que
las artes plásticas porque incluye a los actores y a los espectadores,
en quienes brotan e interaccionan los impulsos que canalizan y
expresan la fuerza creativa y destructiva primordial de la
naturaleza.\\[0pt]
La estética remite a los estados creativos del sueño y la embriaguez
como aquellos en los que los impulsos artísticos de la naturaleza se
manifiestan en el ser humano, aunque no pueden captarse más que en sus
traducciones y efectos.\\[0pt]
Nietzsche piensa la relación entre la naturaleza y los individuos a
partir de la relación entre \textbf{voluntad en sí y voluntad objetivada}
\textbf{de} \textbf{Schopenhauer}.\\[0pt]
El principio de individuación hace posible que lo que es uno en sí,
i.e. la voluntad, aparezca como múltiple en el espacio y el
tiempo. Así, los impulsos creadores originarios no sean algo exclusivo
del hombre, sino impulsos de la naturaleza misma, lo apolíneo y su
antítesis, lo dionisíaco, como poderes artísticos que surgen de la
naturaleza misma, sin mediación del artista humano, y en los cuales
las pulsiones artísticas de ésta se satisfacen por vez primera y por
vía directa Por un lado como mundo de imágenes de los sueños; por otro
lado, como la realidad embriagada.\\[0pt]

Así pues, para N. la estética no remite sólo a una fisiología, sino
también a una metafísica desde la que el mundo sólo se justifica como
obra del \textbf{artista supremo}, el \emph{Uno primordial}. Pero este artista
supremo, uno primordial, es una simple conjetura, una mera hipótesis
especulativa para poder pensar en un origen y un sentido unificados de
todos los procesos de producción de formas, en cuanto actividad
metafísica de la vida. Esta unidad no constituye un nivel del ser más
allá y por encima de la apariencia, sino que es un nombre para
designar la naturaleza en cuanto fuerza creadora y destructora, en
cuanto devenir del mundo cuyo atributo esencial es ser una
contradicción y un sufrimiento originario.\\[0pt]

\uline{\textbf{Relación entre lo apolíneo y lo dionisíaco}}\\[0pt]
Con \textbf{apolíneo} se designa el permanecer fuera de sí, el estar
embelesado (embaladir) ante un mundo inventado y soñado, ante el mundo
de la bella apariencia, como una redención del devenir.\\[0pt]
Con el nombre de \textbf{Dionisio} se designa el devenir, activamente
aprehendido, sentido subjetivamente, como la furiosa voluptuosidad
—placeres sensuales— del creador que al mismo tiempo conoce la ira del
destructor.\\[0pt]
Antagonismo de estas dos experiencias y de los \textbf{apetitos} que están en
su base. El primero quiere que el fenómeno sea eterno, ante él el
hombre se vuelve sosegado, sin deseos, como un mar en calma, ser
restablece, se pone de acuerdo consigo y con toda la existencia. El
segundo apetito empuja el devenir, a la voluptuosidad de
hacer-devenir, de crear y aniquilar.\\[0pt]
El \textbf{devenir}, visto desde el interior sería el continuo crear de
alguien insatisfecho, extremadamente rico, infinitamente tenso y
apremiado, un Dios que sólo supera la tortura de ser con la
transformación y el cambio permanentes: la apariencia como su
redención temporal, alcanzada en cada instante. El mundo como la
sucesión de visiones divinas y redenciones en la apariencia.\\[0pt]
Esta *metafísica*de artista se contrapone a la unilateral
consideración de Schopenhauer , quien no aprecia el arte desde el
artista sino exclusivamente desde el receptor, porque conlleva
liberación y redención en el gozo de lo no real, en oposición a la
realidad, redención de la forma y en su eternidad. A esto se le
contrapone el segundo hecho, el arte desde la vivencia del artista,
sobre todo del músico —la tortura de tener que crear, como impulso
dionisíaco.
Este sufrimiento se expresa en el carácter destructor y doloroso del
tiempo, hecho de instantes que se autosuprimen y se autodestruyen
continuamente, por lo que ha de comprenderse como algo esencialmente
inherente a la vida. El sufrimiento no es algo individual o azaroso,
sino que está inscrito en el sí mismo del ser —es lo que N. trata de
expresar con el Uno primordial. Así, el tiempo, el devenir no es la
negación del ser, sino el modo esencial de su manifestación.\\[0pt]
El ser no puede ser liberado, corregido de estos atributos esenciales
—el sufrimiento, la destrucción, etc.— como pretenden las concepciones
ilustradas y optimistas de la existencia. Ni tampoco debe ser
rechazado y negado para preferir desear el no-ser abriendo así una
evasión nihilista al dolor y la contradicción del mundo, como hacen el
cristianismo y Schopenhauer.\\[0pt]
Pero N. no se opone a este nihilismo con un heroísmo ingenuo,
desesperado que resista el sufrimiento en sus manifestaciones más
extremas y terribles. Sino que sostiene que tenemos el arte para no
perecer a causa de la verdad. El arte puede justificar la vida como es
y reforzar el querer-vivir en lugar de su negación.\\[0pt]
El sufrimiento, mediante su transfiguración por el arte, y la
contradicción pueden producir un placer superior desde el que es
posible la afirmación de la vida como pesimismo de la fuerza y de la
victoria. Incluso los procesos destructivos y degenerativos, pueden
interpretarse como condición de la más alta afirmación y como
modalidades de realización de esa afirmación.\\[0pt]
El efecto y la función de la obra de arte trágica Una trasfiguración
por la que la representación visible, apolínea, del sufrimiento
(Dionisio) tiene un sentido afirmativo. Lo que hace ser no sólo
soportable, sino estimulante al haber sido transformado en un placer
superior que es el que proporciona su sublimación en las formas
artísticas, serenas y mesuradas (Apoolo) que mitigan y disuelven el
horror.\\[0pt]

No todas las artes tienen el mismo \textbf{poder de transfiguración y}
\textbf{sublimación del sufrimiento} y de lo terrible inherente a la
vida. Hay una relación inversamente proporcional a la plasticidad y
apariencia placentera y concentración de sentido y universalidad.\\[0pt]
La unidad primordial se objetiva primero en la música, y luego, de
forma más mediata, en la poesía u en las artes plásticas.\\[0pt]
Para N. la música es el lenguaje inmanente de la voluntad y sentimos
incitada nuestra fantasía a dar forma a aquel mundo de espíritus que
nos habla. Lo dionisíaco es el impulso a romper los límites de la
individuación y hacer salir el propio ser afuera, para fusionarse con
la unidad primordial.\\[0pt]
Lo apolíneo está en función de lo dionisíaco, para que la potencia
destructiva y disolvente de lo dionisíaco se mediatice en una
experiencia sublimada, exaltada. El ser humano es pues, en cuanto a
naturaleza, esencialmente un creador de formas, creación que se
produce más allá de la simple búsqueda de satisfacción de sus
necesidades y de la mera conservación de su existencia.\\[0pt]

\textbf{Los grados de exteriorización} de la unidad primordial son los
siguientes —de más inmediatos a más mediatos—, la embriaguez, el
éxtasis, la música, la poesía, la danza, las artes plásticas y el
sueño.\\[0pt]
Cada uno de estos estados es una traducción, una transformación de la
fuerza fundamental de la vida y de la naturaleza (Uno), a un estado o
a una esfera cada vez más lejana y exterior.\\[0pt]

Desde esta perspectiva, la tragedia griega es la obra de arte con
mayor poder de transfiguración y sublimación del sufrimiento al
lograr, con la jerarquización que lleva a cabo de sus diversos
componentes artísticos, el máximo de equilibrio entre plasticidad y
sentido, entre sueño y embriaguez.\\[0pt]
El gesto y la palabra hunden sus raíces en la música, adquiriendo su
sentido y consistencia.\\[0pt]
El principal recurso estético de la tragedia griega como totalidad
artística era el modo en que la música completaba la poesía y la
intensificaba para despertar e sentimiento de los espectadores. Por
ello era esencial el coro, que representaba la acción con sus cantos y
movimientos.  Con Sófocles y, sobre todo, Eurípedes, el coro su papel
central a los personajes individuales. La tragedia cambia su sentido,
pasando de la expresión del sufrimiento a través del canto del coro a
la representación de una acción por unos actores. Se convierte en mero
espectáculo.\\[0pt]

En suma, para N. el arte apolíneo representa a la unidad primordial de
manera mediata, mientras que la música, como arte dionisiaco, es la
representación inmediata de la unidad primordial como esencia última
del mundo.\\[0pt]
Las artes plásticas reproducen objetos particulares a través de los
cuales el espectador puede tener acceso a la intuición de Idea
(Schopenhauer).\\[0pt]
La música no reproduce objetos particulares, sino que es independiente
del mundo de las formas fenoménicas, no accedemos a través de ella a
ninguna Idea, sino que ella tiene ya por sí misma el estatuto que
tiene las Ideas.\\[0pt]
Su sonido nos habla del ser y reproduce analógicamente su estructura y
su espectro. Los sonidos graves y bajos remiten a lo inorgánico. Los
sonidos agudos al reino animal y vegetal, y la melodía vocal a la vida
y a los impulsos conscientes del ser humano.\\[0pt]
Al desarrollarse como un proceso artística en el tiempo y no emplear
ni la imagen ni la palabra, sino sólo el sonido, constituye el único
arte capaz de expresar o de reproducir lo que la vida misma es como
voluntad o cosa en sí: el devenir constitutivo del mundo.

\subsection{Sócrates y el fin de la época trágica}
\label{sec:orgd10fafd}
Sócrates simboliza el espíritu por el que la época trágica es olvidada
y rechazada a los extramuros de la historia. Con Sócrates la cultura
griega se racionaliza y se aleja así de su primitiva forma estética y
mítica, para reglamentarse y conducirse mediante principios críticos y
discursivos.\\[0pt]
Con el descubrimiento socrático de la razón y del mundo ideal de sus
conceptos puros, se creyó haber descubierto la verdadera realidad, en
confrontación con la cual la otra, la que la vida espontánea nos
ofrece, queda automáticamente descalificada. Es decir, la misión del
hombre consiste en sustituir lo espontáneo por lo racional.\\[0pt]

En esta nueva atmósfera, la tragedia muere a manos de una logicización
que la desvirtúa desde su interior amortiguando, neutralizando y
desalojando de ella su subsuelo corporal, pulsional, musical,
originario.\\[0pt]
Convertida en mero espectáculo, los dioses y seres míticos dejan su
puesto en la escena a personajes que despliegan una retórica de
intención pedagógica y moralizante.\\[0pt]

A partir de Eurípedes, la tragedia se convierte en la escuela de la
vida y adquiere un planteamiento racional y discursivo.\\[0pt]
En la nueva época del socratismo, todo debe poder ser razonado y
explicado, y esta nueva exigencia social y política requiere también
nuevos criterios artísticos.\\[0pt]

Los procesos pulsionales e inconscientes de la creación han de quedar
sustituidos por planes conscientes e intenciones críticas. Así, la
inexorabilidad del destino, que acaba aplastando al héroe trágico con
sus desmedida violencia ciega y terrible, es sustituida ahora por un
simple error de cálculo o por un fallo en el razonamiento.\\[0pt]
El pesimismo trágico deja paso a un optimismo según el cual la
desgracia es sólo efecto de la ignorancia o de la impericia, algo
corregible, puesto que del saber se sigue la felicidad.\\[0pt]


Con la ironía socrática muere la tragedia y da comienzo a la filosofía
como arte o saber desligado de la raíz metafísica, pulsional, del
mundo. Antes del racionalismo socrática, el saber filosófico se
expresaba comúnmente a través de la poesía. Era un saber que brotaba
directamente de la vida.\\[0pt]
Con Sócrates triunfa una nueva filosofía como puro ejercicio lógico de
la razón, que encontrará en la modernidad su gigantesco despliegue en
la construcción de la ciencia moderna y de la técnica.\\[0pt]
Sócrates representa la línea divisoria entre la Antigüedad griega y la
modernidad como época de la razón y del olvido de la tragedia.\\[0pt]

Este olvido significa que no el sueño no la embriaguez son ya estados
artísticos cultivados y cultural y socialmente reconocidos e
identificables de comunión con la unidad primordial, con la
naturaleza, con los otros.\\[0pt]

El arte es ahora sólo un entretenimiento trivial y ocioso, las
tragedias son incomprensibles, y el mundo, en sus aspectos terribles y
contradictorios, algo absurdo y moralmente condenable.\\[0pt]

El espíritu socrático, desarrollado por Platón, da lugar al \emph{mundo}
\emph{verdadero} como trasmundo desde el que se condena el mundo de la vida
y se abre a la historia como proceso del nihilismo.

Desde este momento, la religión revelada cristiana es una evolución de
la religión estética en la grecia arcaica y clásica.\\[0pt]
El \emph{Dios ha muerto} (Hegel) expresa el núcleo de desacralización y de
ateísmo agresivo y militantes con los que la nueva religión cristiana
hizo la guerra al paganismo (pagano: no cristiano) hasta su
exterminio.\\[0pt]
Análogamente, Sócrates inventa la antimetafísica del hombre teórico y
abstracto, un saber desligado de su raíz corporal y de su vínculo
natural con el mundo, que cree que la razón, capaz sólo de producir
ficciones útiles, da acceso a un sentido de lo que es en sí y un saber
universal.\\[0pt]

Sólo la música y la experiencia de comunión corporal que permite
sentir de la unidad primordial nos remite y nos refiere a lo
universal.\\[0pt]
La razón es un poder de crear meras ficciones cuyos conceptos no son
más que abstracciones derivadas del mundo empírico, y que no puede
alcanzar la cosa en sí, sino que interpone siempre de modo irrebasable
el mundo como representación.\\[0pt]
Nietzsche propondrá la vuelta a los griegos entendida como movimiento
de des-secularización (secularizar: hacer laico, no eclesiástico), i.e.,
reactualización y afirmación de la temporalidad trágica —el eterno
retorno— como devenir propio del mundo.
\end{document}