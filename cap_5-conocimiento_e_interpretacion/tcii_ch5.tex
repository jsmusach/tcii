% Created 2023-06-01 dj. 21:53
% Intended LaTeX compiler: pdflatex
\documentclass[a4paper, 11pt, twocolumn, spanish]{article}
\usepackage[utf8]{inputenc}
\usepackage[T1]{fontenc}
\usepackage{graphicx}
\usepackage{longtable}
\usepackage{wrapfig}
\usepackage{rotating}
\usepackage[normalem]{ulem}
\usepackage{amsmath}
\usepackage{amssymb}
\usepackage{capt-of}
\usepackage{hyperref}
\usepackage[T1]{fontenc}
\usepackage[margin=.75in]{geometry}
\setlength\parindent{0pt}
\author{Jordi Serra}
\date{\today}
\title{Apuntes de Teoría del Conocimiento ii\\\medskip
\large V Conocimiento, comprensión e interpretación}
\hypersetup{
 pdfauthor={Jordi Serra},
 pdftitle={Apuntes de Teoría del Conocimiento ii},
 pdfkeywords={},
 pdfsubject={},
 pdfcreator={Emacs 27.1 (Org mode 9.6.2)}, 
 pdflang={English}}
\begin{document}

\maketitle
\tableofcontents


\section{Introducción}
\label{sec:org2d096a2}
\subsection{La hermenéutica como método y como estructura el ser}
\label{sec:orgeb5dc76}
Originalmente la palabra hermenéutica designaba la metodología propia
de la interpretación de la Biblia, que se hacía en función de una
explicación lingüística e histórica, si bien no se limitaba sólo a
esto. El término se generaliza después de designar la tarea de
interpretación científica de textos clásicos o difíciles.\\[0pt]

En los dos últimos siglos, Chleiermacher y, sobre todo, Dilthey,
introducen ente término en su teoría de las ciencias del espíritu para
designar la metodología que de debe seguir la comprensión en su tarea
de asimilación de las formas de la cultura.\\[0pt]
En el pensamiento de Heidegger y de Gadamer, la hermenéutica deja de
ser una simple metodología para pasar a aludir a una determinada
concepción ontológica de la realidad.\\[0pt]

Para Heidegger, comprender es la estructura ontológica fundamental del
hombre el cual no hace más que proyectarse de diversas
maneras. Comprender es la sustancia de toda manifestación humana. Es
lo que trata de decir cuando afirma que el modo propio de ser del
hombre es existir como intérprete o desvelador del ser. Desde esta
perspectiva, la hermenéutica se convierte en una realidad ontológica
desvelada por el análisis fenomenológico de la autocomprensión del ser.\\[0pt]

Pero será Gadamer quién elaborará las líneas básicas de esta
ontología hermenéutica. En su obra \emph{Verdad y método} alude ya a su
intento de superación de la distinción entre la comprensión, como
operación propia de las ciencias humanas o ciencias del espíritu, y la
explicación como operación cognoscitiva propia de las ciencias de la
naturaleza.\\[0pt]
Los acontecimientos humanos e históricos se rigen por las leyes del
mecanismo casual. Se hace difícil, si no imposible, aplicar el método
científico, matemático-experimental a este ámbito.\\[0pt]
La comprensión es un acto de conocimiento específico que tiene lugar a
través de la interpretación, y que tiene el carácter de una fusión de
horizontes. Comprender un texto o una obra de arte es un modo de
conocer distinto al modo de conocer de ciencias como la física o la
biología. Cuando se da la comprensión, se produce una convergencia
entre el horizonte del intérprete o lector y el horizonte que expresa
la obra.

\subsubsection{La circularidad de la comprensión según Schleiermacher}
\label{sec:org1c02641}
cuando Chleiermacher supera la particularidad de la reconciliación de
antigüedad clásica y cristianismo y concibe la tarea de la
hermenéutica desde una generalidad formal, logra establecer la
concordancia con el ideal de objetividad propio de las ciencias
naturales, pero sólo al precio de renunciar a hacer valer la
concreción de la conciencia histórica dentro de la teoría
hermenéutica. Frente a esto la descripción y fundamentación
existencial del círculo hermenéutico por Heidegger representa un giro
decisivo. En el s. XIX ya se hablaba de la estructura circular de la
comprensión, pero siempre en el marco de una relación formal entre lo
individual y el todo. El movimiento circular de la comprensión va y
viene por los textos y acaba superándose en la comprensión completa de
los mismos. Con Schleiermacher la teoría culmina en una teoría del
acto adivinatorio mediante el cual el intérprete entra de lleno en el
autor y resuelve desde allí todo lo extraño del texto. Heidegger
describe este círculo en forma tal que la comprensión del texto se
encuentre determinada continuadamente por el movimiento anticipatorio
de la precomprensión. El círculo del todo y las partes no se anulan en
la comprensión total, sino que alcanza en ellos su realización más
auténtica.

\subsection{Una filosofía de la lectura}
\label{sec:org516b893}
Esta idea de fusión de horizontes ha hecho surgir toda una filosofía
de la lectura, desarrollada por un conjunto de autores al que se
conoce como \textbf{Escuela de Constanza}. En esta escuela, precisando el
pensamiento de Gadamer, añaden que lo que la lectura de un texto o de
una obra requiere es una labor de refectuación de las referencias
potenciales que contiene, desde la nueva situación del lector que la
lee.\\[0pt]
Esto significa que una obra escrita debe ser comprendida, también, en
su totalidad. No consiste en una simple sucesión de frases situadas en
un mismo pie de igualdad, ni independientes. Es un todo, una
totalidad. La relación entre todo y partes requiere una comprensión
circular, ya que el supuesto de circularidad está implicado en el
reconocimiento de las partes y de sus relaciones. De hecho, los
distintos motivos o temas de una obra nunca están a la misma altura.\\[0pt]
Esto lleva a sostener que una obra literaria está siempre inacabada en
un doble sentido.

Primero, porque ofrece diferentes visiones esquemáticas que el
intérprete está llamado a conocer. Toda obra presenta lagunas, lugares
de indeterminación que el lector debe cubrir con una labor de
interpretación. Por muy articuladas que puedan estar las visiones
esquemáticas propuestas en la obra para su ejecución por el lector, la
obra es siempre como una partitura musical, susceptible de ejecuciones
diferentes.\\[0pt]

Por otro lado, una obra escrita está siempre inacabada también en el
sentido de que el mundo que propone o que proyecta se define como el
correlato intencional de una secuencia de frases, del cual queda hacer
un todo para que un tal mundo sea intencionado.\\[0pt]

Según la escuela, a diferencia del objeto que ser percibe en la
observación, el objeto textual no llena intuitivamente esas
expectativas del sujeto, sino que tan sólo induce a
transformarlas. Este proceso de expectativas es el que se produce como
reefectuación de las referencias de la obra. Comprender una obre exige
un viajar a lo largo de ellos, abriéndose a la nuevas expectativas que
tienden a modificar las propias.\\[0pt]

A la vez que hace de la obra literaria una obra, la obra que sólo se
construye en la interacción de su escritura con la acción de
comprenderla. El todo de una obra escrita no puede percibirse de u na
vez. A lo largo del proceso de su lectura se produce un juego de
intercambios entre las expectativas modificadas y los recursos
transformados. No basta con una lectura frase a frase del conjunto
para poder imaginarse un sentido. La obra exige que se le dé una
forma. Y esta es la función propia de la lectura.


\section{La hermenéutica metódica}
\label{sec:orgaa24ea1}
\subsection{La hermenéutica y las humanidades}
\label{sec:orgc40b1b6}
Que aporta o puede aportar a la cultura de nuestra sociedad
tecnificada la hermenéutica como un de los movimientos que han hecho
de la defensa de las humanidades su principal impulso y razón de
ser.\\[0pt]
Las humanidades se han constituido como ciencias de la cultura o de la
herencia de las tradiciones reinterpretadas. Uno de los problemas
principales de este tipo de ciencias es el de la recepción de la
tradición, el de cómo comprender las grandes obras filosóficas,
literarias o artísticas del pasado de tal manera que sea posible
reefectuar su mundo en el presente, imitarlo como modo de apropiarse
su fuerza renovadora. Esta es la cuestión fundamental que llevado la
hermenéutica a constituirse básicamente como un saber de la
interpretación.\\[0pt]


Gadamer es un ejemplo de este tipo de preocupación. Ha polarizado en
buena medida la discusión contemporánea sobre la hermenéutica de las
tradiciones. Esta discusión ha ido delimitando sus posiciones a través
de algunas de las polémicas teóricas actuales. E.g., la disputa del
positivismo en la sociología alemana, la controversia entre
hermenéutica y crítica de las ideologías también en Alemania, la
polémica sobre el estructuralismo en Francia, etc. Lo más destacables
de estas discusiones sea el intento de superar la antinomia entre
ciencias positivas y humanidades, al poderse de manifiesto tanto los
límites epistemológicos de las ciencias formales y empíricas como las
posibilidades del modelo lingüístico para conferir a las ciencias
humanas el estatuto de verdadera ciencia.

\subsection{La oposición entre explicación y comprensión}
\label{sec:org5c9ff79}
Dilthey con la propuesta de las dos cientifidades, la de las ciencias
de la naturaleza —que con su método matemático-experimental y sus
éxitos en la aplicación a la técnica aparecía como el máximo ideal
moderno de conocimiento— frente a las ciencias espirituales como un
saber precario y de segundo orden, que no había encontrado aún el modo
de consolidarse y reafirmarse, la distancia entre las ciencias y las
humanidades parecía insalvable.\\[0pt]
Dilthey pretende proporcionar a las ciencias humanas el mismo carácter
de cientificidad que ya poseían las ciencias de la naturaleza que Kant
había sistematizado.\\[0pt]
El problema era como resolver la paradoja de cómo hacer una ciencia
objetiva y universalmente válida, no sólo del hombre, sino del hombre
individual, con su dinámica psíquica concreta y sin renunciar a toda
la riqueza de sentido que encierran sus aspectos diferenciales.\\[0pt]

El ámbito de la naturaleza es el de los objetos que son susceptibles de
someterse a una explicación de tipo matemático o a los cánones de la
lógica inductiva. La ciencias de la naturaleza puede vertebrarse en
torno a un método matemático o a un método inductivo, ya que sus
fenómenos mantienen relaciones de causa-efecto.\\[0pt]
Pero el ámbito de lo humano es el de lo histórico, donde los
acontecimientos no están gobernados determinísticamente y siguiendo
leyes causales que puedan predecirse, sino que en él interviene la
libertad.\\[0pt]
Luego, las ciencias humanas o históricas no podrán adoptar un método
explicativo, inductivo o matemático como las ciencias de la
naturaleza, sino que habrán de articularse en torno a la comprensión,
a una capacidad de captar una vida ajena que se expresa a través de la
objetivaciones que representa la cultura.\\[0pt]


Es decir, Dilthey dice que es posible un conocimiento de lo individual
objetivo y universalmente válido. Pero debe tomarse la comprensión
como la operación cognoscitiva específica de este tipo de ciencias,
mientras que la explicación sería la operación cognoscitiva propia de
las ciencias de la naturaleza.\\[0pt]
Pero esta diferenciación entre explicación y comprensión encierra
graves problemas y es imposible de fundamentar en la comprensión unas
ciencias humanas con el mismo carácter que tienen las ciencias de la
naturaleza.

\subsection{La verdad de la obra como sentido objetivo de las intenciones de l autor}
\label{sec:orgf15888d}
Dilthey debe recurrir a una metafísica de la vida para dar contenido a
su concepto de comprensión. La vida es la fuerza o la energía
originaria de todo lo que existe, que se despliega creando y
destruyendo los seres, objetivándose en este fluir de formas, de los
acontecimientos o de las interpretaciones.\\[0pt]
Con esta metafísica de base, una obra literaria puede definirse como
un sistema organizado que la cultura ofrece a partir de ese fenómeno
originario que es el de la teleología de la objetivaciones de la
vida.\\[0pt]
La comprensión de esta obra literaria sería la acción, de un lector,
de descifrar los contenidos psíquicos o vitales del autor a partir del
texto en el que sus vivencias se exteriorizan.\\[0pt]

toda interioridad se mostraría en signos externos que pueden ser
percibidos y comprendidos en cuanto signos de una vida ajena o de un
psiquismo extraño y las expresiones de la vida pueden ser tratadas
científicamente mediante el método hermenéutico, como los objetos de
la naturaleza pueden ser tratados científicamente por el método
matemático o el inductivo.\\[0pt]

Por ejemplo, el psicoanálisis. Se autocomprende como una ciencia del
hombre individual y de su dinamismo psíquico concreto, a través de la
comprensión y la interpretación de los signos o los lenguajes en los
que ese psiquismo se objetiva. Esta ciencia se vertebra en torno a un
método hermenéutico que utiliza un conjunto de guías teóricas o
modelos categoriales tipificados.

\subsection{La tipificación metodológica de modelos categoriales y el paso de lo común a lo singular}
\label{sec:orge52f102}

Dilthey proponía construir una topología mediante la identificación
empírica de aspectos comunes en los casos individuales o de conexiones
regulares entre ellos. Al disponer de una categorías generales, se
puede conceptualizar y hacer una ciencia en el ámbito de la vida
individual.  Es decir, D. construye un modelo científico serio, basado
en la idea de intencionalidad de Husserl y del postulado del carácter
idéntico del objeto intencional. Pero en esta propuesta de
clasificación se advierten las grandes limitaciones de su concepción
de las ciencias humanas y de su método hermenéutico.\\[0pt]

Las tipologías no van más allá de un carácter clasificatorio y
propedéutico. Crean un marco de referencia que puede servir de hilo
conductor para identificar una obra literaria situándola por
referencia a una época o a obras literarias.\\[0pt]
La autentica comprensión comenzaría propiamente en el punto mismo en
el que una clasificación de este tipo termina, se trata de pasar del
tipo a lo concreto. El sentido del texto o de la obra literaria es
siempre un sentido singular, no algo genérico o común.\\[0pt]
Este era el punto de partida de la distinción diltheyana entre
ciencias de la naturaleza y ciencias del espíritu, para pasar de lo
común a lo singular.\\[0pt]

Toda obra literaria se presta a una clasificación tipológica por el
horizonte común con otras obras que comparte. Pero cualquier obra
literaria no asume sólo el horizonte de temas y de problemas en el que
surge. Las tipologías no son sólo procedimientos metadológicos
inocentes que preparan el encuentro del lector con la obra,
orientándolo hacia esa región en la que podrá comprenderla. También
puede extraviar por el lado de las abstracciones, de la que la historia
de la literatura está llena. En lugar del núcleo de una obra, se
conoce tan sólo su superficie, su etiqueta exterior.\\[0pt]

Dilthey está muy condicionado por el ideal científico propio de la
Ilustración, construido sobre una relación sujeto-objeto. Así, el lector
comprende la verdad de la obra literaria cuando accede a la
objetividad de las intenciones del autor y de sus vivencias
psicológicas. La verdad de la obra o tiene por qué coincidir con las
intenciones subjetivas de su autor, sino que tiene un significado
autónomo.\\[0pt]

Hoy ya no se comparte esta dicotomía diltheyana entre ciencias de la
naturaleza y ciencias del espíritu, explicación y comprensión.\\[0pt]
Se intenta que comprensión y explicación converjan en un único arco
hermenéutico capaz de integrarlas en un concepción global de la
lectura como reapropiación del sentido. Una lectura de comprensión es
la que permite ir más allá de una semántica superficial para alcanzar
la semántica profunda del texto.


\section{La ontología de la comprensión}
\label{sec:orgaa150da}
\subsection{El carácter hermenéutico de la existencia}
\label{sec:org3e74d91}
Dilthey sitúa el problema de las ciencias del espíritu en el primer
plano de la preocupación epistemológica, señalando el método
hermenéutico, abriendo una línea de reflexión aún abierta. De esta
línea se derivan los planteamientos de Gadamer, que elevan la
hermenéutica al grado de ontología —no tiene sentido como método, como
una simple y extrínseca vía a la verdad, sino sólo si es ya ella misma
el proceso en el que se manifiesta la verdad.\\[0pt]

Frente a Dilthey, Heidegger introduce la novedad de que comprender no
es una actividad humana cualquiera, sino que es la estructura
ontológica fundamental de la existencia con un carácter radicalmente
hermenéutico.\\[0pt]
Heidegger afirma que la sustancia de toda manifestación humana, acción
o comportamiento es comprender.
\begin{quote}
El modo propio de ser del hombre es existir como intérprete o
desvelador del ser.
\end{quote}
\begin{quote}
La comprensión es el modo originario de actualizarse del ser-ahí como
ser-en-el-mundo.
\end{quote}

En la expresión \emph{ser-en-el-mundo}, mundo no significa el mundo como
conjunto de las cosas, ni que el hombre exista en el mundo. Sino que
significa el mundo de los significados fijados por el lenguaje y que
precede siempre a toda comprensión, haciéndola posible, pero
estableciendo al mismo tiempo sus límites.\\[0pt]
Decir que el hombre es ser-en-el-mundo es decir que todo hombre
desarrolla su existencia envuelto en un horizonte de significaciones
lingüísticas del que dependen sus posibilidades de comprensión y de
realización existencial. Es señalar la competencia lingüística como
estructura que hace posible la comprensión al mismo tiempo que la
limita.

\subsection{La complicación del sujeto y objeto en el acto de interpretar}
\label{sec:orgf246c6a}
El giro ontológico heideggeriano es también un giro lingüístico que
permite articular la estructura temporalizada de la historia en el
marco de una ontología hermenéutica.\\[0pt]
Comprender (\emph{ser y tiempo}) es la estructura ontológica fundamental
del hombre que en toda acción se desplaza constantemente de diversas
maneras. Comprender es la sustancia de toda manifestación humana.\\[0pt]
La existencia es concebida como hermenéutica: El modo propio de ser
del hombre es existir como intérprete o desvelador del ser.\\[0pt]

La hermenéutica es una realidad ontológica desvelada por el análisis
fenomenológico de la autocomprensión del ser. Existir es comprenderse
o interpretarse en el mundo.  El comprender indica el movimiento
fundamental de la existencia que la constituye en su finitud y en su
historicidad, y que abraza todo el conjunto de la experiencia del
mundo. No es ambigüedad o exageración sistemática de un aspecto
particular decir que el movimiento de la comprensión es algo universal
y constitutivo. Está en la naturaleza misma de las cosas.\\[0pt]

Desde la fenomenología del comprender, H. manifiesta la imposibilidad
de seguir manteniendo como contrapuestos los conceptos de objetividad
y subjetividad, y la necesidad de admitir la complicación o
pertenencia recíproca de sujeto y objeto. Desde la contraposición es
imposible de saltar de un ámbito a otro (contra Husserl y
Dilthey). Ellos manifiestan el cul-de-sac a que había llegado la
metafísica, y que Heidegger atribuye al olvido del ser como lugar
originario de la mutua pertenencia de sujeto y objeto.\\[0pt]
Para H. el comprender, modo originario de actualizarse el ser-ahí como
ser-en-el-mundo, se actualiza en un círculo hermenéutico que indica
esencialmente la peculiar pertenencia de sujeto y objeto en la
interpretación y que ya no pueden ser llamados de tal modo, pues ambos
términos han nacido y se han desarrollado dentro de una perspectiva
que implica la separación y contraposición del ser que con ellos se
expresa.

\subsection{La estructura de la precomprensión}
\label{sec:orga79d93b}
Lo primero que queda invalidado es la tesis clásica de un saber de la
verdad como conocimiento inmediato, atemporal e inmutable del ser de
las cosas.\\[0pt]
Con la condición ontológica de la existencia como hermenéutica,
H. obliga a reconocer que toda comprensión ha de verse siempre
mediatizada por una interpretación. Frente a la hipótesis filosófica o
científica de un conocimiento de la verdad como conocimiento de la
esencia o de las leyes que rigen el universo, H. señala que lo que se
produce es un círculo hermenéutico entre la estructura de la
precomprensión —característica de la condición ontológica del
ser-en-el-mundo— y la explicación que la comprensión lleva a cabo.\\[0pt]

El sentido de anticipación de H. es análogo el que Gadamer atribuye al
concepto de prejuicio, reivindicando su sentido positivo como
posibilitador de la comprensión: Al interpretar un texto se actualiza
siempre un proyecto.\\[0pt]
Se puede comprender en la medida en que se establece una estrucura de
anticipación, un preconcepto como proyecto de una totalidad que
permita dar sentido a las partes del texto.\\[0pt]
Una obra o un texto sólo exhiben su sentido cuando se leen o
contemplan con cierta expectativa, anticipación de una totalidad en la
que cobran sentido los elementos como parte de la misma. Esta
anticipación manifiesta evidentemente la finitud de nuestro
comprender, pero no sólo su limitación, como indicaba D., sino también
su posibilidad misma.\\[0pt]

Para la teoría tradicional del conocimiento, la verdad de u n hecho o
de un proceso debía ser un sentido objetivo (Dilthey) de naturaleza
que pudiera asumirse como universalmente válido y verdadero para
todos. Lo que garantiza esta objetividad y universalidad de la verdad
era la separación entre sujeto y objeto como polos independientes y
autónomos en la relación de conocimiento.

\subsection{La comprensión como fusión de horizontes}
\label{sec:org18c3469}
Desde de la perspectiva de Heidegger que vincula el hombre con su
mundo lingüístico, no existe nunca un sujeto autónomo y puro, separado
de un mundo de objetos independientes de él.\\[0pt]
Toda relación cogniscitiva exige la actualización de esa estructura de
significados lingüísticos que es la estructura de la precomprensión,
lo que el sujeto conoce no es algo totalmente exterior a él, sino algo
cuyo significado está en la lengua en la que él mismo es o está, y de
la que participa.\\[0pt]

El conocimiento no es más que interpretación o reinterpretación de
significados lingüísticos. Comprender un texto no es un acto del
sujeto en virtud del cuál éste descubre la objetividad de las
intenciones y vivencias del autor. Sino que equivale a apropiarse de
la perspectiva de mundo que contiene y que se da en ella de manera
lingüística. Esto es, será la fusión del horizonte o mundo lingüístico
del lector con el horizonte o mundo lingüístico de la obra.\\[0pt]
Cambia el concepto de verdad: El grado de validez de una lectura como
apropiación no se mide en función de la mayor o menor adecuación entre
lo que el lector comprende y lo que el autor objetivamente quiso
decir, sino en función de hasta qué punto esa lectura constituye para
el lector una experiencia verdadera o experiencia de verdad.

\subsection{La verdad como experiencia verdadera}
\label{sec:org273e86a}
Gadamer elabora el concepto de experiencia verdadera polemizando con
el significado que la experiencia había adquirido en la teoría
empirista anglosejona del conocimiento. Las ciencias empíricas intentan
depurar la experiencia hasta dejarla libre de todo rastro de
historicidad, Gadamer parte de las connotaciones etimológicas de
existencia, el verbo viajar y riesgo. Concibe la experiencia como un
tipo de encuentro con otra realidad, con algo capaz de producir en el
sujeto una verdadera transformación.\\[0pt]

Experiencia de verdad o experiencia verdadera sería un acontecer que
transforma la conciencia del sujeto, la ensancha o la empobrece, la
modifica o invierte etc. Es un acontecer que saca el sujeto fuera de
los límites en los que en ese momento vive, y lo introduce en
realidades distintas, en otros mundos, proyectándole en horizontes más
comprensivos y haciendo, en consecuencia, que cambien posiciones
iniciales.\\[0pt]

Se convierte en el núcleo central de su dialéctica más
íntima. Apropiación de un sentido: Apropiarse de un sentido nuevo a
través de la lectura, significa hacer que algo que era extraño se
convierta en una propiedad mía, en algo propio de mi.\\[0pt]
La apropiación no es posible si el individuo no se abre a ese sentido
y se deja transformar por él.\\[0pt]

\section{La justificación de la conciencia histórica}
\label{sec:orgc287b44}

\subsection{La tradición como estructura de la historia}
\label{sec:org6b29d2d}
Análogamente a la ontología de la comprensión se imponen en la lectura
como recepción de la tradición, como \textbf{articulación de la experiencia}
histórica. En las nociones preheideggerianas de conocimiento
histórico, la objetividad estaba garantizada de éste y su
universalidad era el distanciamiento introducido por la separación
temporal, permitiendo una reconstrucción del pasado a partir de una
lectura filológica-positiva de los documentos transmitidos.\\[0pt]

Del mismo modo que no existe un sujeto y un objeto independientes y
autónomos de la relación de conocimiento, sino un círculo hermenéutico
y una fusión de horizontes, tampoco existe un \textbf{pasado} \textbf{autónomo}
totalmente separado del presente. La historia no es una sucesión
mecánica ed acontecimientos en la que el pasado puede tener un
significado objetivo y determinado de una vez por todas. No hay un
pasado como algo irreversiblemente pasado y al margen ya del presente
y el futuro. Cualquier momento histórico está mediatizado por la
totalidad de los otros momentos. El sentido del presente depende en
buena medida del pasado y del futuro. No hay un presente íntegramente
contemporáneo, como tampoco hay un futuro desligado del presente y del
pasado.\\[0pt]

Entre la historiador y el pasado que trata de conocer, no existe
ningún vacío que separe y sea garantía de la objetividad del
conocimiento, sino la \textbf{cadena} \textbf{de} \textbf{la} \textbf{tradición}: la mediación del
sucederse de la interpretaciones y reinterpretaciones del pasado
dentro de la cual se inserta también la interpretación de este sujeto
historiador. Al proyectar la condición de pertenencia del individuo a
su mundo lingüístico sobre la perspectiva de la temporalidad
histórica, la historia aparece claramente como un proceso de
constitución, disolución y reconstitución continua de los horizontes
lingüísticos que son tradiciones o mundos en los que la comprensión y
la comunicación son posibles.\\[0pt]

Toda obra y acontecimiento permanece en sus efectos, creando el
conjunto de estos el ámbito de la \textbf{tradición envolvente} que hace
posible la comprensión. El objeto de la comprensión no es ningún
significado objetivo, sino un lenguaje en el que el pasado nos habla
como un tú. Se reinterpreta continuamente el pasado proyectando sobre
él un horizonte histórico. Por lo que se puede decir que recibimos la
tradición entrando en un diálogo en el que nos vemos envueltos.

\subsection{La noción de pertenencia}
\label{sec:org8a25814}
La interpretación es la articulación del intérprete y lo comprendido,
significa que antes de todo acto explícito de conocimiento,
cognoscente y conocido, intérprete e interpretado se pertenecen
recíprocamente. Lo conocido está ya dentro del horizonte del
cognoscente, sólo porque el cognoscente está a su vez dentro del mundo
que lo conocido determina.\\[0pt]

Este ámbito de pertenencia recíproca de sujeto y objeto es el
\textbf{ser-ahí}. No lo define como sujeto contrapuesto al mundo, sino como
ser-en-el-mundo articulado como comprensión y preocupación.\\[0pt]
El mundo no significa ni la totalidad de las cosas naturales ni la
comunidad de los hombres, sino la totalidad del horizonte que se
anticipa a la comprensión y como mundo de significados fijados por el
lenguaje.\\[0pt]
El ser-ahí es en el mundo envolvente en un horizonte de significados
que posibilita la comprensión como estructura de anticipación.\\[0pt]
Gadamer, en este planteamiento heideggeriano, examina conjuntamente la
analítica de la comprensión y la temporalidad observando que no es
sólo el \emph{ahí} del \emph{ser-ahí} lo que se confirma en los tres estadios
temporales de la preocupación, sino su poder-ser total.\\[0pt]

No existe un pasado, un presente y un futuro autóctonos y definidos,
sino que todo momento está mediatizado por la totalidad de la
temporalidad.\\[0pt]
Gadamer justifica la conciencia histórica por referencia a la temática
heideggeriana del círculo hermenéutico. Su aportación se puede
concretar en los siguientes puntos:
\begin{itemize}
\item Vínculo entre prejuicio , tradición y autoridad, en un primer
plano fenomenológico.
\item Interpretación ontológica de esta secuencia a partir del concepto
de conciencia de la historia de los efectos.
\item Las consecuencias epistemológicas extraídas, consecuencias
metacríticas, según las cuales una crítica exhaustiva de los
prejuicios o ideologías es imposible en ausencia del punto cero
desde el que ella pudiera ser hecha.
\end{itemize}

\textbf{Pertenencia y círculo hermenéutico}. El círculo hermenéutico no es
 formal, no es subjetivo ni objetivo, sino que describe la comprensión
 como la interpretación del movimiento de la tradición y del
 movimiento del intérprete. La anticipación de sentido que guía
 nuestra comprensión de un texto no es un acto de la subjetividad,
 sino que determina desde la comunidad que nos une con la
 tradición.\\[0pt]
 En nuestra relación con la tradición, esta comunidad está sometida a
 un proceso de continua formación. No es un presupuesto en el que
 estamos siempre, sino que lo instauramos en cuanto comprendemos,
 participamos del acontecer de la tradición y continuamos
 determinándolo así desde nosotros mismos.\\[0pt]
 El círculo de la comprensión no es en este sentido un circulo
 metodológico sino que describe un momento estructural ontológico de
 la comprensión.\\[0pt]
 Sin embargo, el sentido de este círculo que subyace a toda
 comprensión posee una nueva consecuencia hermenéutica llamada
 anticipación de la perfección. También esto es evidente un
 presupuesto formal que guía toda comprensión. Significa que sólo es
 comprensible lo que representa una unidad perfecta de sentido. Se
 hace esta presuposición de la perfección cada vez que se lee un
 texto, y sólo cuando la presuposición misma manifiesta como
 insuficiente cuando el texto no es comprensible, dudamos de la
 transmisión e intentamos adivinar cómo puede remediarse.


\subsection{El modelo del diálogo}
\label{sec:org752a4e3}
El hombre siempre está sujeto al círculo de la comprensión. Nuestra
comprensión está más condicionada por nuestros prejuicios que por
nuestros juicios. Es ilusorio querer eliminar los prejuicios. no
significa aceptar cualquier prejuicio como hecho ineluctable, sino tan
sólo reconocer que el hilo conductor del discurso humano se desarrolla
siempre desde la tradición, puesta en juego continuamente en la medida
en que el discurso se desarrolla siempre de modo abierto y nuevo hacia
el futuro.\\[0pt]

La tematización de la experiencia del juego no es aquí casual. El
juego constituye uno de los modelos más significativos y
característicos de comprensión del proceso hermenéutico. Es un proceso
que implica en él a los jugadores más allá de los propósitos y
provisiones de ellos. Buen jugador es el que no se acoge a esquemas
preconcebidos, sino que se entrega al juego mismo.\\[0pt]
Análogamente, la comprensión no es una visión que se destaca
intemporalmente de una realidad concluida y perfecta, que deja intacta
tanto la realidad comprendida como el horizonte del que comprende. Al
contrario, comprender es siempre un acontecer que modifica en su
transcurso tanto al que interroga como a lo interrogado. El diálogo es
propiamente la forma más esencial y auténtica de comprensión.\\[0pt]

Gadame ha estudiado el significado hermenéutico del diálogo de Platón
y de la ética de Aristóteles. El diálogo no es para Platón una forma
más o menos eficaz de persuasión o de enseñanza, sino el momento mismo
de descubrimiento de la verdad y del esfuerzo de entenderse con
otros. No es posible entenderse con los demás sobre algo si no es
alcanzando juntos una verdad que se manifiesta en un movimiento
dialógico y dialéctico. Gadamer considera sugestivo y actual el
problema metodológico de la ética de Aristóteles. En ella es saber es
visto como un constante esfuerzo por poner a prueba perspectivas cuyo
significado y valor no puede demostrarse de modo universal y
necesario, sino que emerge siempre sólo en virtud de una aplicación
condicionante, no sólo del resultado de la acción, sino también del
carácter del hombre.\\[0pt]

Gadamer recurre al lugar en el que se actualiza el lenguaje. Es en el
discurso con los otros y con nosotros mismos donde acontece la fusión
de horizontes en la que consiste comprender, y que tiene un carácter
de mayor alcance que el de un proceso puramente psicológico o
gnoseológico. A partir de Platón, se ha cometido el error de plantear
el problema del lenguaje en términos de la exactitud y no de verdad,
abriendo el camino a sucesivos proyectos de construcción de lenguajes
artificiales. El defecto común de todo esto es el entender el lenguaje
como producto de un pensamiento reflexionante, incluso considerarlo
como constitutivo de nuestra relación con el mundo. El mundo, en el
lenguaje, no se presenta como objeto, sino que revela su sentido en un
proceso que es hermenéutico e histórico. El proceso hermenéutico es el
único y constante correctivo con el que el pensamiento se sustrae al
dominio de la estructura lingüística. La hermenéutica tiene sentido
sólo si es ya ella misma el proceso en el que se manifiesta la verdad
y no una simple y extrínseca vía de acceso a ella.

\subsection{La polémica reivindicación del prejuicio y de la autoridad}
\label{sec:org73239b6}
Para Gadamer, del estudio de los diálogos de Platón y la ética de
Aristóteles, todo interpretar tiene carácter dialógico. Es decir, toda
comprensión y toda interpretación se producirán de acuerdo con la
dialéctica de la pregunta y la respuesta.\\[0pt]
concluía que el diálogo no es tanto una forma eficaz de persuasión o
de enseñanza, sino el momento mismo de descubrimiento de la verdad y
del esfuerzo por entendernos con los demás.\\[0pt]

\textbf{Resumen de la hermenéutico ontológica}. Ningún individuo puede
abstraerse del mundo lingüístico en el que se ha formado. Todo lo que
conocemos está mediatizado por el juego interpretativo de las
posibilidades del lenguaje.\\[0pt]
 Esa naturaleza lingüística de la experiencia afecta también al
concepto de la tradición, poniéndose de relieve el carácter mediador
del lenguaje desde la perspectiva de la temporalidad histórica.\\[0pt]

Los textos, documentos, las cosas dichas y hechas en el pasado son
mundos que podemos recibir y comprender como horizontes de experiencia
posible para el lector o el intérprete de hoy.\\[0pt]
 Para Gadamer la recepción de la tradición tiene estructura de
diálogo, lo transmitido nos plantearía preguntas y se daría él a sí
mismo la respuesta. El conjunto de respuestas con la nuevas preguntas
y las nuevas interpretaciones a que dan lugar constituyen la
\emph{historia} \emph{efectual} como diálogo en el que nos vemos envueltos.


En este contexto podemos comprender con precisión la reivindicación de
Gadamer de los prejuicios, autoridad y tradición:

Gadamer \textbf{reivindica el prejuicio} como estructura de la
precomprensión. Filosofía del juicio en la que el prejuicio aparece
como lo opuesto a la razón. No es tanto un polo opuesto a una razón
sin presupuestos, sino un componente inevitable de toda comprensión,
ligado a la condición histórica del hombre.\\[0pt]

El prejuicio tiene un sentido positivo como precomprensión y como
posibilitador de la comprensión.\\[0pt]
Por ejemplo, al interpretar un texto se actualiza siempre una
proyección de sentido. Se puede comprender en la medida que se dispone
de una estructura de anticipación, un preconcepto como proyecto de una
totalidad que permite dar sentido a las partes del texto. Esta
anticipación manifiesta la finitud de nuestro comprender, al mismo
tiempo que lo hace posible. Esto conduce a la reivindicación de la
autoridad.\\[0pt]

Gadamer \textbf{reivindica la autoridad} como fundamento del prejuicio. Es
falso que haya solo prejuicios infundados. El prejuicio contra el
prejuicio procede de una prevención contra la autoridad cuando se la
identifica con dominio y violencia. Pero la autoridad no tiene por qué
tener necesariamente como contrapartida la obediencia ciega. Sólo
cuando es tiránica e impersonal, la autoridad se funda en acto de
sumisión y de obediencia de la razón.\\[0pt]

La respuesta a la autoridad puede ser también el reconocimiento,
ligado a la idea de que lo que dice la autoridad competente no
siempre es arbitrario e irracional, sino que puede ser aceptado como
criterio superior de quien sabe más, de un especialista o de un
maestro.  Esto conduce a que la tradición tiene la autoridad, la
tercera reivindicación.

Gadamer \textbf{reivindica la tradición}. Para Gadamer lo consagrado por la
tradición transmitida tiene un peso de autoridad que determina nuestro
ser histórico y nuestra formación y, nuestras posibilidades mismas de
existencia.\\[0pt]

Gadamer no opone tradición y razón. Los argumentos de autoridad, o el
recurso a la tradición transmitida, contribuyen a una interpretación
libre. Esta autoridad de la tradición puede pasar por la criba de la
duda y la crítica.\\[0pt]


Hantes de ser criticada, una tradición exige ser recibida, asumida y
transmitida. La dependencia de la interpretación se pone de relieve en
la definición misma de la comprensión como fusión de horizontes. El
horizonte del presente lo compone nuestro mundo lingüístico o
ideológico, transmitido por la tradición, que, como estructura de la
precomprensión, es el horizonte a partir del cual la comprensión es
posible.

\subsection{El concepto de historia de los efectos}
\label{sec:orgca82937}
El punto de vista de Gadamer es el de las ciencias de l espíritu, el
de las humanidades como ciencias de la cultura o de la herencia de las
tradiciones reinterpretadas. Sus momento críticos están siempre por
debajo de del reconocimiento de la autoridad de la tradición. La
instancia crítica se desarrolla como un momento subordinado a la
conciencia de dependencia respecto de los significados de la
precomprensión, que siempre precede y envuelve a toda interpretación.\\[0pt]

El aspecto hermenéutico de la conciencia histórica. Todas las
corrientes históricas que parten de Schleiermacher y se prolongan a
través de Ranken hasta Dithley, cometen el error de ver la historia
desde un modelo subjetivista. Las épocas individuales y el mismo curso
de la historia son consideradas como una totalidad orgánica que se
trata de entender desde las partes al todo, y de ésta a las partes.\\[0pt]

Se ha esforzado para comprender la individualidad y la unicidad de los
fenómenos históricos desde su interior. Se tratado de alcanzar los
aspectos más profundos y las intenciones escondidas para comprender
una obra mejor que su propio autor.\\[0pt]
Dilthey prevalece el interés por la reconstrucción del pasado como
interés por el conocimiento de la vida en sus expresiones, planteándose
el problema de cómo superar el hiato entre el carácter expresivo y el
carácter significativo de la vida misma, entre el significado pensado
psicológicamente y su expresión histórica. Faltaba, pues, toda
posibilidad de una integración de los significados en la conciencia
hermenéutica, de una explicitación viva del sentido del proceso
histórico en su mediación concreta con el presente y con el futuro.\\[0pt]

Aquí Gadamer se vuelve contra Hegel, desde el análisis heideggeriano de
la finitud de la conciencia y comprensión histórica.\\[0pt]
De Hegel, acepta el buscar el sentido dialéctico de la conciencia
histórica como movimiento en la que se despliega el sentido mismo de
la vida. No se acepta que la búsqueda y explicitación de este sentido,
en la tensión pregunta-respuesta, pueda o deba resolverse en una
pretendida construcción especulativa y total de la historia.\\[0pt]
Para mostrar la estructura hermenéutica del realizarse de la historia
en su facticidad, amplía el análisis fenomenológico de la comprensión
y la temporalidad de Heidegger, desarrolla la \textbf{idea de mediación} con
relevancia ontológica en el nivel del lenguaje como esfera de la
pertenencia y de la tradición.\\[0pt]

Entre el historiador y el pasado que trata de interpretar no existe
ningún vacío, sino la cadena de la tradición o mediación de las
transmisión de la interpretaciones y reinterpretaciones dentro de la
cual se inserta la del propio historiador. Toda obra o acontecimiento
permanece en sus efectos, creando el conjunto de estos efectos el
ámbito de la tradición envolvente que hace posible la comprensión
histórica.\\[0pt]
El objeto de la experiencia hermenéutica no es ningún significado
objetivado, sino un lenguaje en el que el pasado nos habla como un tú,
un diálogo en el que nos vemos envueltos. La apertura que implica la
experiencia hermenéutica tiene estructura de una pregunta. Cuando algo
de adapta a la opinión ya existente surge la pregunta, sin que sea
posible alejarla manteniéndose en el modo habitual.

Todo interpretar y pensar tienen un carácter dialéctico-dialógico. Un
texto sólo puede ser interpretado, comprendido, como respuesta a una
pregunta. Todo pensar tiene la estructura de la pregunta-respuesta.\\[0pt]
La historia es como un compañero en un proceso de comunicación igual
que el tú lo es para el yo. Tanto la historia como su intérprete están
envueltos en un mismo proceso o diálogo que es la tradición, ámbito de
pertenencia de historia e intérprete.\\[0pt]

La tradición como el juego del diálogo, tiene una dinámica interna que
nos-es. todo comprender debe pensarse menos como una acción de la
subjetividad que como el insertarse, tener conciencia de estarlo, en
el acontecer de la tradición. El sentido de la experiencia
hermenéutica es tomar conciencia de nuestras determinaciones
históricas de que no existe comprensión libre de todo prejuicio y que
formamos parte de un coloquio ilimitado en el que no existe la última
palabra, no existe ningún saber absoluto definitivo. Todo saber impone
nuevas preguntas que exigen nuevas experiencias que manifiestan la
limitación de nuestro saber anterior. La tradición en la que estamos
envueltos es un diálogo siempre abierto. Considerarlo cerrado es una
pretensión ilusoria.

\section{La hermenéutica postgadameriana}
\label{sec:orgf59d1d8}

\subsection{La reefectuación de las referencias potenciales de la obra}
\label{sec:org8799e12}
\end{document}