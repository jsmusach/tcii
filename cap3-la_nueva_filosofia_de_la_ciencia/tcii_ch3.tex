% Created 2023-05-28 dg. 22:35
% Intended LaTeX compiler: pdflatex
\documentclass[a4paper, 11pt, twocolumn, spanish]{article}
\usepackage[utf8]{inputenc}
\usepackage[T1]{fontenc}
\usepackage{graphicx}
\usepackage{longtable}
\usepackage{wrapfig}
\usepackage{rotating}
\usepackage[normalem]{ulem}
\usepackage{amsmath}
\usepackage{amssymb}
\usepackage{capt-of}
\usepackage{hyperref}
\usepackage[T1]{fontenc}
\usepackage[margin=.75in]{geometry}
\setlength\parindent{0pt}
\author{Jordi Serra}
\date{\today}
\title{Apuntes de Teoría del Conocimiento ii\\\medskip
\large III La nueva filosofía de la ciencia}
\hypersetup{
 pdfauthor={Jordi Serra},
 pdftitle={Apuntes de Teoría del Conocimiento ii},
 pdfkeywords={},
 pdfsubject={},
 pdfcreator={Emacs 27.1 (Org mode 9.6.2)}, 
 pdflang={English}}
\begin{document}

\maketitle
\tableofcontents


\section{Introducción}
\label{sec:org6112d64}
\subsection{El final del modelo clásico de cientificidad}
\label{sec:orgc4fd2cf}
En las últimas décadas del s. XX hubo un profundo cambio en el ideal
clásico de cientificidad. Se ha perdido el sentido lineal y
acumulativo del devenir científico y se ha introducido la acción
humana en la historia de la ciencia. La ciencia deja de ser un
lenguaje perfecto. El mito de la presunta superioridad objetiva de la
ciencia se viene abajo , al quedar de manifiesto la inserción de la
elección en la práctica científica cotidiana.\\[0pt]

El ideal clásico de cientificidad estaba representado por el
neopositivismo lógico. En contraposición a éste, podemos simplificar
en tres las nociones importantes en la nueva teoría de la ciencia.
\begin{enumerate}
\item Los hechos científicos son relativos al sistema conceptual
utilizado para poner en evidencia su articulación, susceptibles de revisión a medida que se vaya modificando el sistema.
\item La evaluación de las hipótesis científicas, particularmente en
las teorías que revisten un alto grado de complejidad, implica un
tipo de apreciación que supone un largo entrenamiento, no siendo
reductible a reglas formales explícitas de inferencia. Una
reconstrucción puramente lógica de la confirmación científica
debe necesariamente resultar incompleta.
\item Teoría rivales pueden ser inconmensurables entre sí. Puede que
sus sistemas conceptuales respectivos no se corresponden
suficientemente como para hacer posible una comparación
directa. O también los valores metodológicos que se incorporan a
las teorías no son los mismos. En consecuencia, no se dispone de
una metodología que haga posible una comparación entre ellas.
\end{enumerate}

\subsection{Teoría de la ciencia y práctica científica}
\label{sec:orgf6e8b28}
Hasta la fecha, se sostenía que las proposiciones verdaderamente
científicas debían fundarse bien en la intuición (Descartes), bien en
relaciones de observación (Locke), bien en principios sintéticos a
priori (Kant).\\[0pt]

Durante el s. XIX, el auge de las ciencias atrae la atención hacia el
estudio de sus métodos de trabajo y a su historia misma, desviándose
de este punto de vista. La aparición del círculo de Viena, enfatizando
el conocimiento positivo, basado únicamente en la observación sensible
de los hechos y en la lógica formal como instrumento de análisis
(según el modelo de Russell), vuelve a restablecer la vigencia de las
tesis clásicas en teoría de la ciencia.\\[0pt]

Para los positivistas lógicos las proposiciones básicas, en las que se
expresan las observaciones de la ciencia experimental, pueden servir
de base a generalizaciones cada vez más amplias.  Intentan formular la
relación de confirmación entre una hipótesis y una proposición que
exprese una evidencia. Esta corriente surge como una reacción frente a
la retórica y las abstracciones de cierta filosofía basada en el rigor
matemático de los lógicos.\\[0pt]
Se tiene una clara imagen del estado de la ciencia en cada etapa. Se
empieza a partir de los hechos establecidos y se hace uso de los modos
de inferencia para llegar al nivel de las leyes y de las teorías. Se
puede tener una conciencia exacta tanto de significación como de
verdad, sobre el estado en que se está y sobre las relaciones de
proposiciones de cada estado y los hechos a partir de los cuales se ha
razonado.\\[0pt]

Este método no se puede generalizar a cualquier proposición que se
quiera saber si es verdadera o no. El principal fallo de esta
concepción de la ciencia fue no haber prestado atención a la historia
real de las ciencias y a los cambios conceptuales. A partir de Popper,
se trata de construir la teoría de la ciencia sobre una base más o
menos vinculada a la práctica misma de la ciencia, mejor que a partir
de una teoría general del conocimiento como había sido el caso en la
tradición epistemológica clásica.

\section{Principio de verificación y unificación del lenguaje científico. El círculo de Viena}
\label{sec:orgbd647b1}
\subsection{La división del discurso}
\label{sec:orgad0af3f}
\subsubsection{El isomorfismo lenguaje-realidad según el \emph{Tractatus}}
\label{sec:org12e1043}
El \textbf{\emph{tractatus}} está escrito en forma de aforismos numerales según el
sistema decimal y contiene siete proposiciones fundamentales.\\[0pt]
Las dos primeras \emph{El mundo es todo lo que acaece} y \emph{lo que acaece, el
hecho, es la existencia de los hechos atómicos} se refieren al mundo y
a la realidad.\\[0pt]
Mientras que las cuatro siguientes son un desarrollo de la lógica y de
teoría del lenguaje.\\[0pt]
La última proposición, la enigmática frase De lo que no se puede
hablar mejor es callarse, cierra el libro marcando el límite de lo que
se puede pensar y decir (la proposición).\\[0pt]
Aunque la mayor parte del \emph{tractatus} habla de la lógica y lenguaje (de
la proposición), los párrafos iniciales tratan del mundo y de la
visión metafísica del mundo en términos del atomismo lógico
(Russell).\\[0pt]
El mundo es la totalidad de los hechos y el lenguaje es la totalidad
de las proposiciones. Ambos comparten una estructura lógica común y
Wittgenstein relaciona realidad, lógica y lenguaje mediante \textbf{tres
conceptos fundamentales}: hecho atómico, figura lógica y
proposición.\\[0pt]

\begin{enumerate}
\item Hechos atómicos. El constituyente último del mundo son los
objetos o cosas, las entidades que percibimos con los
sentidos. Los objetos son simples y forman parte de los hechos
atómicos. El hecho atómico es la combinación o relación de
objetos o cosas. Éstos son la substancia de que está hecho el
mundo, su constituyente básico. Pero de las cosas del mundo sólo
podemos conocer lo que acaece (hacerse realidad). Esto es, los
hechos atómicos o simples, los hechos compuestos de simples, o
simplemente hechos, cuyo conjunto constituye la realidad.
\item La figura lógica. Paralelamente, el lenguaje opone a las cosas
del mundo nombres. A los hechos atómicos, proposiciones simples;
y a los hechos complejos proposiciones compuestas. El lenguaje
puede representar la realidad del mundo. Cuando por medio de
proposiciones describe hechos, la estructura de las proposiciones
en relación a la de los hechos, y viceversa, se preserva. Este
isomorfismo entre el lenguaje y la realidad, es posible gracias a
la participación de ambas instancias en una misma estructura
común.
\item La proposición o el signo con que expresamos el pensamiento
representa un estado de cosas (un hecho atómico). Si este es
real, la proposición es verdadera. El conjunto de todas ellas
describe el mundo. Sólo las proposiciones, y no los nombres, son
significativas y muestran la lógica de la realidad. Las
proposiciones siempre tienen sentido, aunque sean falsas, porque
siempre describen lo que acaece en el mundo. Y sólo pueden tener
sentido cuando describen lo que acaece en el mundo.
\end{enumerate}

Las proposiciones que no describen hechos carecen de sentido. Éstas
son de dos clases:
\begin{enumerate}
\item Comprende las tautologías o enunciados necesariamente verdaderos,
que nada dicen respecto del mundo. (o sus negaciones, las
contradicciones.)
\item Aquellas proposiciones que no comparten figura lógica con la
realidad que pretenden representar.
\end{enumerate}

Esto último sucede de dos maneras. Porque se la da a un signo un
sentido falso, una mala orientación, construyendo enunciados que
contienen signos carentes de significado, como sucede con las
proposiciones mal construidas o con las de carácter metafísico. O
porque apuntan a objetos que quedan fuera del mundo, trascienden el
mundo, queriendo expresar lo inexpresable, como pasa con las
proposiciones sobre ética y aquellas que quieren esclarecer el sentido
del mundo, las proposiciones metafísicas.

Así pues, sólo las proposiciones delas ciencias empíricas tienen
sentido. La lógica consta únicamente de tautologías, y toda
proposición sobre ética o metafísica es una proposición carente de
sentido. El análisis filosófico ayuda a esclarecer el sentido de las
proposiciones del lenguaje ordinario; las del lenguaje filosófico las
declara carentes de sentido.

\subsubsection{La ciencia habla de la realidad, la filosofía habla del lenguaje}
\label{sec:org30acba5}
El principal criterio de diferenciación consiste en afirmar que sólo
la ciencia habla con legitimidad y sentido acerca dela realidad
extra-lingüística, mientras que la filosofía sólo tiene la tarea de
esclarecer, unificar, sistematizar y analizar el lenguaje
científico. El trabajo científico produce la representación
lingüística o simbólica adecuada a la realidad. La filosofía es una
actividad de segundo orden, metalingüística, que tiene por objeto el
lenguaje y el discurso de las ciencias. Sólo excluyendo la ambición
ontológica o metafísica de la filosofía tradicional, puede asignares
la labor a la filosofía de analizar con ayuda de la lógica formal el
lenguaje científico de la ciencia.\\[0pt]

Los \textbf{enunciados referenciales} son aquellos que se refieren a objetos
que se pueden identificar. Poseen contenido y sentido. Si los objetos
referidos no son lingüísticos los enunciados son \textbf{realistas u
objetivos}. Si sí lo son, entonces se llaman \textbf{metalingüísticos}.\\[0pt]
Uno de los objetivos principales del análisis lógico del lenguaje es
descubrir a los enunciados metalingüísticos con apariencia realista y
reformularlos correctamente a fin de que pongan claramente de
manifiesto que se refieren a palabras y no a cosas.

En el neopotivismo, la wittgensteiniana concepción pictórica del
lenguaje se expresa mediante las siguientes distinciones.
\begin{enumerate}
\item Proposiciones con sentido y pseudoproposiciones. Sólo tiene
sentido un enunciado que pueda ser calificado de verdadero o
falso. La verificabilidad constituye el criterio del
sentido. Cuando una palabra posee un significado, se dice que
designa un concepto, mientras que si esta significación es sólo
aparente y en realidad no a posee, entonces es un
pseudoconcepto.\\[0pt]
Las condiciones para que una proposición tenga significado son:
\begin{itemize}
\item Que las notas empíricas de los términos incluidos en la
proposición sean conocidas.
\item Que haya sido estipulado de qué proposiciones protocolarias es
derivable la proposición a examinar.
\item Que las condiciones de verdad para esa proposición hayan sido
establecidas.
\item Que dispongamos de un método de verificación.
\end{itemize}

\item Enunciados analíticos y enunciados empíricos. Para los
neopositivistas, el principio de verificación se aplica a un
enunciado sustancialmente de dos formas.
\begin{itemize}
\item Determinando su coherencia lógica interna mediante el análisis
de la posibilidad de reducirlo a enunciados más elementales.\\[0pt]
Se actúa así con los lenguajes formales con los que la verdad
se decide sin recurrir a la experiencia porque es determinable
a priori.\\[0pt]

\item Recurriendo a la experiencia, que afecta a enunciados
directamente referidos a la realidad y a los que integran las
ciencias de la naturaleza.\\[0pt]
Son enunciados cuya verificación requiere su confrontación con
los hechos extralingüísticos a los que remiten.\\[0pt]
Es una verificación empírica y a posteriori.
\end{itemize}

Los del primer tipo no aportan ninguna información verdaderamente
que no estuviera ya comprendida en la sintaxis y la semántica del
lenguaje utilizado. Están desprovista de sentido, como las
tautologías en la lógica.\\[0pt]
En cambio, los enunciados empíricos aportan informaciones sobre la
realidad fáctica y extralingüística.\\[0pt]
En cualquier caso, todo enunciado pertenece a una categoría o a la
otra.\\[0pt]

Una distinción de esta misma distinción es la que se establece
entre leyes empíricas y leyes teóricas:
\begin{enumerate}
\item Leyes empíricas son las que pueden ser confirmadas
directamente mediante observaciones empíricas. Son leyes
acerca de hechos observables. Contienen términos que designan
hechos observables por los sentidos.\\[0pt]
Se las obtiene mediante la generalización de los resultados
de observaciones.\\[0pt]
No sólo incluyen leyes cualitativamente simples —todos los
cuervos son negros—, sino también leyes cuantitativas que
surgen de observaciones simples.\\[0pt]
Se las usa para explicar hechos observados y predecir sucesos
futuros observables.
\item Leyes teóricas o hipotéticas son las que contienen términos
que no se refieren a hechos observables. Son leyes acerca de
entidades como moléculas. átomos, electrones, etc. que no
pueden ser medidos de manera simple y directa.
\end{enumerate}

\item Enunciados que expresan juicios éticos o estéticos. Al considerar
el lenguaje básicamente informativo se determinan como ilegítimos
y sinsentido los enunciados que expresan sentimientos o
valores.\\[0pt]
Los preceptos morales sólo expresan emociones positivas o
negativas respecto de acciones y de situaciones
descriptibles. Estos usos lingüísticos no presentan interés para
la ciencia no para la filosofía.\\[0pt]

En esta posición radical en favor del discurso científico como
idealmente objetivo y universal y este desprecio por las
expresiones de la subjetividad explican la ausencia de compromiso
filosófico y ético-político de esta filosofía con la
sociedad. Sólo el desarrollo de la ciencia y del espíritu
científico y analítico, lógico y objetivo, cabe esperar un
progreso real para la humanidad. Mientras ese progreso se
realiza, nadie está obligado a tomar posición en el terreno moral
o político. Cualquier discurso filosófico o científico, que
presenta valores o normas propiamente dichas como si se tratara
de hechos objetivos o de realidades susceptibles de ser descritas
de manera verdadera o falsa y de ser conocidas, cae en la falacia
naturalista. Este sofisma consiste en la confusión entre hechos y
valores, entre lo que es y lo que debe ser.
\end{enumerate}


\subsection{El lenguaje unificado de la ciencia}
\label{sec:orge5f4469}
El ideal de una ciencia unificada es característico del pensamiento
moderno desde Descartes, que se plantean construir una matemática
universal. El neopositivismo retoma este objetivo y lo reformula en el
plano del lenguaje. todas las ciencias han de tener en común la
utilización de un lenguaje. Se espera la superación de la diversidad
de las ciencias si se descubre o se construye ese lenguaje al que se
puedan reconducir los diferentes lenguajes científicos. La unificación
de las ciencias requiere la construcción de un lenguaje universal y
unitario de la ciencia.\\[0pt]

Carnap en \emph{La estructura lógica del mundo} (1928) defiende como
condición básica de ese lenguaje que sus enunciados básicos se
refieran a sensaciones y experiencias sensoriales y no a objetos
físicos, que sólo son construcciones hipotéticas a partir de las
sensaciones. Las cosas son elaboraciones lógicas que realizamos sobre
la base de nuestro contenidos sensoriales. Lo propio de las
proposiciones científicas debe ser la posibilidad de que puedan quedar
reducidas a un lenguaje compuesto de símbolos de los contenidos
sensoriales.\\[0pt]

El problema es que garantiza la objetividad de la ciencia ni su
exigencia universal. Se basaría en experiencias subjetivas de cada
individuo, pudiendo diferir. Los enunciados básicos de este lenguaje
no serían comunes más que en apariencia.\\[0pt]
Neurah dio un giro distinto con su planteamiento fisicalista. El mundo
está constituido por objetos que existen con independencia de mi
experiencia. A esos objetos se refieren directamente los términos y
enunciados del lenguaje científico. Construir un lenguaje básico para
la ciencia representa descubrir los enunciados elementales y los
objetos o hechos elementales constitutivos de la realidad. Luego se
podrá traducir cualquier enunciado a ese lenguaje básico relativo a
estados  procesos del mundo físico.\\[0pt]

En el seno del fisicalismo, durante el s. XX se cuestiona la
naturaleza referencial o realista de su propio lenguaje. Por otra
parte, fracasa también en su empeño por reducir fisicalistamente el
lenguaje de las ciencias humanas, como la psicología. El programa
neopositivista de unificación de las ciencias por la unificación de
sus lenguajes se queda en mera aspiración.

\subsection{La crítica de la metafísica}
\label{sec:org114de7c}
El análisis lógico-positivo del lenguaje responsabiliza del sentido o
bien a errores sintácticos en los enunciados, o bien a abusos
semánticos. Es decir, son enunciados sin sentido aquellos en los que
se combinan palabras que pertenecen a categorías diversas, pero que
considerados por separado o utilizados correctamente, tienen
significado. También son sin sentido los enunciados que incluyen
palabras sin una referencia determinable, como los enunciados
metafísicos.\\[0pt]

El neopositivismo reduce la metafísica a un conjunto e errores
sintácticos y de abusos semánticos del lenguaje. Desde la perspectiva
lógica, los enunciados metafísicos o bien parecen despojados de
sentido, o bien sin el sentido o el alcance que se les quiere
atribuir.
\begin{enumerate}
\item La primera diferencia estaría en la ambivalencia misma de la
palabra ser. A veces se utiliza como cópula que antecede a y se
relaciona con un predicado —yo soy el autor de este libro—,
mientras que en otras designa existencia —yo soy. Este error
resulta agravado por el hecho de que los metafísicos carecen de
una idea clara de esta ambivalencia.
\item El segundo error está en la forma que adquiere el verbo en su
segunda significación, la de existencia. Esta forma verbal
muestra ficticiamente un predicado no existente. La existencia no
es una propiedad. A este respecto sólo la lógica moderna sería
totalmente consecuente al introducir el signo de existencia en
una fórmula sintáctica tal que no puede ser referido como un
predicado a signos de objeto, sino sólo a un predicado.
\end{enumerate}

Es decir, desde la antigüedad, la mayor parte de los metafísicos se
habrían dejado seducir por la forma verbal de la palabra ser, formando
pseudopreposiciones como \emph{yo soy}, \emph{Dios es}, etc. Pero lo
característico de la actitud física sería un determinado modo de
confusión de las palabras y las cosas. La labor del neopositivista
frente a este lenguaje será reconducir sus proposiciones a
descripciones o recomendaciones concernientes al lenguaje y su uso.
La filosofía deja de ser metafísica para reducirse a metalingüística y
transformarse en crítica y análisis lógicos del lenguaje. Este giro
lingüístico debería permitir también a los filósofos entenderse,
puesto que compartirían un punto de vista común (el metalingüístico) y
una referencia común (el lenguaje).

\section{Investigación científica y desarrollo del conocimiento: Popper}
\label{sec:orgcca902f}
\subsection{La crítica a la epistemología inductiva}
\label{sec:orgc7d0be8}
La tradición empirista tiene como método fundamental de la ciencia
moderna la inducción para establecer conceptos y teorías. Popper
rechaza los tres principios básicos del empirismo relativos al
conocimiento:
\begin{enumerate}
\item Discute el \textbf{método inductivo} como método fundamental de desarrollo
de la ciencia y e la ciencia, la suposición de la capacidad mente
humana para alcanzar verdades universales válidas que se
generalizan a partir de la observación repetida de casos
particulares.\\[0pt]
Para Popper no se pueden verificar proposiciones o teorías de
carácter universal a partir de proposiciones particulares que no
las contradigan —la observación de muchos cuervos negros no
implica que todos los cuervos sean negros.
\item No cree que la mente, en el proceso del conocimiento, adopte
solamente una \textbf{actitud pasiva y receptiva}. La mente no es una hoja
de papel en blanco en la que la percepción de entrada a las
sensaciones que se abstraerían para formar conceptos.
\item Rechaza el \textbf{principio de verificación} como criterio de sentido
desde el cual, los pensadores empiristas y positivistas han
realizado la crítica a la metafísica que la reduce al
sinsentido.\\[0pt]
Las teorías no son verificable empíricamente.\\[0pt]
Es necesario un nuevo criterio que separe la ciencia y la
metafísica, \textbf{demarcación}, establezca las distintas características de uno y
otro saber.
\end{enumerate}

\subsection{El problema de la demarcación}
\label{sec:org809ba3a}
Este surge al intentar distinguir las diferencias entre la teoría de
Einstein y las teorías del marxismo, el psicoanálisis y la psicología
del individuo. Popper se pregunta porqué estas últimas son tan
diferentes de las de Einstein, Newton etc. Aún presentándose como
ciencias de hechos, encuentra elementos en común con los mitos
primitivos que con la ciencia, se asemejaban más a la astrología que a
la astronomía.\\[0pt]

Popper sostiene que la diferencia es que las teorías de Einstein \&
cia. podían ser refutadas por un acontecimiento que se predijera
partiendo de ellas, mientras que las otras resultaban siempre
confirmadas por cualquier experiencia posible.\\[0pt]
Sólo las primeras eran auténticas teorías científicas en la medida en
que sólo de ellas es posible decir que son comprobables por la
experiencia.\\[0pt]
Pero no desprecia el valor de las segundas, muy importantes en el
devenir de la historia del hombre.\\[0pt]


La diferencia fundamental entre unas i otras es que las primera no
asumen que el carácter científico de sus teorías viene dado por la
acumulación de hechos que refuerzan el sentido de la teoría
descartando los que no lo hicieran, sino mediante la determinación de
una experiencia decisiva, una prueba en forma de predicción deducida
de la teoría —pero no deducido de la teoría rival, y que pudiera ser o
no precisamente confirmada por la observación.\\[0pt]

Este punto de partida es el que permite manifestar y ahondar en la
diferencia entre su criterio y el criterio positivista de
significado.\\[0pt]
Con el criterio de refutabilidad, como criterio de demarcación, traza
una línea divisoria entre los enunciados o sistemas de enunciados
empíricos y todos los otros enunciados. Así, no se trata de un
problema de sentido o de significación, ni de verdad o aceptabilidad,
sino de demarcación.\\[0pt]

Por al mismo tiempo, el criterio de demarcación envuelve un problema
ontológico de envergadura. Popper se centra en la cuestión del
desarrollo de la ciencia, siendo su referencia polémica las nuevas
ciencias sociales y humanas.

\subsection{El aspecto metodológico}
\label{sec:org3a0fdba}
Popper entiende la filosofía como teoría del conocimiento científico o
epistemológico. El problema fundamental de la epistemología no es el
de la estructura de la ciencia, sino el del desarrollo de la ciencia
—siempre será un asunto a resolver el de a qué cosa hemos de llamar
una ciencia. La epistemología no debe ser propiamente una lógica de la
ciencia, sino un lógica de la investigación científica, una lógica del
procedimiento según el cual se desarrolla el conocimiento
científico. La epistemología se reduce a metodología de la ciencia.\\[0pt]
Popper lo compara con el ajedrez, de la misma manera que que es
posible definir el ajedrez mediante sus reglas, la ciencia empírica
puede definirse por medio de sus reglas metodológicas.\\[0pt]


Esta metodología e la ciencia deberá clarificar el concepto mismo de
ciencia, aunque su problema fundamental será el del desarrollo del
conocimiento. No se trata de precisar un único método, sino que es
apropiado usar tantos métodos necesarios para avanzar en la resolución
del problema dado.\\[0pt]
De este modo, ni el análisis del lenguaje común que parte de
Wittgenstein, ni el análisis del lenguaje científico del circulo de
Viena son métodos exclusivos de la teoría del conocimiento:
\begin{enumerate}
\item El análisis del lenguaje ordinario no pueden servir como método
porque no se refieren al problema fundamental del desarrollo del
conocimiento, que se da precisamente en el paso del lenguaje
ordinario al lenguaje científico.
\item Pero tamoco sirve el análisis lógico del lenguaje científico y su
procedimiento de construcción de modelos ideales (Carnap), por la
precariedad de estos mismos y porque tampoco tiene como objetivo
abordar la cuestión del desarrollo de la ciencia.
\end{enumerate}

El único modelo que se puede postular es el de la discusión racional
común a toda pretensión de racionalidad, la ciencia y la filosofía,
que consiste en exponer claramente los problemas y discutir
argumentativa y críticamente las soluciones propuestas.

\subsection{El aspecto filosófico}
\label{sec:orga060c8d}
Popper, mediante su criterio de demarcación, trata de distinguir entre
dos tipos de enunciados, de manera que podamos saber a cual nos
referimos cuando hablamos de filosofía de la ciencia. Define así su
problema de la demarcación:
\begin{enumerate}
\item Es el problema de delimitar lo que es la ciencia empírica respecto
de lo que es sólo pseudociencia. Cuando debe ser considerada
científicamente una teoría o cuál sería su criterio para
definirla como ciencia. No se trata de responder a la cuestión de
la verdad o falsedad de una teoría.
\item Es el problema de definir qué se entiende por experiencia. en
último término, delimitar la ciencia es definir lo que entendemos
por experiencia.\\[0pt]
Hay muchos sistemas teóricos posibles. Entre ellos tiene que
haber alguno que represente el mundo de nuestra experiencia. Este
es la ciencia empírica.
\item Es el problema de definir qué se entiende por realidad y por
conocimiento de la realidad. Se trata del problema filosófico de
nuestra experiencia e interpretación del mundo.
\end{enumerate}

Este intento de identificar este problema como el problema
metodológico de fijar una convención sobre lo que deba entenderse por
ciencia empírica, \textbf{desplaza al ámbito de discusión de la epistemología
a la discusión de la coherencia} de unas reglas metodológicas.\\[0pt]
Pero los criterios de coherencia están definidos por las propias
reglas del método científico. Supone pues, la renuncia al análisis del
concepto de ciencia, de definir lo que habrá que entender por
experiencia y por realidad.

Esto ha dado pie a algunos \textbf{críticos} a denunciar que es inevitable
que lo que quede demarcado como ciencia dentro de la epistemología de
Popper será conocimiento de realidad y lo que queda fuera de la
ciencia será ilusión, engaño o mitología, por más que Popper se empeña
a hacer ver que su problema de demarcación es diferente al problema
positivista de separar enunciados con sentido de los enunciados sin
sentido.\\[0pt]

En cualquier caso, con la idea de demarcación se eleva una determinada
concepción del método de la ciencia natural a método del conocimiento
científico y racional. Se produce una cierta confusión entre
epistemología y metodología como consecuencia previa de la
desvalorización de un tipo de racionalidad, e.g. ciencias humanas,
frente a otro, ciencias naturales.\\[0pt]
Más que la reducción de la racionalidad a la racionalidad científica,
lo característico de Popper es una concepción \textbf{empirista y
positivista} de la ciencia y de la filosofía, y de la razón. El
problema del desarrollo del conocimiento desemboca en el problema
general de la racionalidad.

\subsection{La falsabilidad como criterio de demarcación}
\label{sec:org55ab0ac}

\subsubsection{Actitud científica y actitud dogmática}
\label{sec:org11f131d}
Popper el método científico engloba el empleo de la hipótesis de
alcance general la deducción de enunciados particulares que afirman
posibilidad de observación de un hecho determinado, la comprobación en
la medida de lo posible y la decisión de abrirse a comprobaciones
ulteriores en función de la evolución del conocimiento, la experiencia
y la ciencia.\\[0pt]

Así, el criterio de cientificidad como criterio de demarcación se
basará fundamentalmente en la prueba de falsabilidad para una
teoría. Es decir, la comprobación en circunstancias precisas en las
que podría verse uno obligado a abandonar la teoría.\\[0pt]
Una actitud científica debe afrontar este tipo de pruebas. Lo que pasa
con las teorías psicoanalíticas y marxistas es que tratan de evitar
este tipo de pruebas decisivas utilizando diversos procedimientos:
\begin{enumerate}
\item Empleando una terminología abstracta y generalista que no se
corresponda con hechos observables, pero que cumpla una función
esencial en la teoría.
\item Utilizando conceptos que permiten neutralizar observaciones
capaces de contradecir e incluso refutar la teoría.
\item Realizar una observación selectiva de datos o de hechos que sólo
descubran y se describan en función de la teoría.
\end{enumerate}

Las nociones de ambivalencia en psicoanálisis, de astucia de la razón
o de superación de la dialéctica en el marxismo, postulan
contradicciones, antítesis. Pero no crean problemas, ya que el
freudismo o marxismo se refuerzan con ellas. Estas teoría no pueden
ser refutadas por la evidencia de una incoherencia o de un
conflicto. Lo que no va en sentido de la teoría queda excluido.\\[0pt]
Se trata de la ausencia de distanciamiento crítico, típico de la
creencia, de la ideología, de la fe, etc. y que inmuniza a la teoría
en la que se cree haciéndola infalible. De ahí el dogmatismo que se
refuerza por la habilidad para interpretarlo todo siempre en la
dirección ortodoxa.

\subsubsection{Falsibilidad y contrastabilidad}
\label{sec:org02bee55}
El criterio de la demarcación de Popper es la falsabilidad,
refutabilidad o contrastabilidad de las teorías. Se trata de la
alternativa al principio positivista de verificabilidad y a la
metodología induccionista. La falsabilidada supone:
\begin{enumerate}
\item Un modo concreto de entender la ciencia empírica al que debe
responder a tal criterio.
\item Un modo de entender el valor de la epistemología desde la que
habrá que comprender la función de tal criterio.
\end{enumerate}

En cuanto al \textbf{criterio de delimitación} de los enunciados científicos
de los no científicos, hay dos aspectos a tener en cuenta: su carácter
propiamente empírico y su carácter evolutivo.\\[0pt]
El \textbf{carácter empírico} dela ciencia implica que los enunciados
científicos nos proporcionan información sobre la experiencia y que
son capaces de explicar la experiencia.\\[0pt]
El \textbf{carácter evolutivo} de la ciencia implica también dos notas de los
enunciados científicos: Que tienen un carácter hipotético, i.e., no
tienen una validez definitiva, y que tienen un carácter progresivo,
i.e., que suponen un aumento real de nuestro conocimiento.\\[0pt]

La fasabilidad, como norma metodológica, hará referencia a la
estructura lógica de los enunciados científicos y, por otra, a los
procedimientos de la investigación científica —a las reglas
metodológicas que se utilizan para manejar enunciados científicos. \\[0pt]
Afirmar que una teoría pertenece a la ciencia empírica es afirmar algo
acerca de algo que no es ni tautológica —que no afirma nada—, ni
contradictoria —lo afirma todo—, ni metafísica —afirma algo que no
puede ser comprobado en la experiencia.\\[0pt]
Enunciado científico es, pues, enunciado que afirma algo sobre la
experiencia. La forma de saberlo es contrastando el enunciado con la
experiencia.\\[0pt]
El problema que surge es saber cómo se puede llevar a cabo esta
contrastación.\\[0pt]

Una teoría científica no es directamente contrastable con la
experiencia, ya que esta es particular y concreta, mientras que la
teoría es universal y abstracta. Sin embargo, los enunciado básicos
—enunciados deducidos de una teoría que describen hechos de la
experiencia— sí pueden ser directamente comparados con la
experiencia.\\[0pt]
\textbf{Contrastar una teoría} con la experiencia es deducir enunciados
singulares a partir de ella y verificar en la práctica estos
enunciados. \\[0pt]
Pueden pasar dos cosas:
\begin{enumerate}
\item Que los enunciados singulares sean refutados por la
experiencia. La teoría queda también refutada.
\item Que los enunciados singulares sean verificados por la
experiencia. La teoría no queda verificada, sino solo probada
provisionalmente.
\end{enumerate}

Los enunciados teóricos sin enunciados universales, que comprenden
infinitos casos de enunciados particulares. La verdad de un enunciado
universal no se puede dar por la verdad de un enunciado particular.\\[0pt]
De esta manera parece que la falsabilidad, como criterio de
demarcación, es útil para dar cuenta del carácter empírico de las
teorías empíricas.\\[0pt]

Pero para llegar a identificar este primer aspecto del carácter
empírico de la ciencia con la falsibilidad de los enunciados
científicos se han llevado a cabo varias reducciones.\\[0pt]
Solo podemos saber si un enunciado habla acerca de la experiencia si
habla acerca de hechos singulares, y si es posible que lo que afirma
acerca de hechos singulares sea falso.\\[0pt]

Esto tiene como consecuencia una limitación empirista del concepto de
ciencia que deja fuera de la racionalidad científica a todo conjunto
de enunciados que no se refieran a experiencias o hechos. Se dejan
fuera los enunciados filosóficos y muchos enunciados metafísicos.\\[0pt]
Sin embargo, los enunciados metafísicos, aunque no falsables, juegan
un papel importante en la ciencia, pero en la medida que, mediante la
evolución de la ciencia, lleguen a ser falsables.\\[0pt]

La distinción entre falsabilidad y falsación se corresponde con la
distinción con la distinción entre el problema de la demarcación y el
problema del método de la ciencia.\\[0pt]
Todo sistema empírica tiene que ser, además, falsable. Los enunciados
que satisfacen la condición de coherencia son incapaces de efectuar
discriminación alguna entre dos enunciados cualesquiera. Los que no
satisfacen la condición de falsabilidad no son capaces de efectuar
discriminación entre dos enunciados cualesquiera que pertenezcan a la
totalidad de todos los enunciados empíricos básicos posibles.

\subsection{La concepción de la ciencia}
\label{sec:orgbdf28cc}
\subsubsection{Enunciados básicos y teorías}
\label{sec:orgfe74f37}
El primer aspecto a determinar es la relación entre ciencia y
experiencia. La ciencia se concibe como un conjunto organizado de
enunciados. El problema de la relación entre la experiencia y la
ciencia se formulará como problema de las relaciones entre enunciados
teóricos y enunciados de hechos, i.e., entre teorías y enunciados
básicos.\\[0pt]
Las teorías se caracterizan respecto de los enunciados básico por ser
falsables, y éstos respecto de a la teoría por ser sus posibles
falsadores. Lo relevante es el método de llevar a cabo la
contrastación entre unos y otros, el método de la falsación de la
teoría o de su sometimiento a falsación.\\[0pt]

Contrastar la teoría será contrastar los enunciados teóricos con los
enunciados básicos. Pero esto no resuelve el problema del empirismo
lógico, el de la justificación del carácter empírico de los enunciados
básicos mismos, el problema de la relación entre lenguaje y
experiencia, a través de la relación entre enunciados y hechos y
experiencia de hechos. Este problema se desdobla en la epistemología
de Popper \textbf{en dos problemas}: El problema de la aceptación de
enunciados básicos para la falsación de una teoría y el problema de la
justificación de esa aceptación.

\subsubsection{Aceptación de enunciados básicos}
\label{sec:orgf0491d3}
Los enunciados básicos cumplen dos funciones en el sistema científico:
\begin{enumerate}
\item obtener una representación exhaustiva de todos los enunciados
básicos lógimanete posibles.
\item Los enunciados básicos aceptados constituyen la base para la
corroboración de la hipótesis. Si contradicen la teoría,
admitimos que nos proporcionan motivo suficiente para la
falsación de ésta. únicamente en el caso de que corroboren una
hipótesis falsadora.
\end{enumerate}

La forma de como deben aceptarse los enunciados básicos para que
cumplan su función de contrastación empírica, Popper señala dos reglas:
\begin{enumerate}
\item No aceptar enunciados básicos esporádicos, que no estén en
conexión lógica con otros enunciados.
\item Se aceptan enunciados básicos como constatación de teorías cuando
se suscitan cuestiones esclarecedoras de ésta.
\end{enumerate}

Estas dos reglas son circulares en el sentido de que la aceptación de
enunciados básicos se lleva a cabo con vistas a la contrastación de
teorías, y por esto, precisamente, no se aceptan enunciados básicos
aislados.\\[0pt]
Los enunciados básicos son enunciados existenciales aislados. Se
refieren a acontecimientos individuales irrepetibles. Por lo tanto,
son incapaces de ser objetivos si se los toma aisladamente. Es decir,
las teorías no son enteramente justificables o verificables, pero sí
siempre contrastables. La objetividad de los enunciados científicos
descansa en el hecho de que pueden contrastarse intersubjetivamente
—que sucede en la comunicación intelectual entre dos sujetos.\\[0pt]

A partir de un enunciado básico aislado lo único que podemos inferir
son enunciados estrictamente existenciales, que no son contrastables
ni falsables. Sólo con los enunciados básicos que están en relación
con otros de carácter más teórico a partir de los cuales se puedan
deducirse, se puede garantizar que sean objetivos, \textbf{contrastables
intersubjetivamente}, indispensable para que puedan servir como
enunciados de contraste empírico de las teorías.

\subsubsection{Ciencia y experiencia}
\label{sec:orgb59213d}
La ciencia, según Popper, es un conjunto de enunciados lógicamente
estructurados y de cuyas relaciones se ocupa la epistemología. Es como
un sistema de disposiciones y que puede ser materia de estudio de la
psicología, puede estar unido a sentimientos de creencia o de
convicción: quizá en un caso al sentimiento de sentirse obligado a
pensar de una manera determinada y, en otro, al de certidumbre
perceptiva.\\[0pt]
El epistemólogo tiene que \textbf{prescindir de estos sentimientos} y de su
intensidad y adoptar el único camino que existe para asegurar la
validez de los razonamientos lógicos, i.e., prepararlos mediante el
análisis —en el caso de las ciencias analíticas— o la presentación de
hechos —en el caso de las ciencias empíricas— para su contrastación.\\[0pt]
Las relaciones lógicas sólo se dan entre enunciados, es imposible
pretender la fundamentación lógica de enunciados por algo que no son
enunciados, i.e. por experiencias.\\[0pt]

Es necesario pues, \textbf{definir experiencia}. Popper acepta el principio
empirista según el cual sólo captamos los hechos mediante la
observación, aunque no justifica ni fundamenta la verdad de ningún
enunciado. La experiencia no nos proporciona conocimiento, sólo lo
hace en la medida en se afirma un enunciado que describe el
hecho. Pero el enunciado, ya por serlo, trasciende la observación
empírica, inmediata. Toda observación está mediatizada ya por la
teoría, de modo que las observaciones perceptivas puras son
imposibles.\\[0pt]

\begin{quote}
La decisión de aceptar un enunciado básico y darse por satisfecho con
él tiene una conexión causal con nuestras experiencias, especialmente
con las perceptivas. Pero no se justifican los enunciados básicos por
medio de ellas. Las experiencias pueden motivar una decisión la
adopción o rechazo de un enunciado, pero ningún enunciado básico puede
quedar justificada por ellas.
\end{quote}

La experiencia es el resultado de decisiones y de interpretaciones
libres. Los hechos aparentes de la experiencia son siempre
interpretaciones a la luz de teorías, por lo que tienen carácter
hipotético, conjetural de todas las teorías.\\[0pt]
Las observaciones, y particularmente los enunciados de observaciones y
los resultados experimentales, son siempre interpretaciones de los
hechos observados realizadas a la luz de teorías.

\subsubsection{La superación del dualismo realidad-teoría}
\label{sec:orga29c0ed}
Si se tiene contacto con el mundo a través de una teoría, con lo que
se tiene contacto es con la teoría, no con el mundo. Si se afirma que
a través de la teoría se da una existencia independiente del mundo, se
tiene que poder distinguir, en la propia teoría, què es lo teórico y
que es el mundo —de lo contrario no será independiente. Esta
dificultad, que surge con el dualismo realidad-teoría,
realidad-lenguaje, desaparecería con la concepción khuniana de las
tradiciones científicas. Los ap priori de la experiencia, de toda
experiencia posible, son hechos del lenguaje. el sujeto, científico,
es una encrucijada de discursos, su propia consistencia es el mundo
como mundo de la tradición que les constituye como lenguaje. De esta
manera, el yo es siempre es mundo y no hay lugar para estos dualismos.

\subsubsection{Decisionismo y convencionalismo}
\label{sec:org799906b}
Popper en contraste con el círculo de Viena sostiene que es imposible
reducir enunciados básicos a la experiencia directa y, en cuanto que
son enunciados teóricos, es imposible reducirlos a cualquier tipo de
experiencia singular.\\[0pt]

Los enunciados básicos son los únicos que pueden ser falsadores de
teorías. Su justificación de su aceptación es imposible, sólo queda la
opción de de aceptarlo o no. Es decir, la aceptación de los enunciados
básicos recae sobre el científico. Es fácil que la comunidad
científica se ponga de acuerdo sobre esta decisión. Es una aceptación
convencional y no definitiva. Esto introduce la posibilidad de una
regresión infinita pero que no resulta perjudicial, ya que no se
pretende probar ningún enunciado por medio de la teoría, sino que
solamente se trata de una posibilidad lógica que siempre se verá
detenida por lo convenido de los científicos en un momento
determinado, en el que cadena de contrastaciones se detiene.\\[0pt]

Así, no es posible responder al problema del conocimiento por medio de
ninguna teoría científica, i.e., falsable.\\[0pt]
La cuestión de que si esta metodología nos proporciona un conocimiento
empírico válido, no solo formalmente sino realmente, es una cuestión
que sólo puede resolverse mediante la fe metafísica en la
inmutabilidad de los procesos naturales. El método científico
presupone la inmutabilidad de los procesos naturales o el principio de
uniformidad de la naturaleza.

\subsubsection{La actitud refutadora como supuesto del carácter empírico de la ciencia}
\label{sec:org6ea1c3f}
Del decisionismo de Popper car en una cierta arbitrariedad, que es el
que él critica en las teorías convencionalistas. Para salir del
psicologismo positivista, se acerca primero al convencionalsismo y,
para superar las sus consecuencias, su salida es un cierto
pragmatismo. Es decir, trata de corregir su convencionalismo
insistiendo en que la contrastación responde a una actitud, decidida,
de falsar, de refutar una teoría.

La actitud refutadora es lo que diferencia entre al convencionalismo
de Popper del convencionalismo clásico, y es la moralidad del
científico la única base del carácter empírico de los enunciados. Su
teoría del conocimiento sostiene que la base empírica de todas las
teorías son los intentos de refutación (tests). Es el presupuesto
último del carácter empírico de la ciencia, sólo esta actitud permite
el desarrollo. Empirismo y desarrollo quedan así conectados
metodológicamente. La garantía empírica de la ciencia es la actitud
moral del científico.

\subsection{Desarrollo del conocimiento científico}
\label{sec:org1c432e7}

\subsubsection{Historia de la ciencia e investigación científica}
\label{sec:orgd644924}
Para la evolución y desarrollo de la ciencia es necesario:
\begin{enumerate}
\item Su carácter abierto y nunca definitivo
\item Su carácter progresivo en virtud del cual nuestro conocimiento va
aumentando y extendiéndose cuantitativamente a cada vez más
fenómenos o aspectos de la realidad
\end{enumerate}

La \textbf{epistemología} de Popper se caracteriza por el progreso de la
ciencia para explicar la ciencia actual y tratar de darle una
explicación que sirva a ese progreso de orientación y guía. \\[0pt]
El \textbf{desarrollo del conocimiento} es un \textbf{fenómeno histórico} que
exigirá una teoría de la historia de la ciencia, para así poder ser
explicado y guiado. Pero también un \textbf{fenómeno del proceso de
investigación científica} y exigirá una teoría de la investigación
científica y una metodología consecuente con dicha teoría.\\[0pt]
Los elementos constitutivos necesarios de la epistemología de Popper
son:
\begin{enumerate}
\item Una teoría de la historia de la ciencia. Ésta deberá comprender
una teoría de los límites del conocimiento científico y una
teoría del progreso del conocimiento científico —i.e. una teoría
del aumento cuantitativo y cualitativo de la ciencia, en la cual
irá incluida una explicación de la diferenciación histórica de las
ciencias.
\item Una teoría de la investigación científica. Deberá comprender una
teoría del carácter provisional de nuestro conocimiento, i.e.,
una teoría de las relaciones entre nuestro conocimiento universal
y nuestra experiencia. Pero también una teoría del aumento de
nuestro conocimiento, del aprendizaje, del descubrimiento de
teorías y fenómenos empíricos y de la construcción de leyes y
teorías.
\item Una guía eficaz para promover el desarrollo de la ciencia. Habrá
de proporcionar directrices para una política científica, i.e.
planificación de la investigación científica a nivel social,
planicificación que habrá de tener en cuenta tanto el carácter
siempre provisional del conocimiento científico, su fiabilidad,
como la continua necesidad del aumento cuantitativo y cualitativo
de la ciencia. Pero también normas para la investigación
científica, i.e., normas que garanticen la crítica, metodología
del antidogmatismo, y normas que posibiliten el descubrimiento
científico.
\end{enumerate}

\subsubsection{La idea de un acercamiento a la verdad}
\label{sec:org01553b3}
Pero Popper se limita al nivel de la investigación científica, en la
que aporta una metodología en los términos convencionalistas ya
expuestos cuya normas para impulsar el desarrollo del conocimiento
cinetíficos son:
\begin{enumerate}
\item Es preciso inventar teorías cada ves más falsables, i.e., teoría
que permitan hablar de hechos nuevos.
\item Es preciso intentar siempre refutar estas teorías.
\end{enumerate}

Estas mismas reglas metodológicas sirven también para explicar el
desarrollo de la ciencia. Vienen a reflejar que si la ciencia se
desarrolla es porque los científicos se inventan teorías cada vez más
falsables y porque intentan continuamente falsar dichas teorías —el
\textbf{aspecto negativo} del desarrollo de la ciencia.\\[0pt]
El \textbf{aspecto positivo} —el aumento del conocimiento— Popper intenta
poner de acuerdo su metodología falsacionista con una teoría realista
del aumento del conocimiento científico. Interpreta el conocimiento de
la ciencia como un acercamiento a la verdad absoluta a través del
concepto de verosimilitud.\\[0pt]

Doble sentido de la idea de verdad. Por una parte, la entiende como el
sentido clásico realista, como correspondencia con los hechos. Por
otra, en sentido idealista, como verdad absoluta inalcanzable, como
principio regulador.\\[0pt]
Si se entiende el progreso de la ciencia en función de un acercamiento
a la verdad absoluta, se debe medir esta aproximación y sólo puede ser
con supuestos a priori: el progreso de la ciencia no dependerá del
grado de adecuación de nuestro conocimiento a la realidad, sino de la
cantidad de enunciados básicos, de hechos, que nuestras teorías
permiten inferir.\\[0pt]

El problema del \textbf{aumento del conocimiento} no consiste sólo en que
cada veza nuestras teorías se refieran a más hechos, sino en que cada
vez conocemos más y mejor la realidad.\\[0pt]
Para eso, Popper acaba admitiendo el éxito temporal de las teorías
científicas. Es decir, termina admitiendo, no sólo que el progreso
consiste en que tenemos teorías cada vez más universales, sino en que
algunas de estas teorías tienen éxito en sus predicciones, que son
provisionalmente verificadas.\\[0pt]
El falsacionismo no es, pues, en rigor, una teoría del progreso de la
ciencia, sino una simple metodología antidogmática junto a la cual
aparece una constatación empírica del progreso científico como aumento
del éxito de la ciencia en su explicación de la realidad.

\subsubsection{Probabilidad y corroborabilidad}
\label{sec:org54a4e21}
Para Popper, el aumento del conocimiento científico así como su éxito
tiene que ser un punto de llegada —mientras que para la lógica
inductivo constituye el centro de la epistemología. La demostración de
que podemos avanzar en el conocimiento de la realidad con actitudes e
instrumentos falsadores.\\[0pt]

Una teoría que se ha tratado de falsar pero que no ha quedado falsada,
sigue siendo una teoría válida, aunque no se la puede dar por
verdadera ya que puede resultar falsa en un futuro. Al ser un
enunciado universal, no puede ser nunca definitivamente verificada,
siempre será una conjetura provisional.\\[0pt]
Es preciso cambiar el concepto de verdad por el de probabilidad. Al
atribuir una probabilidad a un hipótesis estamos haciendo una
evaluación de la misma que puede ser verdadera o probable. Esto nos
lleva a un \emph{cul de sac}:

\begin{quote}
En cuanto a la evolución de las teorías podemos afirmar que es
verdadera o que es probable. Si se la considera verdadera tiene que
ser un enunciado sintético verdadero a priori. Si se la toma como
probable, necesitamos una nueva evaluación, una evaluación de la
evaluación, de orden superior. Entonces estamos atrapados en una
regresión infinita. La apelación a la probabilidad de la hipótesis es
incapaz de mejorar la precaria situación de la lógica inductiva.
\end{quote}

Esta posición tiene relación con la crítica de Popper a la
epistemología inductiva, pues creer en la lógica de la probabilidad
supone creer que se llega a la evaluación a través de un principio de
inducción que adscribe probabilidades a las hipótesis inducidas.\\[0pt]
Pero si se vuelve a atribuir una probabilidad a este principio
estaremos en un regreso al infinito. Si por el contrario se le
atribuye la verdad, nos enfrentamos con el dilema de elegir entre la
regresión infinita y el apriorismo.\\[0pt]
Es decir, la lógica inductiva no nos sirve para caracterizar una
hipótesis cuando ésta ha sido contrastada con la experiencia y ha
resistido la prueba. No podemos decir de ella que sea verdadera y ni
siquiera podemos decir que sea probable. Lo único que podemos decir es
que está más o menos corroborada.\\[0pt]

Que una teoría esté corroborada implica sólo que tal teoría es
aceptable de manera provisional. No dice nada acerca ni de la verdad
ni de la probabilidad.\\[0pt]
Así, decir que una teoría está corroborada cuando es refutable pero
que no ha sido refutada por los intentos hasta el momento, es decir
que se ha aceptado una serie de enunciados deducibles de ella y hemos
tomados esta decisión porque estos enunciados no eran deducibles de
nuestro anterior conocimiento falsamente falsable.\\[0pt]

En conclusión, el conocimiento necesita, en último término, instancias
extralógicas —la sinceridad de nuestros intentos por refutar las
teorías, base del falsacionismo.\\[0pt]
El progreso de la ciencia depende de la universalidad creciente de
nuestras teorías, pero también de la sinceridad de nuestros tientos de
refutación.\\[0pt]
Sin embargo, estas condiciones sólo garantizan que la ciencia no de
detenga, no explican que aumente nuestro conocimiento.

\subsubsection{El mundo 3}
\label{sec:orga6f75db}
Lo que diferencia al ser humano del resto de seres vivos es que
gracias al lenguaje, que permite la representación y la crítica, puede
desarrollarse el proceso de aprendizaje y de progreso del conocimiento
sin involucrarse físicamente en él. El desarrollo de la ciencia se
produce:
\begin{enumerate}
\item Los científicos inventan y someten a prueba las teorías
destinadas a resolver problemas que se plantean a partir de
teorías existentes.
\item Entre las teorías se produce una competencia que viene a ser como
una lucha por la supervivencia. Unas teorías son eliminadas, bien
porque no sobreviven a una prueba de falsabilidad, bien porque
las sustituyan otras teorías más poderosas capaces de resolver
más problemas.
\end{enumerate}


Popper tiene varias tesis del mundo 3:
\begin{enumerate}
\item En el mundo 3 podemos descubrir nuevos problemas que estaban allí
antes de ser descubiertos y antes de que se hiciesen
conscientes. Esto es, antes de que en el mundo 2 apareciese algo
correspondiente a ellos.
\item En algún sentido el mundo 3 es autónomo. Podemos hacer en este
mundo descubrimientos teóricos del mismo modo de que podemos
hacer descubrimientos geográficos en el mundo 1.
\item Tesis fundamental: casi todo nuestro conocimiento subjetivo
(mundo 2) depende del mundo 3 (virtualmente), de las teorías
formuladas lingüísticamente.
\end{enumerate}

La plena conciencia de sí mismo depende ed todas estas teorías
(mundo 3) y de que los animales, aunque sean capaces de tener
sentimientos, sensaciones, memoria, conciencia, no poseen plena
coincidencia de sí mismos que constituye uno de los resultados del
lenguaje humano y el desarrollo del mundo 3 específicamente humano.\\[0pt]

El desarrollo del conocimiento científico, se desarrolla fuera del
organismo humano. El conocimiento científico es relativamente autónomo
respecto de los individuos, no está inscrito en el genoma ni en el
cerebro. El conocimiento está en los libros y bases de datos
disponibles para las nuevas generaciones que continuarán desarrollando
a través de la invención y de la crítica. Este conjunto de
conocimiento es el que Popper llama mundo 3, Una producción
específicamente humana por medio el lenguaje. Es el mundo de los
problemas y de las hipótesis teóricas.

\subsection{La apuesta por un racionalismo crítico}
\label{sec:org773c968}
Popper sostiene que la posibilidad de encontrar puntos de referencia
últimos en los que basar nuestro saber es una ilusión. Propone una
actitud racionalista crítica que acepte que cualquier teoría es
provisional, revisable, superable.\\[0pt]

Este mismo carácter de provisionalidad afecta al racionalismo crítico
mismo, que no puede afirmar de manera universalmente obligatoria que
haya que adoptar una actitud racional de experimentación y de
discusión argumental. Optar por la razón no es una cuestión racional,
sino una decisión de la que sólo podemos decir que da buenos
resultados, pero no que sea lógicamente necesaria a priori.\\[0pt]
Se trata pues de un racionalismo suspendido en un abismo de
irracionalidad, ya que en favor de la decisión por la razón no caben
argumentaciones no demostraciones racionales.\\[0pt]

Se trata de una tesis muy controvertida, que ha sido objeto de debate
en polémicas entre Popper y la escuela de Frankfurt (Habermas), que
defiende la posibilidad y la necesidad de apuntalar e incluso de
fundar racionalmente la elección a favor de la discusión racional
universal.



\section{La ciencia en la historia y en la sociedad}
\label{sec:orge42e81c}

\subsection{El rechazo de la concepción dominante de la ciencia}
\label{sec:org3736d3a}
La nueva teoría de la ciencia —Kuhn, Laktatos, Polanyi, Hanson,
Feyerabend— somete a duras críticas los siguientes aspectos de lo que
llaman \emph{imagen dominante de la ciencia} (desde Descartes hasta Popper):
\begin{enumerate}
\item La idea de un método universal y único que sería el propio de la
ciencia moderna (e.g. el método inductivo).
\item La idea de un criterio universal de cientificidad que permitiría
distinguir rigurosamente la ciencia de las pseudociencias o
teorías no científicas (e.g. criterio de falsabilidad)
\item La idea de un progreso lineal, continuo y acumulativo de la
ciencia, un progreso en línea recta sin rupturas,
discontinuidades no revoluciones. El desarrollo del conocimiento
es entendido como una acumulación de conocimientos cada vez más
amplia. Una teoría reemplazaría a otra teoría incluyéndola, por
lo que sería más potente que la anterior, que contiene más
verdad.
\item La idea de realismo y de la objetividad de la ciencia. La ciencia
sería una descripción cada vez más precisa de la realidad
objetiva, a la que se aproximaría a la manera de una imagen cada
vez más precisa.
\end{enumerate}

Las trasnformaciones profundas en el seno de la nueva epistemología
científica —la imposibilidad de separar al sujeto y al objeto en la
investigación; el rechazo de un concepto estático de verdad y su
superación por una verdad históricamente situada; la caída de un
método como acceso a la verdad— configuran un contexto de referencia
desde el que se tratan de superar ciertos principios dogmáticos de la
teoría tradicional del conocimiento. Este giro de la episteme
contemporánea impulsa una mayor conciencia de los procesos históricos
de constitución de la razón.

\subsection{Kuhn: El devenir de la ciencia como sucesión de paradigmas}
\label{sec:org7c2a7a6}

\subsubsection{El concepto de paradigma}
\label{sec:org6d3f29a}
Un paradigma es un ideal común de explicación, un conjunto de formas
simbólicas, un modelo teórico y una serie de métodos para la solución
de problemas empleados en la formación de los investigadores. Es el
marco en cuyo interior se desarrolla la actividad científica en un
momento dado. Constituye una especie de matriz común gracias a la cual
los especialistas hablan de las mismas cosas en los mismos términos y
adoptan las mismas actitudes. Es como la matriz simbólica de una
comunidad científica:

\subsubsection{Ciencia normal y revolución científica}
\label{sec:org7709a3c}
La actividad científica se desarrolla en el marco de un paradigma
vigente que es el que define ciencia norma en un momento dado. En la
evolución histórica de la ciencia se distinguen períodos de ciencia
normal, marcados por la aceptación general de un paradigma, y
períodos de ciencia normal, marcados por la aceptación general de un
paradigma, y períodos de revolución, en los que hay varios paradigmas
en competición. La elección de uno o de otro no depende de criterios
lógicos o metodológicos, sino del conjunto de valores del grupo social
al que el científico pertenece.\\[0pt]

El ejercicio de la ciencia normal consiste en aplicar el paradigma
dentro del que se trabaja y resolver los problemas que se presentan
sin cuestionar el paradigma mismo. Cuando surgen experiencias u
observaciones que cuestionan el mismo paradigma, se suele pensar que
es un error de cálculo o de experimentación. Este sería un punto que
se opone a la noción de Popper de falsación.\\[0pt]
Según Kuhn, para ser un buen científico se tiene que tener bien
asimilado el paradigma en el que se ejerce. Así, el espíritu crítico
no sería una característica de la investigación científica normal.\\[0pt]
Sólo cuando los contraejemplos no buscados aumenten y resulten cada
vez más difíciles de ignorar se empieza a producir un cuestionamiento
del paradigma pudiéndose iniciar un período de revolución.(E.g.,
revolución copernicana, con el mecanicismo que culmina con
Newton. Otro ejemplo el paso de la física Newtoniana al relativismo ed
Einstein.)\\[0pt]

Puesto que los paradigmas son inconmensurables, los componentes del
paradigma antiguo dejan simplemente de existir en el nuevo. Esto
significa que diferentes paradigmas implica necesariamente fenómenos
distintos, aunque se pueda compartir terminología (el nombre no hace
la cosa, planeta, estrella, gen etc.). Así, no hay continuidad ni
profundización en el desarrollo del saber.


\subsubsection{La crítica al modelo fundacionista de la ciencia}
\label{sec:orgad65e8d}
La metodología de validación de los neopositivistas son sólo válidos
en un período de ciencia normal, que se ejerce en un paradigma
dado. En los períodos de revolución en el que compiten varios
paradigmas, no hay estructuras lógicas o metodológicas para el
asentimiento a uno u otro paradigma. La elección depende sobretodo de
los sistemas de valores presentes en los entornos a los que los
científicos ejercen. El paso de un paradigma a otro no se explica desde
un punto de vista lógico, no hay secuencias de modificaciones
regulares, explicables, cada una en función de la precedente, sino
cambios súbitos y masivos tras los cuales sobrevive un nuevo período
de ciencia normal durante el que los criterios de logicidad están de
nuevo en vigor hasta otra nueva revolución.\\[0pt]
La adopción de un nuevo paradigma también equivale a una
reestructuración de las implicaciones sociales, y no puede
comprenderse sin referencia a una diversidad de factores psicológicos
y sociales tanto como lógicos.\\[0pt]

Cualquier conjunto de fenómenos es susceptible de ser
explicado-interpretado de maneras diversas, a partir de diversas
teorías o paradigmas, y todas convincentes. Esta diversidad de
explicaciones plausibles depende de lo que se pretenda con la teoría o
de lo que se espere de ella, predecir exactamente acontecimientos,
dominar técnicamente la naturaleza o gozar de una concepción
psicológica adecuada. Esto implica que no tiene sentido hablar de un
metalenguaje universal o de una metateoría neutra para decidir de
manera racional entre diversas teorías i diversos paradigmas. Tampoco
hay un acceso a la realidad que haga válida la concepción de la verdad
como adecuación entre teoría y realidad. Toda realidad está
mediatizada por una interpretación que es de naturaleza lingüística. Y
al no haber ni referencia exterior, ni criterio absoluto como la
razón, que permita comparar y juzgar los paradigmas, no se puede decir
que el paso de un paradigma a otro constituya un progreso en el
desarrollo del conocimiento.

\subsection{Feyerabend: hacia una teoría anarquista del conocimiento}
\label{sec:org91f4a41}

\subsubsection{Contra el método}
\label{sec:orga990ea1}
Feyerabend se enfrenta directamente a la reducción dela explicación
científica a la predicción, según el esquema neopositivista. El nuevo
modelo de explicación deductivista del neopositivismo se presentaba
como uno de los mejores resultados dela metodología positivista. Lo que
Feyerabend discute es el presupuesto de base según el cual teorías que
compiten entre sí pueden siempre compararse formalmente de tal manera
que se pueda llegar a aceptar una y rechazar otras. En este sentido,
siguiendo a Popper, en la idea de que es deseable que ninguna teoría,
ninguna ciencia normal según Kuhn, queda tomar posesión de un dominio
científico determinado. Muchas posibilidades deben ser mantenidas al
mismo tiempo, de tal forma que toda posibilidad de apertura sea
ensayada, y toda configuración puesta a prueba.\\[0pt]

La puesta a prueba no consiste en comparar una teoría con una
experiencia, sino en ver como se comportan varias teorías que son
mutuamente incompatibles, sin que ninguna pueda dar cuenta por sí sola
de todos los hechos.\\[0pt]
Para Feyerabend, una teoría es una manera de ver el mundo, no una
forma de poner en orden un conjunto de datos. Puesto que las teorías
que defendemos tienen una influencia sobre nuestras creencias y
nuestras esperanzas, tienden a modelar nuestra experiencia. No hay un
conjunto neutro de hechos o de relaciones de observación que una
teoría u otra puedan explicar. Los logicistas han creído siempre que
todo lo que se debe averiguar para explicar la naturaleza de la
confirmación científica es el tipo de teoría que mejor explica
determinado conjunto de hechos o datos. Pero si el conjunto de hechos
en cuestión no está dado, sino que depende del contexto teórico en el
interior del cual ha sido formulado. Las teorías en competición son
inconmensurables, que no hay medio de compararlas directamente y
lógicamente las unas con las otras.\\[0pt]

Por tanto, la exigencia de que una teoría deba estar de acuerdo con
las que eran anteriormente válidas en su mismo dominio,no podría
cumplirse por esta misma razón. La búsqueda de una lógica inductiva
susceptible de llevar a generalizaciones cada vez mayores, está
destinada a fracasar.\\[0pt]
Para Feyerabend hay que descartar el hecho de que las teorías pueden
ser establecidas por la experiencia. Así que no hay otros medios que
los psicológicos para establecer el acuerdo.\\[0pt]
Un ejemplo de esto es la defensa galileana de la teoría
copernicana. Galileo no estableció el nuevo sistema, como la teoría
positivista nos la haría creer. El sistema copernicano estaba ya en
desacuerdo con los hechos desde muchos puntos de vista. Galileo debió
recurrir a determinadas estrategias psicólogas para que se aceptase la
teoría: su arte de persuadir, su habilidad retórica, etc. De modo que
esa idea de un método científico, como tal, no es más que un
mito. Recurrir a ella es creer tener un acceso privilegiado a la
verdad de manera objetiva. Contra ello, Feyerabend proclama que no hay
método.

\subsubsection{Ciencia y política}
\label{sec:org35610cf}
Feyerabend destaca el papel de la argumentación, de la persuasión,
dela retórica y de la propaganda en el triunfo de una teoría
científica. La vigencia y aceptación de una teoría científica no se
debe a sea más verdadera o más objetiva, por lo que la autoridad del
científico no se legitima en virtud de una referencia neutra a la
realidad. Será un poder de naturaleza política. Hacer ciencia también
es hacer política, aunque de una manera discreta, incluso
inconsciente, pues siempre se trata de conquistar el poder.\\[0pt]
Algunas observaciones:
\begin{enumerate}
\item No se pude negar que la política interviene también en la
investigación y el desarrollo científico. La política es la que
decide qué sectores promover, el destino de las subvenciones a
tal o cual programa de investigación. Los científicos, si quieren
llevar a buen término sus investigaciones, están obligados a
tener muy en cuenta al poder. No obstante, debe subrayarse que la
politización procede de fuera de la investigación
científica. Puede orientarla, estimularla, frenarla, pero no
puede sustituirla. Un debate científico no se produce de la misma
manera que un debate político. En el debate político se vence a
un oponente y se impone el propio programa político. El poder
político es resultado de la dominación de la voluntad de un
individuo sobre los otros. Por otro lado, un verdadero debate
científico se zanja mediante la experiencia, la resistencia o no
de la realidad, que confirma o desconfirma una predicción y que
se deja o no manipular técnicamente con vistas a la realización
de una predicción y que se deja o no manipular técnicamente con
vistas a la realización de un objetivo.
\item No se puede hacer del científico un creyente que se limita a
aplicar las recetas de la investigación en los períodos de
ciencia normal y que cambia por motivaciones en las que tienen un
papel determinante el azar, el contexto histórico, los intereses
personales y corporativos, las circunstancias sociales, la fuerza
retórica y la propaganda. No hay progreso interno al desarrollo
científico no progreso en relación con las concepciones
precientíficas (mitos, metafísica, religiones). La ciencia es
pues, una manera de relacionarse con el mundo que no admite
privilegio de ninguna clase. Esto está en la línea de la
filosofía del lenguaje de Wittgenstein. Con Kuhn y Feyerabend,
pretenden extraer su concepción de la ciencia del análisis
objetivo de la historia misma de las ciencias.
\end{enumerate}
\end{document}