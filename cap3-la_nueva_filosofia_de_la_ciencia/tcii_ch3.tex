% Created 2023-05-25 dj. 22:55
% Intended LaTeX compiler: pdflatex
\documentclass[a4paper, 11pt, twocolumn, spanish]{article}
\usepackage[utf8]{inputenc}
\usepackage[T1]{fontenc}
\usepackage{graphicx}
\usepackage{longtable}
\usepackage{wrapfig}
\usepackage{rotating}
\usepackage[normalem]{ulem}
\usepackage{amsmath}
\usepackage{amssymb}
\usepackage{capt-of}
\usepackage{hyperref}
\usepackage[T1]{fontenc}
\usepackage[margin=.75in]{geometry}
\setlength\parindent{0pt}
\author{Jordi Serra}
\date{\today}
\title{Apuntes de Teoría del Conocimiento ii\\\medskip
\large III La nueva filosofía de la ciencia}
\hypersetup{
 pdfauthor={Jordi Serra},
 pdftitle={Apuntes de Teoría del Conocimiento ii},
 pdfkeywords={},
 pdfsubject={},
 pdfcreator={Emacs 27.1 (Org mode 9.6.2)}, 
 pdflang={English}}
\begin{document}

\maketitle
\tableofcontents


\section{Introducción}
\label{sec:org3dd3606}
\subsection{El final del modelo clásico de cientificidad}
\label{sec:org439ec82}
En las últimas décadas del s. XX hubo un profundo cambio en el ideal
clásico de cientificidad. Se ha perdido el sentido lineal y
acumulativo del devenir científico y se ha introducido la acción
humana en la historia de la ciencia. La ciencia deja de ser un
lenguaje perfecto. El mito de la presunta superioridad objetiva de la
ciencia se viene abajo , al quedar de manifiesto la inserción de la
elección en la práctica científica cotidiana.\\[0pt]

El ideal clásico de cientificidad estaba representado por el
neopositivismo lógico. En contraposición a éste, podemos simplificar
en tres las nociones importantes en la nueva teoría de la ciencia.
\begin{enumerate}
\item Los hechos científicos son relativos al sistema conceptual
utilizado para poner en evidencia su articulación, susceptibles de revisión a medida que se vaya modificando el sistema.
\item La evaluación de las hipótesis científicas, particularmente en
las teorías que revisten un alto grado de complejidad, implica un
tipo de apreciación que supone un largo entrenamiento, no siendo
reductible a reglas formales explícitas de inferencia. Una
reconstrucción puramente lógica de la confirmación científica
debe necesariamente resultar incompleta.
\item Teoría rivales pueden ser inconmensurables entre sí. Puede que
sus sistemas conceptuales respectivos no se corresponden
suficientemente como para hacer posible una comparación
directa. O también los valores metodológicos que se incorporan a
las teorías no son los mismos. En consecuencia, no se dispone de
una metodología que haga posible una comparación entre ellas.
\end{enumerate}

\subsection{Teoría de la ciencia y práctica científica}
\label{sec:orgeb7e84a}
Hasta la fecha, se sostenía que las proposiciones verdaderamente
científicas debían fundarse bien en la intuición (Descartes), bien en
relaciones de observación (Locke), bien en principios sintéticos a
priori (Kant).\\[0pt]

Durante el s. XIX, el auge de las ciencias atrae la atención hacia el
estudio de sus métodos de trabajo y a su historia misma, desviándose
de este punto de vista. La aparición del círculo de Viena, enfatizando
el conocimiento positivo, basado únicamente en la observación sensible
de los hechos y en la lógica formal como instrumento de análisis
(según el modelo de Russell), vuelve a restablecer la vigencia de las
tesis clásicas en teoría de la ciencia.\\[0pt]

Para los positivistas lógicos las proposiciones básicas, en las que se
expresan las observaciones de la ciencia experimental, pueden servir
de base a generalizaciones cada vez más amplias.  Intentan formular la
relación de confirmación entre una hipótesis y una proposición que
exprese una evidencia. Esta corriente surge como una reacción frente a
la retórica y las abstracciones de cierta filosofía basada en el rigor
matemático de los lógicos.\\[0pt]
Se tiene una clara imagen del estado de la ciencia en cada etapa. Se
empieza a partir de los hechos establecidos y se hace uso de los modos
de inferencia para llegar al nivel de las leyes y de las teorías. Se
puede tener una conciencia exacta tanto de significación como de
verdad, sobre el estado en que se está y sobre las relaciones de
proposiciones de cada estado y los hechos a partir de los cuales se ha
razonado.\\[0pt]

Este método no se puede generalizar a cualquier proposición que se
quiera saber si es verdadera o no. El principal fallo de esta
concepción de la ciencia fue no haber prestado atención a la historia
real de las ciencias y a los cambios conceptuales. A partir de Popper,
se trata de construir la teoría de la ciencia sobre una base más o
menos vinculada a la práctica misma de la ciencia, mejor que a partir
de una teoría general del conocimiento como había sido el caso en la
tradición epistemológica clásica.

\section{Principio de verificación y unificación del lenguaje científico. El círculo de Viena}
\label{sec:orge6e7353}
\subsection{La división del discurso}
\label{sec:org15ea999}
\subsubsection{El isomorfismo lenguaje-realidad según el \emph{Tractatus}}
\label{sec:orgd9d7653}
El \textbf{\emph{tractatus}} está escrito en forma de aforismos numerales según el
sistema decimal y contiene siete proposiciones fundamentales.\\[0pt]
Las dos primeras \emph{El mundo es todo lo que acaece} y \emph{lo que acaece, el
hecho, es la existencia de los hechos atómicos} se refieren al mundo y
a la realidad.\\[0pt]
Mientras que las cuatro siguientes son un desarrollo de la lógica y de
teoría del lenguaje.\\[0pt]
La última proposición, la enigmática frase De lo que no se puede
hablar mejor es callarse, cierra el libro marcando el límite de lo que
se puede pensar y decir (la proposición).\\[0pt]
Aunque la mayor parte del tractatus habla de la lógica y lenguaje (de
la proposición), los párrafos iniciales tratan del mundo y de la
visión metafísica del mundo en términos del atomismo lógico
(Russell).\\[0pt]
El mundo es la totalidad de los hechos y el lenguaje es la totalidad
de las proposiciones. Ambos comparten una estructura lógica común y
Wittgenstein relaciona realidad, lógica y lenguaje mediante \textbf{tres
conceptos fundamentales}: hecho atómico, figura lógica y
proposición.\\[0pt]

\begin{enumerate}
\item Hechos atómicos. El constituyente último del mundo son los
objetos o cosas, las entidades que percibimos con los
sentidos. Los objetos son simples y forman parte de los hechos
atómicos. El hecho atómico es la combinación o relación de
objetos o cosas. Éstos son la substancia de que está hecho el
mundo, su constituyente básico. Pero de las cosas del mundo sólo
podemos conocer lo que acaece (hacerse realidad). Esto es, los
hechos atómicos o simples, los hechos compuestos de simples, o
simplemente hechos, cuyo conjunto constituye la realidad.
\item La figura lógica. Paralelamente, el lenguaje opone a las cosas
del mundo nombres. A los hechos atómicos, proposiciones simples;
y a los hechos complejos proposiciones compuestas. El lenguaje
puede representar la realidad del mundo. Cuando por medio de
proposiciones describe hechos, la estructura de las proposiciones
en relación a la de los hechos, y viceversa, se preserva. Este
isomorfismo entre el lenguaje y la realidad, es posible gracias a
la participación de ambas instancias en una misma estructura
común.
\item La proposición o el signo con que expresamos el pensamiento
representa un estado de cosas (un hecho atómico). Si este es
real, la proposición es verdadera. El conjunto de todas ellas
describe el mundo. Sólo las proposiciones, y no los nombres, son
significativas y muestran la lógica de la realidad. Las
proposiciones siempre tienen sentido, aunque sean falsas, porque
siempre describen lo que acaece en el mundo. Y sólo pueden tener
sentido cuando describen lo que acaece en el mundo.
\end{enumerate}

Las proposiciones que no describen hechos carecen de sentido. Éstas
son de dos clases:
\begin{enumerate}
\item Comprende las tautologías o enunciados necesariamente verdaderos,
que nada dicen respecto del mundo. (o sus negaciones, las
contradicciones.)
\item Aquellas proposiciones que no comparten figura lógica con la
realidad que pretenden representar.
\end{enumerate}

Esto último sucede de dos maneras. Porque se la da a un signo un
sentido falso, una mala orientación, construyendo enunciados que
contienen signos carentes de significado, como sucede con las
proposiciones mal construidas o con las de carácter metafísico. O
porque apuntan a objetos que quedan fuera del mundo, trascienden el
mundo, queriendo expresar lo inexpresable, como pasa con las
proposiciones sobre ética y aquellas que quieren esclarecer el sentido
del mundo, las proposiciones metafísicas.

Así pues, sólo las proposiciones delas ciencias empíricas tienen
sentido. La lógica consta únicamente de tautologías, y toda
proposición sobre ética o metafísica es una proposición carente de
sentido. El análisis filosófico ayuda a esclarecer el sentido de las
proposiciones del lenguaje ordinario; las del lenguaje filosófico las
declara carentes de sentido.

\subsubsection{La ciencia habla de la realidad, la filosofía habla del lenguaje}
\label{sec:orgcb098d8}
El principal criterio de diferenciación consiste en afirmar que sólo
la ciencia habla con legitimidad y sentido acerca dela realidad
extra-lingüística, mientras que la filosofía sólo tiene la tarea de
esclarecer, unificar, sistematizar y analizar el lenguaje
científico. El trabajo científico produce la representación
lingüística o simbólica adecuada a la realidad. La filosofía es una
actividad de segundo orden, metalingüística, que tiene por objeto el
lenguaje y el discurso de las ciencias. Sólo excluyendo la ambición
ontológica o metafísica de la filosofía tradicional, puede asignares
la labor a la filosofía de analizar con ayuda de la lógica formal el
lenguaje científico de la ciencia.\\[0pt]

Los \textbf{enunciados referenciales} son aquellos que se refieren a objetos
que se pueden identificar. Poseen contenido y sentido. Si los objetos
referidos no son lingüísticos los enunciados son \textbf{realistas u
objetivos}. Si sí lo son, entonces se llaman \textbf{metalingüísticos}.\\[0pt]
Uno de los objetivos principales del análisis lógico del lenguaje es
descubrir a los enunciados metalingüísticos con apariencia realista y
reformularlos correctamente a fin de que pongan claramente de
manifiesto que se refieren a palabras y no a cosas.

En el neopotivismo, la wittgensteiniana concepción pictórica del
lenguaje se expresa mediante las siguientes distinciones.
\begin{enumerate}
\item Proposiciones con sentido y pseudoproposiciones. Sólo tiene
sentido un enunciado que pueda ser calificado de verdadero o
falso. La verificabilidad constituye el criterio del
sentido. Cuando una palabra posee un significado, se dice que
designa un concepto, mientras que si esta significación es sólo
aparente y en realidad no a posee, entonces es un
pseudoconcepto.\\[0pt]
Las condiciones para que una proposición tenga significado son:
\begin{itemize}
\item Que las notas empíricas de los términos incluidos en la
proposición sean conocidas.
\item Que haya sido estipulado de qué proposiciones protocolarias es
derivable la proposición a examinar.
\item Que las condiciones de verdad para esa proposición hayan sido
establecidas.
\item Que dispongamos de un método de verificación.
\end{itemize}

\item Enunciados analíticos y enunciados empíricos. Para los
neopositivistas, el principio de verificación se aplica a un
enunciado sustancialmente de dos formas.
\begin{itemize}
\item Determinando su coherencia lógica interna mediante el análisis
de la posibilidad de reducirlo a enunciados más elementales.\\[0pt]
Se actúa así con los lenguajes formales con los que la verdad
se decide sin recurrir a la experiencia porque es determinable
a priori.\\[0pt]

\item Recurriendo a la experiencia, que afecta a enunciados
directamente referidos a la realidad y a los que integran las
ciencias de la naturaleza.\\[0pt]
Son enunciados cuya verificación requiere su confrontación con
los hechos extralingüísticos a los que remiten.\\[0pt]
Es una verificación empírica y a posteriori.
\end{itemize}

Los del primer tipo no aportan ninguna información verdaderamente
que no estuviera ya comprendida en la sintaxis y la semántica del
lenguaje utilizado. Están desprovista de sentido, como las
tautologías en la lógica.\\[0pt]
En cambio, los enunciados empíricos aportan informaciones sobre la
realidad fáctica y extralingüística.\\[0pt]
En cualquier caso, todo enunciado pertenece a una categoría o a la
otra.\\[0pt]

Una distinción de esta misma distinción es la que se establece
entre leyes empíricas y leyes teóricas:
\begin{enumerate}
\item Leyes empíricas son las que pueden ser confirmadas
directamente mediante observaciones empíricas. Son leyes
acerca de hechos observables. Contienen términos que designan
hechos observables por los sentidos.\\[0pt]
Se las obtiene mediante la generalización de los resultados
de observaciones.\\[0pt]
No sólo incluyen leyes cualitativamente simples —todos los
cuervos son negros—, sino también leyes cuantitativas que
surgen de observaciones simples.\\[0pt]
Se las usa para explicar hechos observados y predecir sucesos
futuros observables.
\item Leyes teóricas o hipotéticas son las que contienen términos
que no se refieren a hechos observables. Son leyes acerca de
entidades como moléculas. átomos, electrones, etc. que no
pueden ser medidos de manera simple y directa.
\end{enumerate}

\item Enunciados que expresan juicios éticos o estéticos. Al considerar
el lenguaje básicamente informativo se determinan como ilegítimos
y sinsentido los enunciados que expresan sentimientos o
valores.\\[0pt]
Los preceptos morales sólo expresan emociones positivas o
negativas respecto de acciones y de situaciones
descriptibles. Estos usos lingüísticos no presentan interés para
la ciencia no para la filosofía.\\[0pt]

En esta posición radical en favor del discurso científico como
idealmente objetivo y universal y este desprecio por las
expresiones de la subjetividad explican la ausencia de compromiso
filosófico y ético-político de esta filosofía con la
sociedad. Sólo el desarrollo de la ciencia y del espíritu
científico y analítico, lógico y objetivo, cabe esperar un
progreso real para la humanidad. Mientras ese progreso se
realiza, nadie está obligado a tomar posición en el terreno moral
o político. Cualquier discurso filosófico o científico, que
presenta valores o normas propiamente dichas como si se tratara
de hechos objetivos o de realidades susceptibles de ser descritas
de manera verdadera o falsa y de ser conocidas, cae en la falacia
naturalista. Este sofisma consiste en la confusión entre hechos y
valores, entre lo que es y lo que debe ser.
\end{enumerate}


\subsection{El lenguaje unificado de la ciencia}
\label{sec:org1b4ee40}
El ideal de una ciencia unificada es característico del pensamiento
moderno desde Descartes, que se plantean construir una matemática
universal. El neopositivismo retoma este objetivo y lo reformula en el
plano del lenguaje. todas las ciencias han de tener en común la
utilización de un lenguaje. Se espera la superación de la diversidad
de las ciencias si se descubre o se construye ese lenguaje al que se
puedan reconducir los diferentes lenguajes científicos. La unificación
de las ciencias requiere la construcción de un lenguaje universal y
unitario de la ciencia.\\[0pt]

Carnap en \emph{La estructura lógica del mundo} (1928) defiende como
condición básica de ese lenguaje que sus enunciados básicos se
refieran a sensaciones y experiencias sensoriales y no a objetos
físicos, que sólo son construcciones hipotéticas a partir de las
sensaciones. Las cosas son elaboraciones lógicas que realizamos sobre
la base de nuestro contenidos sensoriales. Lo propio de las
proposiciones científicas debe ser la posibilidad de que puedan quedar
reducidas a un lenguaje compuesto de símbolos de los contenidos
sensoriales.\\[0pt]

El problema es que garantiza la objetividad de la ciencia ni su
exigencia universal. Se basaría en experiencias subjetivas de cada
individuo, pudiendo diferir. Los enunciados básicos de este lenguaje
no serían comunes más que en apariencia.\\[0pt]
Neurah dió un giro distinto con su planteamiento fisicalista. El mundo
está constituido por objetos que existen con independencia de mi
experiencia. A esos objetos se refieren directamente los términos y
enunciados del lenguaje científico. Construir un lenguaje básico para
la ciencia representa descubrir los enunciados elementales y los
objetos o hechos elementales constitutivos de la realidad. Luego se
podrá traducir cualquier enunciado a ese lenguaje básico relativo a
estados  procesos del mundo físico.\\[0pt]

En el seno del fisicalismo, durante el s. XX se cuestiona la
naturaleza referencial o realista de su propio lenguaje. Por otra
parte, fracasa también en su empeño por reducir fisicalistamente el
lenguaje de las ciencias humanas, como la psicología. El programa
neopositivista de unificación de las ciencias por la unificación de
sus lenguajes se queda en mera aspiración.

\subsection{La crítica de la metafísica}
\label{sec:org86cc4ea}
El análisis lógico-positivo del lenguaje responsabiliza del sentido o
bien a errores sintácticos en los enunciados, o bien a abusos
semánticos. Es decir, son enunciados sin sentido aquellos en los que
se combinan palabras que pertenecen a categorías diversas, pero que
considerados por separado o utilizados correctamente, tienen
significado. También son sin sentido los enunciados que incluyen
palabras sin una referencia determinable, como los enunciados
metafísicos.\\[0pt]

El neopositivismo reduce la metafísica a un conjunto e errores
sintácticos y de abusos semánticos del lenguaje. Desde la perspectiva
lógica, los enunciados metafísicos o bien parecen despojados de
sentido, o bien sin el sentido o el alcance que se les quiere
atribuir.
\begin{enumerate}
\item La primera diferencia estaría en la ambivalencia misma de la
palabra ser. A veces se utiliza como cópula que antecede a y se
relaciona con un predicado —yo soy el autor de este libro—,
mientras que en otras designa existencia —yo soy. Este error
resulta agravado por el hecho de que los metafísicos carecen de
una idea clara de esta ambivalencia.
\item El segundo error está en la forma que adquiere el verbo en su
segunda significación, la de existencia. Esta forma verbal
muestra ficticiamente un predicado no existente. La existencia no
es una propiedad. A este respecto sólo la lógica moderna sería
totalmente consecuente al introducir el signo de existencia en
una fórmula sintáctica tal que no puede ser referido como un
predicado a signos de objeto, sino sólo a un predicado.
\end{enumerate}

Es decir, desde la antigüedad, la mayor parte de los metafísicos se
habrían dejado seducir por la forma verbal de la palabra ser, formando
pseudopreposiciones como \emph{yo soy}, \emph{Dios es}, etc. Pero lo
característico de la actitud física sería un determinado modo de
confusión de las palabras y las cosas. La labor del neopositivista
frente a este lenguaje será reconducir sus proposiciones a
descripciones o recomendaciones concernientes al lenguaje y su uso.
La filosofía deja de ser metafísica para reducirse a metalingüística y
transformarse en crítica y análisis lógicos del lenguaje. Este giro
lingüístico debería permitir también a los filósofos entenderse,
puesto que compartirían un punto de vista común (el metalingüístico) y
una referencia común (el lenguaje).

\section{Investigación científica y desarrollo del conocimiento: Popper}
\label{sec:orgc11629a}
\end{document}