% Created 2023-05-25 dj. 17:53
% Intended LaTeX compiler: pdflatex
\documentclass[a4paper, 11pt, twocolumn, spanish]{article}
\usepackage[utf8]{inputenc}
\usepackage[T1]{fontenc}
\usepackage{graphicx}
\usepackage{longtable}
\usepackage{wrapfig}
\usepackage{rotating}
\usepackage[normalem]{ulem}
\usepackage{amsmath}
\usepackage{amssymb}
\usepackage{capt-of}
\usepackage{hyperref}
\usepackage[T1]{fontenc}
\usepackage[margin=.75in]{geometry}
\setlength\parindent{0pt}
\author{Jordi Serra}
\date{\today}
\title{Apuntes de Teoría del Conocimiento ii\\\medskip
\large III La nueva filosofía de la ciencia}
\hypersetup{
 pdfauthor={Jordi Serra},
 pdftitle={Apuntes de Teoría del Conocimiento ii},
 pdfkeywords={},
 pdfsubject={},
 pdfcreator={Emacs 28.2 (Org mode 9.5.5)}, 
 pdflang={English}}
\begin{document}

\maketitle
\tableofcontents


\section{Introducción}
\label{sec:org8509c00}
\subsection{El final del modelo clásico de cientificidad}
\label{sec:orgfe6443a}
En las últimas décadas del s. XX hubo un profundo cambio en el ideal
clásico de cientificidad. Se ha perdido el sentido lineal y
acumulativo del devenir científico y se ha introducido la acción
humana en la historia de la ciencia. La ciencia deja de ser un
lenguaje perfecto. El mito de la presunta superioridad objetiva de la
ciencia se viene abajo , al quedar de manifiesto la inserción de la
elección en la práctica científica cotidiana.\\

El ideal clásico de cientificidad estaba representado por el
neopositivismo lógico. En contraposición a éste, podemos simplificar
en tres las nociones importantes en la nueva teoría de la ciencia.
\begin{enumerate}
\item Los hechos científicos son relativos al sistema conceptual
utilizado para poner en evidencia su articulación, susceptibles de revisión a medida que se vaya modificando el sistema.
\item La evaluación de las hipótesis científicas, particularmente en
las teorías que revisten un alto grado de complejidad, implica un
tipo de apreciación que supone un largo entrenamiento, no siendo
reductible a reglas formales explícitas de inferencia. Una
reconstrucción puramente lógica de la confirmación científica
debe necesariamente resultar incompleta.
\item Teoría rivales pueden ser inconmensurables entre sí. Puede que
sus sistemas conceptuales respectivos no se corresponden
suficientemente como para hacer posible una comparación
directa. O también los valores metodológicos que se incorporan a
las teorías no son los mismos. En consecuencia, no se dispone de
una metodología que haga posible una comparación entre ellas.
\end{enumerate}

\subsection{Teoría de la ciencia y práctica científica}
\label{sec:orga855db6}
Hasta la fecha, se sostenía que las proposiciones verdaderamente
científicas debían fundarse bien en la intuición (Descartes), bien en
relaciones de observación (Locke), bien en principios sintéticos a
priori (Kant).\\

Durante el s. XIX, el auge de las ciencias atrae la atención hacia el
estudio de sus métodos de trabajo y a su historia misma, desviándose
de este punto de vista. La aparición del círculo de Viena, enfatizando
el conocimiento positivo, basado únicamente en la observación sensible
de los hechos y en la lógica formal como instrumento de análisis
(según el modelo de Russell), vuelve a restablecer la vigencia de las
tesis clásicas en teoría de la ciencia.\\

Para los positivistas lógicos las proposiciones básicas, en las que se
expresan las observaciones de la ciencia experimental, pueden servir
de base a generalizaciones cada vez más amplias.  Intentan formular la
relación de confirmación entre una hipótesis y una proposición que
exprese una evidencia. Esta corriente surge como una reacción frente a
la retórica y las abstracciones de cierta filosofía basada en el rigor
matemático de los lógicos.\\
Se tiene una clara imagen del estado de la ciencia en cada etapa. Se
empieza a partir de los hechos establecidos y se hace uso de los modos
de inferencia para llegar al nivel de las leyes y de las teorías. Se
puede tener una conciencia exacta tanto de significación como de
verdad, sobre el estado en que se está y sobre las relaciones de
proposiciones de cada estado y los hechos a partir de los cuales se ha
razonado.\\

Este método no se puede generalizar a cualquier proposición que se
quiera saber si es verdadera o no. El principal fallo de esta
concepción de la ciencia fue no haber prestado atención a la historia
real de las ciencias y a los cambios conceptuales. A partir de Popper,
se trata de construir la teoría de la ciencia sobre una base más o
menos vinculada a la práctica misma de la ciencia, mejor que a partir
de una teoría general del conocimiento como había sido el caso en la
tradición epistemológica clásica.

\section{Principio de verificación y unificación del lenguaje científico. El círculo de Viena}
\label{sec:orgeb30ab1}
\subsection{La división del discurso}
\label{sec:org4f1479f}
\subsubsection{El isomorfismo lenguaje-realidad según el \emph{Tractatus}}
\label{sec:org7bd20f3}
El \textbf{\emph{tractatus}} está escrito en forma de aforismos numerales según el
sistema decimal y contiene siete proposiciones fundamentales.\\
Las dos primeras \emph{El mundo es todo lo que acaece} y \emph{lo que acaece, el
hecho, es la existencia de los hechos atómicos} se refieren al mundo y
a la realidad.\\
Mientras que las cuatro siguientes son un desarrollo de la lógica y de
teoría del lenguaje.\\
La última proposición, la enigmática frase De lo que no se puede
hablar mejor es callarse, cierra el libro marcando el límite de lo que
se puede pensar y decir (la proposición).\\
Aunque la mayor parte del tractatus habla de la lógica y lenguaje (de
la proposición), los párrafos iniciales tratan del mundo y de la
visión metafísica del mundo en términos del atomismo lógico
(Russell).\\
El mundo es la totalidad de los hechos y el lenguaje es la totalidad
de las proposiciones. Ambos comparten una estructura lógica común y
Wittgenstein relaciona realidad, lógica y lenguaje mediante \textbf{tres
conceptos fundamentales}: hecho atómico, figura lógica y
proposición.\\

\begin{enumerate}
\item Hechos atómicos. El constituyente último del mundo son los
objetos o cosas, las entidades que percibimos con los
sentidos. Los objetos son simples y forman parte de los hechos
atómicos. El hecho atómico es la combinación o relación de
objetos o cosas. Éstos son la substancia de que está hecho el
mundo, su constituyente básico. Pero de las cosas del mundo sólo
podemos conocer lo que acaece (hacerse realidad). Esto es, los
hechos atómicos o simples, los hechos compuestos de simples, o
simplemente hechos, cuyo conjunto constituye la realidad.
\item La figura lógica. Paralelamente, el lenguaje opone a las cosas
del mundo nombres. A los hechos atómicos, proposiciones simples;
y a los hechos complejos proposiciones compuestas. El lenguaje
puede representar la realidad del mundo. Cuando por medio de
proposiciones describe hechos, la estructura de las proposiciones
en relación a la de los hechos, y viceversa, se preserva. Este
isomorfismo entre el lenguaje y la realidad, es posible gracias a
la participación de ambas instancias en una misma estructura
común.
\item La proposición o el signo con que expresamos el pensamiento
representa un estado de cosas (un hecho atómico). Si este es
real, la proposición es verdadera. El conjunto de todas ellas
describe el mundo. Sólo las proposiciones, y no los nombres, son
significativas y muestran la lógica de la realidad. Las
proposiciones siempre tienen sentido, aunque sean falsas, porque
siempre describen lo que acaece en el mundo. Y sólo pueden tener
sentido cuando describen lo que acaece en el mundo.
\end{enumerate}

Las proposiciones que no describen hechos carecen de sentido. Éstas
son de dos clases:
\begin{enumerate}
\item Comprende las tautologías o enunciados necesariamente verdaderos,
que nada dicen respecto del mundo. (o sus negaciones, las
contradicciones.)
\item Aquellas proposiciones que no comparten figura lógica con la
realidad que prentenden representar.
\end{enumerate}
\end{document}